%!TEX root = ../../dissertation.tex
%%%%%%%%%%%%%%%%%%%%%%%%%%%%%%%%%%%%%%%%%%%%%%%%%%%%%%%%%%%%%%%%%%%%%%%%%%%%%%%%
\chapter{Measuring and Modeling Reliable Video Streaming}
\label{chap:streaming}

The Web would not have seen that big an increase in traffic if it were not for the tight integration of video streaming into every browser. 
% Hmm, hardly true for 1991.
Most forms of today's Web-based video delivery take advantage of \gls{HTTP} and \gls{TCP} to transport video. This is a completely different approach to what was used and was traditionally understood as video streaming in the past.

Streaming itself can be conducted in many ways, resulting in an ever-increasing number of protocols. Furthermore, the current boom in smartphones creates an increasing plurality of access network technologies. Each of them exhibits characteristic \gls{QoS} properties. These are typically:

\begin{itemize}
	\item The \textit{bandwidth}, or the maximum throughput a user can achieve, which is always limited by at least one link, serving as the bottleneck. In most cases this will be the access link, but can also be any other link in times of high load. Bandwidth on an access medium can also be shared between all of its participants as is the case with any radio technology or cable Internet access.

	\item The \textit{delay} is the time data travels between a sender and a recipient. The term \textit{jitter} is used  for the delay variation and occurs, for example, when successive packets travel on different routes or through different radio receivers during mobility events.
% Nicht zu vergessen Puffer und resultierende (möglicherweise bloatige) Effekte

	\item \textit{Loss} occurs when data packets do not reach the target. One source of loss can be an imperfect physical medium that flips some bits in a data packet. The responsible higher protocol layer will recognize this and drop the packet or repeat the request.
\end{itemize}

Video streaming needs to cope with all of these circumstances and still work well. This chapter investigates the model Web streaming uses. It differs significantly from models for traditional streaming, which are mostly specific to a single protocol.

The presented method rather aims to evaluate the performance by capturing generic behavioral patterns of streaming mechanisms from the perspective of a streaming application. Specifically, the model is based on the level of the playback buffer, which is a common attribute to any media playback. Thus, it can consider any network and playback behavior while maintaining flexibility with regards to the actual streaming server implementation, the network, and also the protocol stack.

After defining an appropriate performance metric and researching various playback strategies, the model is implemented in a network emulation testbed. A measurement campaign is then conducted, testing the influences of various \gls{QoS} settings and different playback strategies. 

The work in this chapter was originally based on two publications, \cite{metzger2011delivery} and \cite{6229739}, but has been extended to include a broader categorization and modeling effort.


%%%%%%%%%%%%%%%%%%%%%%%%%%%%%%%%%%%%%%%%%%%%%%%%%%%%%%%%%%%%%%%%%%%%%%%%%%%%%%%%
%!TEX root = ../../dissertation.tex
%%%%%%%%%%%%%%%%%%%%%%%%%%%%%%%%%%%%%%%%%%%%%%%%%%%%%%%%%%%%%%%%%%%%%%%%%%%%%%%%
\section{Streaming Definition and Classification}
\label{c3:sec:background}

Before diving into the model some technical groundwork has to be laid. The protocols commonly used in the past and present are described, and a classification scheme is set up. This is followed by a summary of other work related to this approach.


%%%%%%%%%%%%%%%%%%%%%%%%%%%%%%%%%%%%%%%%%%%%%%%%%%%%%%%%%%%%%%%%%%%%%%%%%%%%%%%%
\subsection{Video Streaming Definition} 

Any digitally stored \textbf{video} consists of a number of frames, organized into variable-sized groups of pictures, and audio samples which are played in sequence. Frames, single images of the video, do not only make use of typical spatial image compression mechanisms but encompass also temporal motion compensation, creating a dependency between frames. Videos can be encoded with a bit rate that is constant or variable over time. Typically, a variable bit rate encoding is chosen as these schemes offer a much higher compression rate. To correctly display a frame, all previous frames in a group need to be present. 

\textbf{Streaming}, or to be more precise video streaming, is the process of playing a video while it is still being transmitted over a medium. 
% Wie glaub ich schon zuvor angemerkt, "play one part of a video while other parts are still being transmitted".
As there is no need to have a file stored locally, received frames are typically put in a buffer to be played at the correct time. The amount of buffered video depends on the allocated buffer size as well as the video bit rate, and the transmission bit rate. It can be controlled by the time offset between receiving the first frame of a video and actually playing it.


%%%%%%%%%%%%%%%%%%%%%%%%%%%%%%%%%%%%%%%%%%%%%%%%%%%%%%%%%%%%%%%%%%%%%%%%%%%%%%%%
\subsection{Streaming Classification}

Video streaming is a broad term covering a wide spectrum of applications as well as possible implementations. The following distinction criteria can serve to break down and classify this field.

\subsubsection{Video Source}
The first criterion is the source of the video transmission, with the two major sources being a stored file or a live source. Stored video can be streamed and played at any point in time. Live sources, on the other hand, are transmitting only at a fixed point in time. Depending on the type of content the timeliness of playback may also be important (imagine watching a game that is being played right now).


\subsubsection{Adaptivity of Content}
Video streaming can also be distinguished based on its adaptivity. In the simplest case there is no adaptivity present and the video is available in only one bit rate (which may still be a variable bit rate). 

But there are cases where an adaptation of the bit rate would be helpful, for example to accommodate for a client's screen size. Usually, adaptation is used to tune the video stream to the currently available connection bandwidth. Adaptation can be achieved in two ways. Either by preparing and storing several encoding levels beforehand, or by encoding the video on-the-fly for a specific target. While the latter approach has a much higher adaptation capability it cannot be precomputed, and computational effort scales linearly with the number of  number of clients (as opposed to a constant, one-off compute time effort for stored quality levels).

Adaptation also increases the necessary amount of control and information exchange. In a non-adaptive context, streaming would only require a single interaction to start the streaming while any single adaptation adds another set of interactions to change between quality levels.


\subsubsection{Location of Control}
Another matter is the location of control for a stream, with several possible ways to choose from. Horizontal and vertical control can be distinguished.
% Da muss man sich aber viel State merken, bis man die Def.s zu horiz und vert beide gelesen hat! Hier schon einen Hinweis einfügen? H --> nodes in the network, V --> layers in the stack?

In a horizontal direction control can be placed either at the streaming server, the streaming client, or possibly somewhere in the network path in between.

A controller located at the client side typically means that the video player itself is in control of the streaming process. The player starts the streaming and adjusts its requests to the server based on the player's needs. In this situation the server can be very lightweight as no decision logic needs to be present there. This is called \textbf{pull-based} streaming.

Control can also be placed in the server. This is called a \textbf{push-based} approach as video data is pushed to the recipient. For this kind of control to work properly, state has to be kept imposing a certain memory overhead. The amount of state and processing of state can become a limiting factor for large streaming servers. 
Contrary, pull-based streaming usually does not require much or any state at all at the server. % Quasi-redundant, siehe voriger Absatz

Control information may also need to be exchanged to communicate the state between the two endpoints. This can happen either explicitly through the exchange of signaling messages, or implicitly by drawing conclusions on another participants' resources and behavior, for example through other protocols in the stack. Push-based protocols usually employ an explicit state exchange.

While not being able to control everything about streaming, the network may still be able to influence or manipulate an ongoing video stream. (Non-)transparent proxies come to mind, which could intercept streaming requests and redirect them to another server located in the proximity of the requesting client. Alternatively, a network could explicitly expose its \gls{QoS} metrics for streaming applications to optimize themselves, or these application could send bandwidth reservation requests to the network.

Additionally, control can be distributed vertically to different positions in the protocol stack. While streaming is usually conducted through a dedicated application layer protocol or directly through an application's behavior, portions of control functions can also be offloaded to deeper layers. A typical example would be the use of \gls{TCP} for reliable streaming as described in the next paragraphs.


\subsubsection{Reliability of the Underlying Transport Protocol}
A major differentiation can also be made based on the reliability of streaming. In the simplest case, streaming can act similar to a simple file download and just progressively download the video file in question while already starting to display the video contents. This is conducted by using \gls{TCP} as a transport protocol, guaranteeing that no packet is lost in the process. \gls{TCP} does this by retransmitting packets it thinks are lost at the cost of added latency and reduced throughput during retransmission. This reliability can however also cause the progress of the whole video stream to stall if video data does not reach the client in time before its playback buffer is depleted, and therefore result in a perceptible loss of quality. This situation can be alleviated or even avoided by carefully planning the playback process and the buffering behavior.

On the other side stand streaming protocols that base themselves on \gls{UDP}, which offers no reliability features like \gls{TCP} and just sends out packets as-is. When packets are lost, the video can still progress but parts of the video output may be distorted or lost. Additionally, unreliable streaming protocols must take over other control features that would otherwise have been taken care of \gls{TCP}, e.g., the adherence to an allotted or fair share bandwidth and congestion control. Otherwise, a high usage of this protocol could lead to a scenario of congestive collapse as described in~\cite{rfc896}.

Transport protocols that offer congestion control and no reliable delivery might be a desirable middle ground between these two extremes. \gls{DCCP}~\cite{kohler2006designing} is an example for such a compromise and might prove beneficial for the streaming process.


%%
\subsubsection{Multiplexing of Delivery}
Finally, the number of targets of an individual video stream can also differ. A stream is unicast if the control loop is exactly between one sender and one recipient. Servers can still support multiple unicast streams at once, they are just completely independent of each other. A multicast stream is simultaneously sent to a group of recipients, stream control is established at the sender for the whole group. Therefore, multicasting is always using a push-based approach to control.


%%%%%%%%%%%%%%%%%%%%%%%%%%%%%%%%%%%%%%%%%%%%%%%%%%%%%%%%%%%%%%%%%%%%%%%%%%%%%%%%
\subsection{Survey of Protocols}

With these classification criteria at hand, an investigation for the motifs that are present in existing protocols can now be undertaken. The section largely describes \gls{rtp} and compares it with \gls{HTTP}-based approaches, including \gls{DASH}, while also mentioning some other, proprietary, streaming protocols.


%%
\subsubsection{\texorpdfstring{\acrshort{rtp}}{rtp} and Related Protocols}

\gls{rtp}~\cite{rfc3550} is typically used in conjunction with its sister-protocol \gls{rtcp} and often also employs \gls{RTSP}~\cite{rfc2326}. According to literature, they are the classic approach to video streaming\footnote{For example, refer to \cite[p.~589ff]{kurose2008computer} or \cite[p.~426ff]{peterson2007computer}}.
The protocol suite employs a \textit{push-based approach} with the \gls{rtp} server application in full control of the streaming process. Control and information exchange is conducted out of band through \gls{RTSP} and \gls{rtcp}. Therefore, multicast is also easily possible with \gls{rtp} but not mandatory.

\gls{rtp} has also no inherent adaptivity nor reliability mechanisms. Neither does it conduct congestion control on its own. Moreover, \gls{rtp} generally runs on top of \gls{UDP}, which also does not provide congestion control. These must be provided by the server-side application implementation, if necessary. In case of multicasting the potential to conduct transport adaptations is very limited, as the server has to take all the recipients into consideration for its decisions.


\paragraph{\gls{rtp}}

\gls{rtp} itself provides just the packet format and header for the transport of the actual multimedia data. Every stream type is transported in a separate session. This includes the presence of both video and audio, which must then be synchronized to each other. Each session uses its own \gls{UDP} source-destination port pair.

The \gls{rtp} specification itself defines only the most basic packet header, with several additional specs describing dedicated profiles for various content types. For today's prevalent MPEG-4 protocols, including H.264, multiple profiles, defined in \cite{rfc3640,rfc6184,rfc6416}, and with this many ways to embed video into \gls{rtp} packets, are available. Common to all is the variable-size \gls{rtp} header of at least \SI{16}{\byte}. Video codecs may embed their own sub-structure inside the packet. For example, if dealing with an MPEG-4 \gls{ES}, the payload may contain one or more \glspl{AU}.


\paragraph{\texorpdfstring{\acrshort{rtcp}}{rtcp}}

\gls{rtcp} is used to exchange feedback and control information between receivers and sender and vice versa. These sender and receiver reports are transmitted on a separate \gls{UDP} connection at small intervals and are scaled in such a way that the bandwidth should not exceed \SI{5}{\percent} of the actual stream's bandwidth. The reports will include statistics related to lost packets as well as the packet delay and its variation. Based on these, a sender can adjust its streams to fit the current conditions. Likewise, a receiver may tune its video buffering behavior or may even switch stream sources.


\paragraph{Stream Initiation}

\gls{rtp} and \gls{rtcp} on their own provide no means to discover, initiate, and control the streaming process and have to rely on additional protocols. \gls{RTSP} is one of these, sitting atop of either \gls{UDP} or \gls{TCP}. It provides a set of commands the client can issue to a streaming server to control a stream and the streaming state at the server.

In case of multicasting, stream management can also be conducted directly by joining predetermined multicast groups through the use of \gls{IGMP}~\cite{rfc4604} without the need for \gls{RTSP}. However all intermediary routers have to support this mode. Therefore, it is usually only seen in closed networks where the whole network infrastructure is owned by a single organization. For example, this scheme is often employed by \gls{IPTV} providers in their access network.

\gls{rtp} is also used extensively in conjunction with other protocol suites, including \gls{SDP}~\cite{rfc2327} and \gls{SAP}~\cite{rfc2974} for stream discovery or in realtime communication protocols such as \gls{SIP}~\cite{rfc3261} and \gls{XMPP}~\cite{rfc6120,rfc6121} with the Jingle multimedia session control extension\footnote{The draft standard is available at \url{http://xmpp.org/extensions/xep-0166.html}.}.

The prevalent presence of middleboxes in the Internet also poses issues with \Gls{rtp}'s requirement of several simultaneous \gls{UDP} streams. \gls{NAT} nodes are especially problematic because of the difficulty to forward incoming \gls{UDP} packets to the destined host. This can be partially circumvented by using \gls{NAT} traversal techniques like \gls{STUN}~\cite{rfc5389} or \gls{ICE}~\cite{rfc5245}, sometimes though unreliably.


%%
\subsubsection{\texorpdfstring{\acrshort{HTTP}}{HTTP} Streaming}

When compared to \gls{rtp}, \gls{HTTP} streaming represents a much less specialized approach by only reusing existing general purpose protocols. The \gls{HTTP}/1.1~\autocite{rfc2616} application layer protocol is the basis of the Web and is a request/response protocol mainly to retrieve and pull files from and to a remote location.\footnote{In June of 2014 the original \acrshort{RFC} has been obsoleted and replaced with a new set of specifications in preparation of \gls{HTTP}/2. The specifications are: \autocite{rfc7230,rfc7231,rfc7232,rfc7233,rfc7234,rfc7235,rfc7236,rfc7237,rfc7238,rfc7239}} The protocol is stateless for the server, requests are fully independent of each other and will be responded to only with the provided metadata.\footnote{State can still be achieved through other paths, like cookies, but this is out of scope.} This holds true even when more than one request is sent over the same \gls{TCP} connection, which can be achieved using persistent \gls{HTTP} connections. Additionally, requests can be sent over one connection without waiting for the answer of the previous request. This is called pipelining and can reduce the round-trip time delay between two consecutive requests.

\gls{HTTP} can also be easily exploited for video streaming. The file to be retrieved should of course contain video and all frames have to be stored sequentially. If there are separate streams present in file, most commonly at least video and audio, they must be interwoven. Video metadata necessary for the start of the playback process, which typically includes sub-stream information and codec parameters, needs to be positioned at the beginning of the file or at least before the position in the file where its needed. Alternatively, streams can also be stored in separate files, potentially simplifying the file structure. However, this increases the complexity of synchronizing both streams at the video player.\footnote{Stream synchronization issues are for example present in \gls{rtp}.}

The actual streaming is controlled completely by the player application at the client. This player simply has to issue a \gls{HTTP} `GET'-request for a video file stored at the Web server. The file can already be read during the transmission process and extracted video data will be put in the player's buffer. If there is enough video in the buffer, playback can be started. The complexity in this process comes from the need to keep track of the amount of video in the buffer and to avoid to run out of buffered data at any point during playback. Approaches to this task will be explained in detail in Section~\ref{c3:sec:modeling}. \gls{HTTP} also allows so-called range requests, which allows to download only certain portions of a file, indicated by the Byte position. Streaming players can exploit this to enable skipping to certain positions. This again needs metadata to correctly infer the byte position in a file from the video playback position. Otherwise, the range requests have to be guessed. 


%%
\paragraph{Reliability and Adaptivity}

\gls{HTTP} uses \gls{TCP} as transport protocol, which has implications to \gls{HTTP} streaming and distinguishes itself significantly from \gls{rtp}/\gls{UDP}-based approaches. \gls{TCP}'s three large features are arguably reliability, congestion control, and flow control.

Reliability means that at the transport layer and above no packets are lost and file requested by a \gls{HTTP} application will always be transmitted in full to the client (as long as the connection is not completely interrupted). \gls{TCP}'s sender side detects lost packets either by timeouts waiting for the corresponding acknowledgment or, preferably, through duplicate acknowledgments of previous packets. If either of this happens, the lost packet is retransmitted, causing a noticeably increase and variation in latency. But this also means that the transmission of all consecutive packets has to wait for the lost packet to be retransmitted, causing \gls{HOL} blocking. On links with high loss the transmission could be stalled to such a degree that the incoming bitrate is lower than the bitrate of the playing stream, draining the buffer until it runs empty. 
Reliability for video streaming means that the whole video has to be fully played in sequence and no implicit quality adjustments based on the current bandwidth are possible.
% Ich verstehe diese Interpretation von Reliability for video streaming nicht so wirklich. Gut, Du kannst keine Segmente überspringen (außer Dein TCP-Stack lügt den Server an). Ansonsten sind doch Range Requests das Gegenbeispiel zu "has to be fully played in sequence", oder nicht?

In addition, \gls{TCP} also employs congestion control and avoidance mechanisms. While the sending rate of an \gls{UDP} application is completely controlled by the application's logic, \gls{TCP} detects and throttles the transmission to its fair share of the current bottleneck link bandwidth. This can also cause the transmission rate to become lower than the video bitrate. 

The third transmission rate influencing mechanism is flow control. The receiver (and especially also the receiving application) can notify the sender how much data it can receive in the next time window, and thus can throttle the transmission rate itself. This is an important method of control for the player. Usually, \gls{TCP}'s transmission fair share rate is expected to be much higher than the stream's video bitrate. While this makes sense for a simple file download to finish it as soon as possible, this behavior is unwanted for streaming. Rather, one wants to match the stream bitrate, with a bit of additional headroom to compensate for rate variations, to keep the playback buffer size in certain bounds that neither overwhelm the receiving device nor should the buffer be in danger of running empty. 

\begin{figure}[htbp]
	\centering
	\includegraphics[width=1.0\textwidth]{images/streaming-transfer-modes.pdf}
	\caption{Comparison of several possible streaming transmission modes depicting the timing of the sent packets (source:~\cite{ma2011mobile}).}
\label{c3:fig:streamingtransfermodes}
\end{figure}

The first of the alternatives to achieve control over the playback buffer using \gls{HTTP}-streaming is to appropriately size \gls{TCP}'s flow control receive window by the application. 
Alternatively, the \gls{HTTP}-server can also manually throttle the download process, through various pacing strategies. The second and third transmission diagrams in Figure~\ref{c3:fig:streamingtransfermodes} depict two possible strategies compared to a regular \gls{HTTP} file transmission in the first transmission diagram. A third way is to either partition the stream file into smaller evenly-spaced segments that have to be requested independently, or to use the aforementioned range requests on the stream's file. Through this, the receiving application can delay the request of new ranges or segments so that it matches a targeted bitrate over a longer timeframe.

These mechanism, however, can also result in a very bursty block-like transmission, a so-called ON-OFF pattern, and can cause undesirably interactions with \gls{TCP}'s flow and congestion control mechanisms as observed in \cite{alcock2011afcyt}. Overall, stretching out the transmission may reduce server load spikes and the required buffer size on the client device but also makes streaming more vulnerable to insufficient network \gls{QoS} parameters. These specific approaches to pace to a target rate can generally be subsumed under the term \textit{Application Layer Flow Control}, which is also being implemented by some Web streaming services, e.g. YouTube~\cite{metzger2011delivery}.

These flow control mechanisms only adapt the transmission rate to the stream's bit rate but not vice versa. Having access to the video in different bitrates may be desirable for many use cases, especially for reacting to changing network conditions. Take a vertical handover from a high bandwidth WiFi network to a \gls{UMTS} network with a much lower throughput as an example. 

Quality adaptation with \gls{HTTP} streaming is generally achieved through the described range request or file segmentation mechanisms. For both approaches, multiple versions of the file or the segments have to be generated in different encoding quality levels. Also, the video file format needs to be able to support switching the stream and have an index to correlate the video files with their quality level and temporal position. A longer overview is for example given in \cite{ma2011mobile} or \cite{watching-video1}.

Several formal \textbf{adaptive streaming} protocols have been standardized or are in the process of standardization. \gls{HLS}~\cite{pantos2011livestreaming} defines a playlist format to be stored separately on a server that links to all available stream variants and segments thereof in sequence. \gls{DASH}~\cite{Stockhammer:2011:DAS:1943552.1943572} is a \acrshort{ISO}/\acrshort{IEC}~\cite{iso-iec-23009-1} and 3GP-DASH~\cite{3gpp.26.247} standard. 
% Ist gemeint: "3GPP standard (cite 3GP-DASH)"?
Herein, all video segments are gathered in an alternative \acrshort{XML}-based presentation scheme. Using stored video for \gls{HTTP} streaming is more appropriate due to the file-based nature of \gls{HTTP}. But the streaming protocols have also been successfully employed for live content, both \gls{HLS} as well as \gls{DASH} support this.

Other than the two streaming endpoints, the network can also, to a degree, control parts of \gls{HTTP} streaming. \gls{HTTP} allows to have forward and reverse proxies placed in the transmission's path. Proxies are usually not employed for video streaming but can potentially alter and adapt the stream to the needs of certain clients. However, an adaptation inside the network usually requires much greater efforts with less effects than at the endpoints.

Through \acrshort{DNS} video stream requests can also be redirected to a server instance chosen by the stream source. The content provider has to internally distribute the stream files to all the caches that are advertised through \gls{DNS}. Caches are usually placed in close vicinity of potential receivers. This creates a so-called \gls{CDN} avoiding both long transmission paths through the Internet and the single server bottleneck. \glspl{CDN} can also be used to achieve a multicast-like effect, which \gls{TCP}-based streaming cannot provide itself. Only traffic on very few links close to the recipient has to be replicated. However, the the access links of stream receivers are typically separate entities anyway, so even actual multicast-enabled streaming would not save any bandwidth there. For an exemplary investigation of YouTube's \gls{CDN} structure refer to \cite{rafetseder2011explyt}.

Currently, \gls{HTTP} is in the process of receiving major remodeling with the efforts of WebSocket\footnote{\url{http://www.websocket.org/}}~\cite{rfc6455}, SPDY~\cite{google2011SPDYdef,google2010SPDYwp}, and ultimately also HTTP/2~\cite{http20draft}. All three improve the flow multiplexing capabilities of \gls{HTTP} and allow the server to initiate transmissions on its own. This enables more control possibilities for the server and can improve any segment-based adaption scheme as these segments do not need to be requested anymore but could just be pushed by the server at a convenient time.

%%%%%%%%%%%%%%%%%%%%%%%%%%%%%%%%%%%%%%%%%%%%%%%%%%%%%%%%%%%%%%%%%%%%%%%%%%%%%%%%
\subsubsection{Other Approaches and Classification Matrix}

There are also other proprietary and standardized streaming systems usually tailored to specific requirements and applications. \gls{MBMS}~\cite{3gpp.22.146,3gpp.22.246} is a \gls{3GPP} specification for multicasting multimedia traffic in the mobile network architecture. The explicit control structure of protocol suites like \gls{MBMS} and also \gls{IMS}~\cite{3gpp.23.228} weaves application and network layer tightly together. This theoretically allows for an improved streaming performance at the cost of universally applicable behavior. \gls{RTMP}~\cite{rtmpspec} is a proprietary streaming protocol which in the past has seen widespread use through its implementation in the Adobe Flash Player plugin in Web browsers. 

Also leading a niche existence are \gls{P2P} based streaming approaches. In \gls{P2P} there is no explicit server. Instead, connections are made and stream data is exchanged between equal hosts, avoiding a centralized server's bottleneck. \gls{P2P} streaming is used for example, in Tribler\footnote{\url{https://www.tribler.org/trac}} and Zattoo\footnote{\url{http://zattoo.com/int/}}. Table~\ref{c3:tab:streamingclassification} attempts to summarize all the protocols of interest to this research and apply the proposed classification criteria to them.

\begin{table}[htb]
\caption{Streaming protocol classification matrix.}
\label{c3:tab:streamingclassification}
	\centering
	\begin{tabu}{X[0.8]XX[1.2]XXXX[0.85]} 
	\toprule
	\textbf{Protocol} & \textbf{Vertical Location of Control} & \textbf{Horizontal Location of Control} & \textbf{Reliable Transport} & \textbf{Video Type} & \textbf{Adaptivity} & \textbf{Multicast} \\ 
	\midrule
	\gls{rtp} & out-of-band, application layer protocol & server-side and limited intermediary (translators and mixers) & unreliable (\gls{UDP}) & low delay live streaming & server-side adaptation (transcoding) & using \gls{IGMP}\\
	simple \gls{HTTP} & in-band, streaming application & client-side & reliable (\gls{TCP}) & stored, not live & none & emulated through \gls{CDN}\\
	adaptive \gls{HTTP} (e.g. \gls{DASH}) & in-band, streaming application & client-side & reliable (\gls{TCP}) & stored and near-live & client-side with file segmentation & emulated through \gls{CDN}\\
	\bottomrule
	\end{tabu}
\end{table}
% Die schaut ja fürchterlich aus. Querformat und Zwischenstriche waren aus?

%% reliable streaming
	%Web-based video streaming in general
	%Web, Flash, HTML5
	%relevance for mobile and future Internet
	%Drawbacks (no streaming, scaling, signaling, etc)
	%Internet Load and Video delivery performing with load
	%Example of YouTube
	%applicable quality metrics (normal metrics don't apply, no video quality scaling, observable only initial buffering time, stalls during playback, number, length, frequency thereof)


%%%%%%%%%%%%%%%%%%%%%%%%%%%%%%%%%%%%%%%%%%%%%%%%%%%%%%%%%%%%%%%%%%%%%%%%%%%%%%%%
%\subsubsection{Network Stack Layers and Streaming}

%In network layering models, it is often assumed that the layers are independent (or at least strongly decoupled) from, and only present narrow interfaces to, one another. From a conceptual point of view, media streaming is a process governing the application layer. Thus, the application and its behavior might be thought to dominate the overall streaming process and associated quality. In this Section, we will show that this is not necessarily the case.

%Figure \ref{c3:fig:timescales} overviews the approximate time scales on which activities on different layers may take place, spanning a remarkable range of twelve orders of magnitude. Multiple layers might implement the same or similar functionality, e.g. flow control in the application and on transport layer, resulting in nested control loops, which might be coupled due to the timing constraints. 

%\begin{figure}[htbp]
%	\includegraphics[width=\textwidth]{images/timescales.pdf}
%	\caption{Relevant time scales in the layers of the stack}
%	\label{c3:fig:timescales}
%\end{figure}


%\subsubsection{Network Layer}

%As seen in Figure \ref{c3:fig:timescales}, the time constants found in different network implementations range from nanoseconds (for Gigabit Ethernet) to seconds (for UMTS and \gls{LTE}/\gls{SAE} wireless networks), depending on the technology used. This also influences the achievable round-trip time across such networks, which directly affects the performance of higher-layer protocols: \gls{IP}, \gls{ICMP}, \gls{UDP}, \gls{TCP}, and subsequently all application-layer protocols are all subject to these timing constraints.

%In the case of wireless networks, typical effects of wireless connectivity relating to physical phenomena like fading and interference come into play. Flaky radio connectivity is a major source of packet loss and excessive delay. Certain cellular mobile technologies like \gls{UMTS} and its evolutions implement loss concealment themselves, confounding IP's assumption of a host-to-network layer lacking guaranteed delivery. Other peculiarities of cellular mobile networks include a \gls{MTU} opaque to IP, and delay variances as functions of packet sizes \cite{Arlos10} and radio access technologies \cite{laner2011dissecting}.


%Then, the technological progress enables both handsets and the network to become faster: Comparing the delay budgets given by x for \gls{UMTS} (2005) and y for \gls{HSDPA} R99 (2009) respectively, it is seen that the delay caused by processing on the mobile terminals decreased by a factor of 30, and that the core network has become faster by an order of magnitude as well. \cite{svoboda2006composition} has additional information on the delay budgets per network entity, varying the packet size as the parameter.



% \begin{figure}[htbp]
% \centering
% \includegraphics[width=0.8\textwidth]{images/nif.pdf}
% \caption{Theoretical information exchange paths between streaming partners.}
% \label{c3:fig:nif}
% \end{figure}

%One future trend is said to be an increase in the required communication confidentiality and authentication. One of the goals might be to enable full end-to-end encryption on the transport level of the network. This could be achieved either by providing an encrypting alternative to TCP, e.g. CurveCP \cite{curvecpwww} and TCPCrypt \cite{tcpcrypt}, or by using HTTPS and moving other functionality further up the stack.


% \subsubsection{Video Delivery Architecture}
% \label{c3:sec:videodeliveryarchitecture}
% Large Internet sites are not hosted at one central site anymore, but are usually served through geographically distributed entities forming a load-balancing structure. Such load balancing mechanisms have a long history on the Web, e.g. in the form of mirror servers a user can select manually.

% In today's Content Distribution Networks (CDN), a much larger number of mirror servers is available, and selection of a server is no longer carried out explicitly by the user, but implicitly through DNS: Content is addressed using URLs (\texttt{http://somedomain/somepath} in its simplest form), and the CDN's DNS servers are configured to resolve certain domain names to different IP addresses, depending on where the query originated.

% To get an insight into the structure of YouTube's content distribution network, we undertook a two-step measurement campaign \cite{rafetseder2011explyt}. First, we downloaded and manually parsed the HTML code served by YouTube's web servers. We could thus enumerate and learn about the structure of domain names in the system. The most relevant category of domain names for our purposes takes the form of \texttt{v$\alpha$.lscache$\beta$.c.youtube.com}, where $0<\alpha<25$ and $0<\beta<9$. Not all permutations of names are found at all times. We also noted that there are hostnames that seem to point to geographical locations, but have not succeeded so far in exhaustively mapping those two types of names.

% The second part of our campaign consisted of active measurements on forty distributed computers (part of the \textit{Seattle} Internet testbed\footnote{\url{https://seattle.poly.edu/html/}} \cite{Cappos:2009:SPE:1508865.1508905}) for over 600 hours. We learned that the frontend web server name, \texttt{www.youtube.com}, resolves to multiple IP addresses per geographical location of the probing host which are mostly disjoint from sets of addresses found on other hosts. The number of frontend IP addresses also changes over time, e.g. to account for load variations such as load increases during the evening hours in the hosts' time zones. The actual video cache servers only have one IP address per name and location each, but sometime this address is seen to change during the day. 

% When looking at the resolved addresses per frontend server and time zone, two interesting time-dependent scaling effects can be seen: First, servers become reachable or vanish in a coordinated manner controlled by the time of day, i.e. in a 24 hour pattern. We speculate this provides a gain in efficiency for the overall system to turn on parts of the resource pool for load balancing only when there is demand.

% The second type of effect occurs much more seldom. It stretches out over multiple days and is best described as follows: A new block of server IP addresses is made available in addition to the existing ones. After a few days of parallel operation, a previously active block is taken out of service. The new block continues to serve. Comparison measurements  performed in parallel show that this switch-over between IP address blocks has a positive effect on the latency to the servers, as the latency to non-YouTube destinations show no improvements at all.

%%%%%%%%%%%%%%%%%%%%%%%%%%%%%%%%%%%%%%%%%%%%%%%%%%%%%%%%%%%%%%%%%%%%%%%%%%%%%%%%
%!TEX root = ../../dissertation.tex
%%%%%%%%%%%%%%%%%%%%%%%%%%%%%%%%%%%%%%%%%%%%%%%%%%%%%%%%%%%%%%%%%%%%%%%%%%%%%%%%
\section{Related Work}
\label{c4:sec:relwork}

The investigations conducted in both this and the subsequent chapter do not fall strictly into an existing research category but instead aim to provide diverse insights into the control plane from the perspective of the core network. Nonetheless, a selection of publications from the tackled fields is collected here and the interesting aspects for this work are noted. In the following sections the related work is divided into four distinct fields.

Work in the first and second sections evaluate properties of the mobile network and its traffic. They are distinguished in their approach to the investigation, as the first group uses active measurements from mobile devices or conclude from other sources of traffic whereas to the other one has access to passive measurements from inside a \gls{3G} mobile network. Publications from the third category can be generally subsumed under the term \textit{traffic modeling} and may not be specific to cellular networks. The final field concerns itself with investigative work conducted by the responsible standardization and organizational bodies themselves, i.e., the \gls{3GPP} and \gls{GSMA}.


%%
\subsection{Device Active Measurement Investigations}

The approach taken by active measurement studies is simple yet still very insightful. They are performed by writing custom application layer measurement programs for a mobile device. Specific traffic patterns are then generated, recorded, and evaluated. While this can provide very detailed information about the higher network layers, it is limited both in lower layer information as well as scale, due to being limited to a rather low number of devices.

Despite being more or less completely specified in the \gls{3GPP} documents, there is no open layer 1 and 2 (together also called ``baseband'') implementation for \gls{3G}.\footnote{Apart from OsmocomBB (\url{http://bb.osmocom.org/trac/}), but it only provides \gls{GSM} and partial \gls{GPRS} functionality.} Therefore, the baseband's behavior can not be directly instrumented from the application layer. Attempts to infer some properties are still worth conducting as the following selection of publication demonstrates.

In~\cite{Xu:2011:CDN:2007116.2007149} Xu et al.\ use data from a location service combined with active measurements to determine the possible geographic location of a \gls{GGSN} in order to improve the location of application content caches for the current network infrastructure. Similarly, in \cite{sigcomm11middleboxes} Wang et al.\ developed a program to probe mobile networks for middle boxes. That term includes any node that alters traffic and affects performance not intended by the actual end-to-end protocols. Examples are \gls{CGN}~\cite{rfc7021}, firewalls, or intercepting \gls{HTTP} proxies. A large number of such nodes were present in the investigated mobile networks and resulted in increased device power usage and download durations and even pose security issues themselves.

Concerning methods to infer specific baseband and \gls{RRC} state machine timer values with active measurements, a 2007 paper~\cite{4640935} presents a way to do this by transmitting packets with a varying inter-departure time and studying the resulting arrival pattern. Indeed, the dynamics of the radio interface's \gls{RRC} signaling and involved state machines are under investigation by several publications. However, almost all focus solely on the impact at the radio interface but pay little attention to potential implications in the \gls{CN}.

The aforementioned work is continued in \cite{5360763} and uses the presented tools to derive \gls{RRC} transitions and power usage from traffic patterns. They found, that operators have a rather larger freedom in configuring the mobile network control plane state machines and deviate from the standard and even omit some states completely.

A further example of cross-layer influences in mobile cellular networks is \cite{qian2011profiling}. It discusses the impact of application layer behavior on \gls{RRC} signaling and its consequences for device energy consumption and radio channel allocation efficiency. The authors argue that there is much room for improvement in this area, and propose some enhancements.

This is further elaborated on by research from Schwartz et al.\cite{schwartz2013angrybirds} using the same technique to analyze the radio signaling load and thus power efficiency from several mobile phone applications. The impact of custom set state machine timers interacting with application traffic is further investigated and the \gls{QoE} is investigated.


%%
\subsection{Research Based On Network Traces}

The second approach to mobile network investigations comes in the form of recording and evaluation traffic traces inside the network. This brings a much larger experiment scale with it, albeit usually at the cost of some finer grained details in the higher protocol layers because of aggregation to flow level. With core network measurements, the signaling traffic of the observed link can also be directly investigated, which is a huge benefit compared to the guesswork in active measurements.

The authors of \cite{4675847} investigate the influence of individual \gls{CN} nodes on the one-way delay distribution of user traffic packets. According to the work, the latency portion added by the \gls{SGSN} is larger but also fluctuating more, while the \gls{GGSN} added a small but steady amount of latency. This provides us with initial clues on the expected load impact of the \gls{CN} for the investigations in this work.

Following up on the topic of mobile network one-way delays is Laner~et~al.\ in \cite{laner2012delaycomparison}. The end-to-end latency of an early \gls{LTE}/\gls{EPC} network implementation is compared to that of a \gls{HSPA} network at several measurement points in the networks. The results show a lower median latency for \gls{LTE}, despite some scenarios still being in favor of \gls{3G} networks.

The authors of \cite{Shafiq:2012:FLC:2254756.2254767} limit their focus to a specific subset of connected devices, namely those of \gls{M2M} type. These are small automated devices that periodically send out data, e.g., sensor readings, or receive control commands. The paper attempts to characterize these on the basis of their generated mobile network traffic. The patterns are clearly distinguishable from traffic caused by other device types such as smartphones.

A 2012 publication~\cite{Zhang:2012:UCC:2377677.2377764} presents us with a more general look on the traffic composition of cellular access networks in comparison to wired access network. More and shorter flows are occurring in the case of cellular networks. It will be interesting to see if this shorter-but-more theme is also evident in signaling traffic. Additionally, even traffic pattern distinctions between types of applications are made showing a wide range of possible outcomes across the investigated applications.

Both the authors of \cite{shafiq2011characterizing} and \cite{paul2011understanding} take the approach of looking at high-level user traffic characteristics in a mobile network, focusing on temporal and spatial variations of user traffic volume and peeking at the influence of different devices on this metric. Additionally, \cite{baer2011two} delivers a theoretical introduction on how to conduct large scale network measurements and compares some data evaluation approaches. The 2008 paper of \cite{4570772} takes a look at times scales and time of day deviations observed in aggregated user traffic in a mobile network.

Up until now no trace-based investigation considered the control plane in their evaluation. The following publications include this at least to some degree.

In 2006, Svoboda~et~al.~\cite{svoboda2006composition} conducted a core network measurement study of various user traffic related patterns, and also provided an initial insight into \gls{PDP} context activity and durations. Another paper~\cite{lee2007detection} combines simulations based on WiFi and synthetic traces with prior knowledge of \gls{RRC} states and their effects to investigate detection methods for signaling \gls{DDoS} occurring on the radio interface. A possible magnitude of this type of attack is discussed. This also gives an indication of the correlation between user traffic patterns and radio signaling.

A 2010 publication\cite{Qian:2010:CRR:1879141.1879159} uses the indirect \gls{RRC} inferring method described earlier on a core network \gls{TCP} trace data set and finds that the involved \gls{RRC} state machine is largely inefficient in terms of signaling overhead and the device's energy consumption for the traffic patterns seen in the data. 

A more recent publication at \cite{he2012panoramic} performs a \gls{RRC} investigation at the path between \gls{RNC} and \gls{SGSN}. The authors classify their evaluations based on device model and vendor and on the application type, and find that different devices have strongly different \gls{RRC} characteristics, which could possibly also have an impact on \gls{gtp} signaling. Here, the \gls{RRC} evaluation was done in a direct manner using explicit logs from the \gls{RNC}. A final paper~\cite{Ricciato2010551} recaps some general attack scenarios on \gls{3G} networks that exploit the specific \gls{3GPP} system design. These are often closely related to the control plane.


%%
\subsection{Traffic Modeling}

Extracting viable models from mobile traffic measurements will also play a significant role onwards. The first related work is a survey of source modeling approaches for \gls{GPRS} user traffic from the year 2000 \cite{staehle2000source}. Models for \gls{HTTP} traffic and user behavior are compared and a combined model is recommended. One has to keep in mind, though, that due to the rapid developments in the Web in recent years those models might no longer be valid. 

Similarly, the authors of \cite{965876} derive a synthetic \gls{UMTS} traffic model from wired dial-up traces. By using a batch Markovian arrival process they characterize session traffic in most cases with a lognormal distribution.

Work conducted in \cite{Halepovic:2005:CMU:1089803.1089969} derives a model for the users' mobility in a mobile network. The mobility model is however more focused on the circuit-switched voice communication features of a phone. Likewise, the authors of \cite{Pesch2005385} introduce a traffic model for \gls{SIP} \gls{VoIP} communication in \gls{UMTS} networks. However, this model is specific to the \gls{IMS} domain of \gls{UMTS} and potentially not applicable to the more common over-the-top pure \gls{SIP} traffic. The model additionally investigates some initial \gls{UMTS} control plane timing values, such as the processing time of \gls{PDP} context activation messages.

A further publication in 2005 \cite{Landman200568} attempts to model the delay of \gls{IP} packets passing through an \gls{UMTS} network using a batch Markovian arrival process. However, the model specifically focuses solely on the delay originating from processing at the radio link and not at the core nodes.

Finally, a further paper by Laner~et~al.~\cite{6214330} investigates, amongst other things, a user's session duration and throughput in a \gls{HSDPA} network. The duration is modeled as an exponential distributions and the throughput using a lognormal distribution, albeit both exhibit additional heavy tail characteristics.


%%
\subsection{\texorpdfstring{\acrshort{3GPP}}{3GPP} and \texorpdfstring{\acrshort{GSMA}}{GSMA} Related Work}

The two associations related to the mobile network under scrutiny, the \gls{3GPP} as well as the \gls{GSMA} themselves have also released some studies and recommendations concerning potential effects of and issues with the control plane. 

In reaction to the mentioned \gls{RRC} signaling \gls{DDoS} the \gls{GSMA} released some best practices \cite{gsma2011fdbestpract} intended to reduce the number of signaling messages in these circumstances. The cause of the \gls{DDoS} were in most cases mobile devices that circumvented the \gls{RRC} state transition timers and explicitly switched the radio to the idle state after a transmission was finished, re-enabling it whenever needed. This can greatly reduce the power usage but increases the number of signaling messages to be sent, and thus the load in the radio network and possibly also inside the \gls{CN}. With the presented ``Fast Dormancy'' mechanism, mobile devices are supposed to reduce the radio signaling amount while still saving energy. The implications of this mechanism on the core are not investigated.

A \gls{3GPP}-released study \cite{3gpp.22.801} also describes the diverse traffic mix originating from modern smartphones and its associated signaling problems.

The aim of the study in \gls{TS}~23.843~\cite{3gpp.23.843} is to document some of the control plane bottlenecks and attack vectors on the \gls{CN}. This also includes the interesting case of \gls{GTP-C} overload and causes for this scenario. Some approaches to alleviate the effects are also presented, but mostly targeted at the \gls{EPC}. The final study is an extension to the last one \cite{3gpp.29.807} and focuses solely on \gls{GTP-C} overload control to be included in a future version of the \gls{3GPP} architecture. Therefore, the mostly unfinished document again targets just at a future version of \gls{LTE} and provides no investigation of the actual load situation in current \gls{3G} networks.

All of the presented publications relate only to some degree to the forthcoming investigations. The combination of the aspects of \gls{CN} signaling with a statistical evaluation and load modeling of \gls{PDP} contexts should be a genuine contribution of the thesis.



%24.826 \cite{3gpp.24.826} Study on impacts on signalling between User Equipment (UE) and core network from energy saving; deals mostly with switching off cells and moving over UEs, not actual core network efficiency

%%%%%%%%%%%%%%%%%%%%%%%%%%%%%%%%%%%%%%%%%%%%%%%%%%%%%%%%%%%%%%%%%%%%%%%%%%%%%%%%
%!TEX root = ../../dissertation.tex
%%%%%%%%%%%%%%%%%%%%%%%%%%%%%%%%%%%%%%%%%%%%%%%%%%%%%%%%%%%%%%%%%%%%%%%%%%%%%%%
\section{Modeling Mobile Network Load}
\label{c4:sec:modeling}

The next logical step after the collection of empirical data and the execution of a statistical analysis lies in the creation of models abstracting this real system. While some loss of precision is incurred, models are much more flexible and can have numerous applications. Load models and the derived information on the network \gls{QoS} parameters can serve as a basis for the video measurement framework of Section~\ref{c3:sec:measurements}, which uses arbitrarily chosen values for its latency and loss experiments. Using the load model a more realistic mobile network could be emulated. Additionally, network operators can also be supported in predicting the signaling load in their core network with the benefit of improved network engineering and correctly scaling core components.

On the basis of the tunnel distributions attained in Section~\ref{c4:sec:evaluations}, models for both a traditional \gls{GGSN} as well as a virtualized \gls{GGSN} are introduced. The performance trade-offs when using a virtual \gls{GGSN} are further studied, discussing different options to consider when using the virtual node.

The modeling and simulation of the resulting models was conducted in cooperation with the University of Würzburg and partially published in \cite{metzger2014lossmodel}.

%\cite{trangia-lbvs}

%%%%%%%%%%%%%%%%%%%%%%%%%%%%%%%%%%%%%%%%%%%%%%%%%%%%%%%%%%%%%%%%%%%%%%%%%%%%%%%
\subsection{Queuing Theory Basics}

To understand the modeling process some knowledge on queuing theory is required. The next few sections give a short overview on the definitions used in the subsequent sections.


%%
\subsubsection{Little's Law}

A basic queuing system can be expressed as a stream of customers arriving at an arbitrary system with a rate $\lambda$. This system then processes the customers, taking an average time of $W$ on a number of processors until the customers depart again. On average $L$ customers will be in the system. The representation --- and queuing theory in general for that matter --- was originally devised for telephone networks by Erlang~\cite{erlang1917solution}. From that, Little's~Law~\cite{little1961proof} can be formulated as

\begin{equation}
	\phantom{,}L = \lambda W\text{,}
\end{equation}

which holds universally, independent of any specific arrival or service time process.


%%
\subsubsection{Kendall's Notation}

To distinguish the variations of a queuing system's parameter a simple convention and naming scheme was devised by Kendall in 1953~\cite{kendall1953stochastic} and later extended on.  In its simplest form the notation reads $A/S/s$ with $A$ denoting the arrival distribution, $S$ the service time distribution, and $s$ the number of servers. Here, an extended notation will be used, 

\begin{equation}
	A/S/s/q
\end{equation}

which additionally describes the queue length $q$. With this naming scheme, a queuing system ($q=\infty$) can be easily distinguished from a blocking or loss system ($q=0$). The most commonly used arrival processes and service time distributions are summarized in Table~\ref{c4:tbl:kendalldistributions}.

\begin{table}[htb]
\caption{Typical abbreviation of processes in Kendall's notation.}
\label{c4:tbl:kendalldistributions}
	\begin{tabu}{X[l]X[7]}
	\toprule
	\textbf{Symbol} & \textbf{Description} \\
	\midrule
	$M$ & Markovian, i.e., Poisson, arrival process or exponential service time distribution \\
	$D$ & Deterministic arrival process or service time distribution \\
	$G$ & General arrival process or service time distribution with no special assumptions \\
	$GI$ & General arrival process with independent arrivals; also called regenerative \\
	\bottomrule
	\end{tabu}
\end{table}


%%
\subsubsection{Information Gain}

Depending on the complexity of the specific queuing system model, much information can be gained from an analysis of the given model. In simple cases the state probability can be mathematically determined, i.e., the probability that exactly $m$ customers are in the system concurrently. If this number is higher than the number of processors, this also determines the queue length, or the blocking probability $p_B$ if there is no queue. Other properties include for example the waiting time of customers.

\begin{figure}[htb]
	\centering
	\includegraphics[width=1.0\textwidth]{images/markovchain.pdf}
	\caption{$M/M/c/\infty$ Markov chain model.}
\label{c4:fig:markovchain}
\end{figure}

One such basic queuing system is $M/M/1/\infty$~\cite[pp.~94-99]{Kleinrock:1975:TVQ:1096491}, on which stationary analysis can be applied upon. Both the one processor queue and $M/M/c/\infty$ can also be easily expressed as a Markov chain due to their memoryless property. Figure~\ref{c4:fig:markovchain} depicts the state transitions of a system with $c$ processors and a queue length of $i-c$.
% Bin mit der Queue Length nicht ganz einverstanden. Für i<c gibt's doch keine Queue. Außerdem fehlen mir für ein .../.../.../\infty-System noch \dots rechts vom State i.

More complex models are often not tractable by stationary analysis or other mathematical tools any more and no general solution is known. This is especially true for the class of $G/G/c$ systems, which can only be directly solved under certain conditions. System parameters may still be investigated using numerical queuing simulation. Here both the arrival and the serving process are implemented in a \gls{DES} using random numbers drawn from the desired distributions in order to determine the system load and blocking probability.


%%%%%%%%%%%%%%%%%%%%%%%%%%%%%%%%%%%%%%%%%%%%%%%%%%%%%%%%%%%%%%%%%%%%%%%%%%%%%%%
\subsection{GGSN Model Rationale and General Queuing Theoretic Representation}

The \gls{GGSN} was already determined to be critical to the \gls{CN}'s load. Therefore, the network will be represented by this node in the model. Additionally, most of the load influencing factors are at least to a degree related to the \gls{gtp} tunnels. So, to dimension a mobile network based on its control plane load, the number of supported tunnels has to be modeled. 

\begin{figure}[htb]
	\centering
	\includegraphics[width=0.6\textwidth]{images/GGn-model.pdf}
	\caption{Queuing system representation of a mobile network's \acrshort{GGSN}.}
\label{c4:fig:ggn-model}
\end{figure}

Figure~\ref{c4:fig:ggn-model} shows this model for the proposed tunnel load metric. It is in its generic form a $G/G/c/0$ system. Tunnels enter the system governed by a general random distribution, and are served at the \gls{GGSN} for the duration of their existence. This duration also follows a general distribution. Afterwards, tunnels leave the system again through the reception of a \gls{gtp} tunnel delete message. If all $c$ serving units are filled, blocking occurs, and arriving tunnel requests are rejected.

The number of serving units corresponds to available resources at the \gls{GGSN}. The maximum supported number of concurrent tunnels is hard to estimate as it depends on a number of factors, most of which are unknown for this modeling process. This could include soft-limits like the specific configuration, and hard-limits, e.g., the \gls{GGSN}'s processing and memory constraints.

For the purpose of creating an initial toy model the generic $G/G/c/0$ is simplified to a $M/M/\infty$ system. As stated, no actual limit to the number of virtual servers is known and the data also does not indicate any obvious limits. Thus, an unlimited system with neither blocking nor queuing is assumed for this simple model.

Now, assuming both a Poisson arrival and an exponential serving process (temporarily neglecting the fact that no basic function matched the \gls{GGSN}'s serving process), a stationary analysis can be conducted. As seen in the statistical evaluation, the former condition may hold, but the serving time is definitely not exponentially distributed. However, for the toy model this assumption is still made to get an initial grasp of the model.

The diurnal influences seen in the tunnel arrivals in the trace data are also temporarily ignored and only the overall empirical distribution is taken into account. Through distribution fitting with moment matching the overall arrival rate is set to be $\lambda\approx 25.641$ in the trace. The exponential service time distribution is calculated to have the parameter $\mu\approx \num{1.587e-4}$. Using Little's Law this gives an estimate for the mean number of concurrent tunnels at the \gls{GGSN} in a $M/M/\infty$ system of 

\begin{equation}
	\phantom{.}L=\frac{\lambda}{\mu}\approx 161.6\text{.} %=161598.14.
\end{equation}

As stated, the amount of state held at the node and propagated through the network is directly related to the number of tunnels. Therefore, this metric can serve as an initial estimate of the load at the \gls{GGSN}.


%%%%%%%%%%%%%%%%%%%%%%%%%%%%%%%%%%%%%%%%%%%%%%%%%%%%%%%%%%%%%%%%%%%%%%%%%%%%%%%
\subsection{Representative GGSN Models} 

With the experience from the toy model at hand more appropriate models can now be constructed to better accommodate for the core network's properties. Two models are provided here.
The first describes a monolithic version of a \gls{GGSN}, closely resembling the system used traditionally in the network. The second model is that of a hypothetical virtualized \gls{GGSN} using \gls{NFV}. In \gls{NFV}~\cite{nfv_whitepaper} custom monolithic network nodes are replaced by commodity hardware. The tasks solved by the original hardware is migrated to a pure software implementation.


%%
\subsubsection{Monolithic \texorpdfstring{\acrshort{GGSN}}{GGSN}}

The \gls{3gpp} architecture considers the \gls{GGSN} to be one fixed monolithic entity, even if in reality it often consists of multiple servers. The entire \gls{GGSN} is purchased from a vendor as a single entity, they do not integrate well with other existing network infrastructure. 
% Was soll uns der zweite Halbsatz sagen?
Nor can idle instances be deactivated or reused for other purposes.

\begin{figure}[htb]
	\centering
	\includegraphics[width=0.3\textwidth]{images/ggsn-monolithic.pdf}
	\caption{Traditional control plane load modeling approach to a \acrshort{GGSN}.}
\label{c4:fig:model-ggsn-monolithic}
\end{figure}

The queuing theoretic equivalent is displayed in Figure~\ref{c4:fig:model-ggsn-monolithic} and is very similar to the basic toy model. New tunnels requests arrive according to a Poisson distribution with a rate of $\lambda(t)$ at the \gls{GGSN}. The periodic time-of-day dependence of these exponentially distributed \gls{IAT} and the corresponding distribution fits were extrapolated from the trace data.

Furthermore, the model has a maximum tunnel capacity of $c$. When this capacity is reached, blocking will occur and further incoming tunnels are rejected. The governing factors of the capacity are mostly the node's available memory and processing capabilities. Monolithic \glspl{GGSN} need to be preemptively dimensioned in such a way that blocking rarely happens, often resulting in gross overdimensioning as the node can not be easily scaled after it has been deployed.

When an incoming tunnel request is accepted one of the \gls{GGSN}'s serving units will be occupied for the tunnel's duration $x(t)$. Following the trace data, this duration is assumed to be of an arbitrary, non-Markovian service time distribution, again with a slight time-of-day dependence.

Combining the model with the exponential and rational function fits functions previously depicted in Tables~\ref{c4:tab:IAT-fits} and \ref{c4:tab:fits-duration} this results in a \textbf{non-stationary Erlang loss model}, or more precisely

\begin{equation}
	\phantom{.}M(t)/G(t)/c/0\text{.}
\end{equation}

With this model, high control plane load can be indirectly described as the system's blocking probability $p_B$. The peak load can be ascertained by looking at the busy hour period where the arrival rate is the highest. No exact mathematical solution is known for this type of model. Only if the service time distribution can be confined to certain specific distributions, e.g. Phase-type distributions, some approximations can be made \cite{davis1995nonstationaryerlang}. 
However, the evaluated trace data does not give any indication of the presence of such a Phase-type in the the service time distribution. Therefore, the model falls back to a non-stationary general distribution  and a simulative approach to evaluate the model will be taken in Section~\ref{c4:sec:simulation}.


%%
\subsubsection{Virtualized \texorpdfstring{\acrshort{GGSN}}{GGSN}}
\label{c4:sec:virtual_ggsn}

\begin{figure}[htb]
	\centering
	\includegraphics[width=0.6\textwidth]{images/ggsn-virtualized.pdf}
	\caption{Model of a \acrshort{GGSN} using \acrshort{NFV}.}
% Hmmm.... ein Kastl gefüllt mit schwarzen Kastln?
\label{c4:fig:model-ggsn-virtualized}
\end{figure}

In the second model virtualization concepts are introduced. The assumptions of the non-stationary Markov arrival process $\lambda(t)$ and the serving time distributions $x(t)$ are carried over. However, instead of one server processing every tunnel, the system is now partitioned into individual server instances coordinated by a load balancer in Figure~\ref{c4:fig:model-ggsn-virtualized}. 

One virtual \gls{GGSN} has up to $s$ servers instances $s_i$. Each of the individual instances can be much smaller than the monolithic \gls{GGSN}, having a concurrent tunnel serving capacity of $c_i \ll c$ and a total system capacity of $c_{virt} = \sum_{i=1}^{s} c_i = \| \overrightarrow{c}\|_1 \text{ with } \overrightarrow{c} = \{c_1, c_2, \ldots ,c_i, \ldots ,c_s\}$. % 1-norm des Kapazitätsvektors
The complete model now reads:

\begin{equation}
	\phantom{.}M(t)/G(t)/\|\overrightarrow{c}\|_1/0\text{.}
\end{equation}

The instances do not have a static uptime. Instead, their life cycle is managed by a \gls{VMM} and adjusted to the current load of the network. New tunnels are either placed on running instances or new ones are provisioned on demand. The \gls{VMM} can have multiple optimization goals. A prominent example is the minimization of server instance and energy usage.  Another set of example provisioning rules is discussed in the implementation of the model simulation in Section~\ref{c4:sec:simulation}. 

A target criterion could be to keep the blocking probability inside a certain target range. If the \gls{VMM} decision rules are not carefully selected additional blocking could occur. Despite not having reached its maximum capacity, this system will still reject tunnel requests during the provisioning phase when no tunnel slots are free. This could be remedied by a request queue. However, this makes the system more complex without providing real benefit, as failed tunnel requests are retransmitted by the network control plane or another attempt might also be made directly by the mobile device after a timeout.

To place incoming tunnel state on one of the available servers and manage the servers a load balancer is required. To ensure that the system can scale down to its actual needs, the balancer should place tunnels on servers that are the fullest, keeping the reserve free. It may even migrate tunnel state from almost empty servers away so that these can be shut down, when certain conditions are fulfilled. Keeping instance close to their capacity should also have no impact on the performance a mobile device associated to a specific tunnel experiences. Adequate strategies for both load balancing and migration should be considered in subsequent research.

Through this virtualized model, which suggests to use technologies from cloud computing in the network and replace specialized nodes with commodity hardware, network operators can scale the \gls{GGSN} out instead of only up. Today, these network components are typically sold in a static and monolithic form and can not be easily extended with off-the-shelf hardware in order to accommodate to a changing environment. The system in this model can however be easily scaled out to additional low cost machines instead of completely replacing the existing \gls{GGSN} with a more powerful version. 

It is also entirely possible that the described single-arrival-process approaches might not the best way to describe control plane load. Several load influencing factors discussed earlier have direct influence on the tunnel arrivals and duration, e.g., the device type or the radio access technology. Therefore, amongst others, multidimensional queuing networks or fluid flow models could be more appropriate in subsequent research. Still the non-stationary Erlang loss model described here should be sufficient for basic core network control plane load estimations.



%%%%%%%%%%%%%%%%%%%%%%%%%%%%%%%%%%%%%%%%%%%%%%%%%%%%%%%%%%%%%%%%%%%%%%%%%%%%%%%%
%!TEX root = ../../dissertation.tex
%%%%%%%%%%%%%%%%%%%%%%%%%%%%%%%%%%%%%%%%%%%%%%%%%%%%%%%%%%%%%%%%%%%%%%%%%%%%%%%%
\section{Measurements}
\label{c3:sec:measurements}

With the buffer-based playback model and strategies at hand, this section demonstrates how to conduct actual evaluations of reliable streaming protocols with it.

As discussed, there are numerous incarnations of reliable streaming protocols in use. Almost all of them follow the same basic approach, but each with slight variations in execution, choice of playback strategies, and corresponding parameters. Exactly these choices can have a large impact on the streaming process and resulting quality. 
% Oder: "Those are exactly the choices which can have...". Auch ungelenkig :-)

The problem lies in comparing these protocols to each other. Each of them is usually tied to a specific --- and most often proprietary closed source --- streaming player. Setting up all these players in one testbed is a huge effort and requires very specific software environments to be used on the client computers. Moreover, these players are built with user interaction and not automation in mind, hampering efforts of directly measuring the outcome. This can still be achieved through extensive workarounds, but those must be tailored to every individual player application. 

The approach presented here avoids this hassle and provides a concise way to test any conceivable playback strategy in one single test setup.


%%
\subsection{Progressive Streaming Measurement Framework}

To enable quick evaluations for reliable streaming the framework follows a two-phase approach, separating the active online recording phase from the passive playback emulation. Recording network data is very time intensive and cannot be sped up when conducting an investigation of a real world process, and not relying on simulated data. The framework still replicates the steps a user would perform to consume a media stream on a playback device, but simply separates them. Through appropriate configuration different scenarios can be modeled, e.g., network conditions and behavior or specifics of the user device.
 
\begin{figure}[htb]
	\centering
	\includegraphics[width=0.9\textwidth]{images/measurement-model.pdf}
	\caption{Overview of the measurement framework for progressive streaming playback strategies.}
\label{c3:fig:framework}
\end{figure}

Figure~\ref{c3:fig:framework} depicts the usage of the framework for a streaming evaluation testbed. In phase one the actual transmission of the stream is conducted and recorded as a packet level network trace. These traces should at least consist of the size and timestamp of every incoming packet.

Stream data is transmitted to the client from a server which can be any actual streaming service on the Internet or a local server under the testbed's control, eliminating undesired side effects caused by the Internet connection. The traffic is further directed through a network emulation node capable of altering the network \gls{QoS} parameters, i.e., latency, jitter, and packet loss. The parameters can be set according to stochastic models derived from actual network architectures such as the \gls{3G} mobile network architecture.

Instead of network emulation, any preexisting architecture can also be placed here to achieve more accurate results for the intended target. This is especially helpful for complex infrastructures hard to model or with no good and concise models available yet. Of special note for this work are the previously discussed mobile networks, which encapsulate the user traffic into tunnels and exhibit complex control plane interactions that can influence the streaming process.

Additionally the received video file is decoded yielding a trace of all video frame sizes and playback timestamps. All data gained in the process is stored as a basis for the second phase. More detailed traces can additionally be used to scrutinize other layers of the connection, e.g., the dynamics of \gls{TCP} receive window size. 

In the second pass, both the network as well as the video data are then used to feed the actual reliable streaming playback model described before. This is conducted by a closed-loop emulation process calculating the current buffer fill level based on the collected transmission and video frame traces for every point in time. 

All of the non-adaptive reliable streaming strategies can be tested on the same trace set. In the simplest form of \gls{HTTP} streaming the transmission is not controlled by the streaming application and no rate control is conducted. Therefore, recording the packet trace and simulating playback are completely decoupled, as the latter cannot influence the former. This enables a fast and efficient comparison of various non-feedback protocols which are all subjected to the same network conditions.

The emulator then generates playback stalling statistics, specifically their number and duration, to compare the effect of the different strategies on the same trace. With these results, parameter settings for playback strategies can also be iteratively tested and improved, leading to an empirical calibration of playback strategies instead of relying on best practices.

One of the drawbacks of this model-based emulation approach is of course the reliance on suitable models and playback strategies for the stream protocols under scrutiny. Obtaining these from proprietary closed source streaming clients can be a difficult and time consuming reverse-engineering process.


%%
\subsection{Adaptive Streaming Measurement Framework}

Up to this point, the measurement framework is only suitable for simple reliable streaming strategies but neglects any adaptive strategy. This second iteration of the framework modifies the base framework and allows for the testing of adaptive playback strategies. However, to achieve this, the advantageous two-phase setup cannot be employed anymore.

\begin{figure}[htb]
	\centering
	\includegraphics[width=0.9\textwidth]{images/feedback-measurement-model.pdf}
	\caption{Overview of the measurement framework for adaptive streaming playback strategies.}
\label{c3:fig:framework-feedback}
\end{figure}


Figure~\ref{c3:fig:framework-feedback} shows the adapted framework. The playback emulation process is now directly fed with the stream transmission without recording it first. The emulation is now an online process and has to be conducted in realtime. This enables the emulator to react on the current streaming state and request an alteration from the server. The adaptation spectrum ranges from the timing of stream segment retrieval to the chosen quality level of future segments. 

While allowing for a wider range of playback strategies, this approach is also inherently slower as it does not allow a speed-up beyond realtime, limiting its usability somewhat. 
% Naja, trivial parallelisierbar wäre der Ansatz
Therefore, a transition to a full simulative approach is suggested. This is path is further explored and discussed in Section~\ref{c6:sec:mobilestreamingtestbed}.


%%
\subsection{Technical Implementation}

To conduct actual measurements, the described two phase progressive streaming measurement framework has been implemented as a network testbed. The three individual components of the framework in Figure~\ref{c3:fig:framework} are represented by three interconnected physical nodes running Linux. 

The streaming server houses an Apache httpd Web server\footnote{\url{https://httpd.apache.org/}}, hosting the files that are to be streamed. Alternatively, traffic from any viable Internet streaming service can also be directly routed through the network emulation node, making the local streaming server superfluous.

The network emulation node uses existing \gls{QoS} capabilities of the Linux kernel, dubbed NetEm~\cite{hemminger2005network}, to add latency and packet loss to the transmission as well as to act as a bandwidth bottleneck. The additional delay will be set to a deterministic value for the following experiments. The loss follows a uniform distribution without any correlation in the transmission.

Curl\footnote{\url{http://curl.haxx.se/}} is used to both retrieve the streaming file and record the transmission process at the client node. If so desired, tcpdump\footnote{\url{http://www.tcpdump.org/}} can also be facilitated to achieve a higher recording precision. The video file is then parsed for its frame timings and sizes using mplayer\footnote{\url{http://www.mplayerhq.hu/}} with FFmpeg\footnote{\url{https://www.ffmpeg.org/}}. 

The traces are then put into the actual playback emulation, which can be run on any computer. It is implemented by custom Python-based code and statistically evaluated with Python\footnote{The emulation code is publicly available at \url{https://github.com/fmetzger/thesis-bufferemulation}.} as well as R.


%%
\subsection{Measurement Series and Evaluations with the Framework}

This testbed is now used to conduct a comparative study of two theoretical and two real world playback strategies. They are tested for their susceptibility to worsening network \gls{QoS}, specifically latency and loss. The evaluated strategies are the described YouTube and Firefox strategies as well as the null strategy and the predictive strategy with perfect knowledge and an optimally-sized initial pre-buffering phase that compensates for every degradation in the network conditions.

\begin{table}[htbp]
\centering
\caption{Parameters of the video used in the streaming emulation measurement series.}
\label{c3:tbl:videoparams}
	\begin{tabu}{X[l]X[r]}
		\toprule
		\textbf{Parameter} & \textbf{Value} \\
		\midrule
		Video duration  & \SI{92.5}{\second}\\
		Size & \SI{9.61}{\mebi\byte} \\
		Frame rate & \SI{23.976}{\per\second} \\
		Average video bitrate & \SI{871}{\kilo\bit\per\second} \\
		Codec & \acrshort{AVC} \\
		\bottomrule
	\end{tabu}
\end{table}

The video used in the experiment was streamed from the YouTube web site providing a realistic foundation for the experiments. This also enables a server side pacing mechanism adjusted to the video bitrate for free. Details on the video used in the experiment are available in Table~\ref{c3:tbl:videoparams}. 

Two measurement series are performed with this video, both only differ in the network emulator settings. The first series increasingly adds packet loss to the stream, with the second series altering the packet delay. In both scenarios the link bandwidth was limited to a typical \gls{DSL} value of \SI{16}{\mega\bit\per\second} in the downlink direction and \SI{1}{\mega\bit\per\second} up.

It can be stated that all playback strategies will generally work similarly well under good network conditions as long as the \gls{TCP} ``goodput'', i.e., the rate at which the payload is transported by \gls{TCP}, is higher than the video bitrate. With sufficient goodput video streams will start with almost no delay or intermediate buffering. 

However, if the achievable throughput is close to the average video bitrate, the buffer can be quickly drained by short deviations from the average rates. High latency and loss are the typical limiting factors for \gls{TCP} as many congestion control algorithms depend on the \gls{RTT}. If the \gls{RTT} is high, the congestion window will increase less quickly. High latency can also trigger timeouts and retransmissions, which in turn decrease the congestion window again. 

Packet loss can affect \gls{TCP} goodput even more. A lost packet results in duplicate acknowledgments followed by retransmissions and a decrease of the congestion window. The problem is worsened if the acknowledgments are also lost. The connection could stall on missing old segments without which the playback cannot proceed. In addition to the reduction of goodput this results in a delay burst and high jitter for the streaming application. Further influences will be discussed in Chapter~\ref{chap:mobilestreaming}.


%%
\subsubsection{Latency Measurement Series}

In the latency measurement series, the emulator delays forwarding the packets for a constant amount of time. The latency was increased in \SI{100}{\milli\second} steps, up to a total of \SI{5000}{\milli\second}. The added latency is split up evenly between the uplink and the downlink. Each individual experiment was also replicated five times and corresponding error bars are provided in each figure.

\begin{figure}[htbp]
	\centering
	\includegraphics[width=0.9\textwidth]{images/R-playbackemulation-stallduration-latency.pdf}
	\caption{Stalling duration in relation to transmission latency with a local polynomial least-squares fit.}
% Worauf wird da lokal gefittet? (Frage ich mich auch bei den anderen Graphen dieser Art.)
\label{c3:fig:eval-latency-stallingtime}
\end{figure}

Figure~\ref{c3:fig:eval-latency-stallingtime} depicts the relation between the added latency and the stalling duration of the playback strategies. The stalling time increases as expected with the additional latency, but Firefox's strategy seems to have a slight edge under high latency. Overall, the stalling duration quickly reaches a length comparable to the actual video duration and even surpasses that. Someone watching a stream under these conditions might find this not acceptable any longer.

\begin{figure}[htbp]
	\centering
	\includegraphics[width=0.9\textwidth]{images/R-playbackemulation-stallnumber-latency.pdf}
	\caption{Number of stalls in relation to transmission latency with a local polynomial least-squares fit.}
\label{c3:fig:eval-latency-numstalls}
\end{figure}

Figure~\ref{c3:fig:eval-latency-numstalls} additionally shows number of stalling events occurring during the playback with the null strategy at the high end and the predictive strategy just showing the expected single stall before playback start.

\begin{figure}[htb]
	\centering
	\includegraphics[width=0.9\textwidth]{images/R-playbackemulation-qoe-latency.pdf}
	\caption{Calculated \acrshort{MOS} for the latency measurement series.}
\label{c3:fig:eval-latency-qoe}
\end{figure}

Using the model from Equation~\ref{c3:eqn:hossfeld-stalling-model} the \gls{QoE} is calculated for this series and depicted in Figure~\ref{c3:fig:eval-latency-qoe}. The \gls{MOS} quickly drops below a value of $3$, which is generally accepted as still being of fair quality, and stays around $2$ for the most portion of the latency series. Starting at around \SI{3000}{\milli\second} of latency all four strategies achieve virtually the same poor quality, according to this model. But especially between \SI{1000}{\milli\second} and \SI{2000}{\milli\second} the difference in quality of these strategies is visible, with both the theoretical predictive and the YouTube strategy reaching a much higher \gls{MOS} than the other two.
% Anschließende Frage: Was wäre gemäß des Hossi-Modells eigentlich die ideale Abspielstrategie? Und schreiben wir bei Gelegenheit ein Paper über die Frage?

Overall, it can be said that in this specific latency scenario the real world playback strategies seem to honor the fact that more video interruptions lead to a worse experience than fewer but longer stalling events. Through mobility and handovers mobile devices fairly commonly experience short bursts of latency of several seconds. According to the measurement series, the resulting stalling behavior could still very well be bearable for streaming users if the latency does not reach too high values on average. For example, at \SI{1000}{\milli\second} latency a \gls{MOS} of $3$ could still be easily achievable.
% Lese ich Fig. 4.12 falsch? Ich sehe da nur bei predictive eine MOS von 3 bei 1s Delay.

%%
\subsubsection{Loss Measurement Series}

In the loss measurement series, uncorrelated and uniformly distributed loss was added in both the uplink and the downlink direction. The loss was incrementally increased in \SI{0.5}{\percent} steps up to a total additional loss of \SI{12.5}{\percent}. About \SI{4}{\percent} of the individual experiments did not finish correctly, and the video was not completely transmitted. This was caused by the \gls{TCP} stack, which at some point terminated the connection after too many packets were lost back to back, and curl giving up after several retries. This unpredictability and failure rate also leads to the high variability seen in the results of the loss measurement series. Nonetheless, a certain trend can still be derived from the results. Again, five experiments for each entry with corresponding error bars are conducted.


\begin{figure}[htbp]
	\centering
	\includegraphics[width=0.9\textwidth]{images/R-playbackemulation-stallduration-loss.pdf}
	\caption{Stalling duration in relation to the packet loss with a local polynomial least-squares fit.}
\label{c3:fig:eval-loss-stallingtime}
\end{figure}

Figure~\ref{c3:fig:eval-loss-stallingtime} shows the resulting relative stalling duration in the packet loss measurement series. Loss of up to about \SI{2.5}{\percent} seems to have no discernible impact on the streaming process. Anything beyond this point sees a large increase in the stalling duration. With a relative stalling duration of almost four times the actual video length at \SI{12.5}{\percent} packet loss any streaming attempt is practically rendered unusable. Also, all four tested playback strategies handle high loss equally unsatisfactorily, with the exception of some Firefox results producing a lower stalling duration.

This general behavior could hint to 
% eher "could be explained by"
the transport protocol's reliable transport feature, catching any occurring loss. However, the detection and retransmission of lost segments takes time and leads to a bursty increase in latency. It also represents a possible reason of the increased stalling time. 

At least it can be safely assumed that in actual production networks high values of packet loss are usually less likely to occur than high latency. 
% Ja im Interweb, naja im heimischen WLAN.
The only major source of packet loss should be network congestion, which should only occur in moments of high network load. Therefore, it can be expected that playback strategies are better optimized for scenarios with high latency than loss.

\begin{figure}[htbp]
	\centering
	\includegraphics[width=0.9\textwidth]{images/R-playbackemulation-stallnumber-loss.pdf}
	\caption{Number of playback stalls in relation to packet loss with local polynomial least-squares fit.}
\label{c3:fig:eval-loss-numstalls}
\end{figure}

Figure~\ref{c3:fig:eval-loss-numstalls} clearly shows the extremity of the null strategy in terms of the number of experienced stalls compared to any other strategy. The same loss measurement series is used as basis here. The null strategy runs into two orders of magnitude more stalling phases than the three other strategies. The number of stalling events of the three other strategies remain relatively low. But the individual stalling events will be rather lengthy ones when keeping the total stalling duration in mind. Therefore, in the packet loss scenario the factor that will degrade \gls{QoE} the most seems to be the duration of the stalling events but not their number, with the exception of the null strategy.

\begin{figure}[htb]
	\centering
	\includegraphics[width=0.9\textwidth]{images/R-playbackemulation-qoe-loss.pdf}
	\caption{Calculated \acrshort{MOS} for the loss measurement series.}
\label{c3:fig:eval-loss-qoe}
\end{figure}

Looking at the \gls{QoE} of the loss series, displayed in Figure~\ref{c3:fig:eval-loss-qoe}, the difference in \gls{MOS} between the playback strategies is much smaller than the one observed in the latency series. Additionally, the \gls{MOS} degradation happens much more slowly and evenly dispersed across the loss values. A \gls{MOS} of $3$ is only undercut at around \SI{5}{\percent} packet loss. But the fluctuations between the individual experiments is rather large in this series, often giving results $2$ \gls{MOS} units apart for the same packet loss percentage, making loss a very unpredictable factor as it often triggers secondary effects.

All in all, when planning a network for streaming applications, the maximum loss should be kept below the \SI{2.5}{\percent} mark to achieve reasonable streaming quality. All the existing strategies already seem to work rather well for typically experienced network \gls{QoS} scenarios. Only when extraordinary network conditions are present the strategies break down. But this is not different to many other network applications, which work best with pristine \gls{QoS}.





%% original python plots
% \begin{figure}[htb]
%     \centering
%     \includegraphics[width=\textwidth]{images/eval-latency-stallingtime.pdf}
%     \caption{Stalling duration in relation to transmission latency.}
%     \label{c3:fig:eval-latency-stallingtime}
% \end{figure}

% \begin{figure}[htb]
%     \centering
%     \includegraphics[width=\textwidth]{images/eval-latency-frequency.pdf}
%     \caption{Number of stalls in relation to transmission latency.}
%     \label{c3:fig:eval-latency-numstalls}
% \end{figure}

% \begin{figure}[htb]
%     \centering
%     \includegraphics[width=\textwidth]{images/eval-loss4mb-stallingtime.pdf}
%     \caption{Stalling duration in relation to packet loss.}
%     \label{c3:fig:eval-loss-stallingtime}
% \end{figure}

% \begin{figure}[htb]
%     \centering
%     \includegraphics[width=\textwidth]{images/eval-loss4mb-frequency.pdf}
%     \caption{Number of playback stalls in relation to packet loss}
%     \label{c3:fig:eval-loss-numstalls}
% \end{figure}


% \begin{figure}[htbp]
%     \centering
%     \includegraphics[width=\textwidth]{images/R-delayseries.pdf}
%     \caption{Total buffering time and linear smooth for degraded network parameter scenarios. Latency Graph.}
%     \label{c3:fig:delayseries}
% \end{figure}

% \begin{figure}[htbp]
%     \centering
%     \includegraphics[width=\textwidth]{images/R-lossseries.pdf}
%     \caption{Total buffering time and linear model for degraded network parameter scenarios. Loss Graph.}
%     \label{c3:fig:lossseries}
% \end{figure}

%%%%%%%%%%%%%%%%%%%%%%%%%%%%%%%%%%%%%%%%%%%%%%%%%%%%%%%%%%%%%%%%%%%%%%%%%%%%%%%%
\section{Reliable Streaming Summary}
\label{c3:sec:conclusion}

In the course of this chapter, a complete toolset to classify, model, emulate, measure, and finally to evaluate video streaming and interpret the results was given.

The multitude of approaches to reliable \gls{HTTP} streaming in recent years, required the creation of specific tools. Previous advances were generally not flexible or fast enough to deal with the influx of new system parameters. The presented measurement framework, based on a fundamental playback model, can be an answer to this issue. Implemented as an emulation in a testbed or a \gls{DES} it can deal with almost any reliable streaming protocol.

Out of these reliable protocols, a differentiation and categorization solely based on their playback strategies was made and compared to real world examples. With the help of the measurement framework and testbed, these strategies were quickly evaluated for their behavior under difficult network \gls{QoS} conditions. Although they might not behave perfectly when stressed, they probably will still be the way forward in the future, because of their reduced complexity and the shift of control logic from the server to the client as well as from the protocol to the application.

This also makes them very interesting for usage on mobile devices, for the growing mobile application ecosystems, and for the very distinct behavior of streaming in mobile networks in contrast to classical wireline networks. The description of the mobile streaming environment and approaches for an investigation will be conducted in the next chapters.



%observe existing systems, e.g. YouTube \cite{metzger2011delivery}

%This paper investigates how Web-based media delivery works in general, and how meaningful measurement of its streaming quality can be achieved with future network developments and degraded network parameters in mind.
%Our model reports perceivable artifacts of buffer underruns, e.g. skips or stalls, which could then be fed into a QoE model to yield actual user QoE values.



% mention relationship to rtp again

%The purpose of this model and its evaluations is manifold. It could lead to protocols tailor-made for specific networks or an improved network planning process. -> or empirical approaches


% merge into chapter summary
%Through these to exemplary experiments, we tried to show that network QoS parameters have a direct measurable impact on the application layer, namely on HTTP streaming quality. While the models scale rather well with latency, any HTTP streaming is almost impossible with high packet loss values.Comparing the presented playback models, we conclude that every model represents a trade-off between several parameters, e.g. as measured here, the number and length of stalls. With the knowledge gained from the experiments, playback models could be tailor-made to best suit certain conditions and user requirements. 
%

%Streaming to mobile devices, especially in mobility scenarios, arises several new issues not seen at wireline connected devices.
 
%The lower-layer protocols on the radio link can cause additional unexpected behavior.
 
%Handover between radio cells can cause long periods of very high delay (up to seconds)

%interplay with server-side bandwidth pacing methods as employed by YouTube.