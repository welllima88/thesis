%!TEX root = ../../dissertation.tex
%%%%%%%%%%%%%%%%%%%%%%%%%%%%%%%%%%%%%%%%%%%%%%%%%%%%%%%%%%%%%%%%%%%%%%%%%%%%%%%%
\chapter{Evaluating Mobile Signaling Traffic and Load}
\label{chap:mobilenetsmeasuring}

With the architectural and methodological overview concluded this chapter can now move on to the actual evaluation. To this end, the previously described dataset is explored for any control plane load related signs. The evaluation, including a statistical analysis, is given in Section~\ref{c4:sec:evaluations}. The measurement data backs up a number of assumptions on the behavior of different device and operating system types, but also reveals some remarkable differences 
% "differences" zu was? Zur Annahme? Welche wäre das?
in signaling characteristics.

The results of these measurements are then used to construct a queuing theoretic load model for a \gls{CN} in Section~\ref{c4:sec:modeling}. This model is then extended with  virtualization modifications to it and followed by a numerical simulation in Section~\ref{c4:sec:simulation} to confirm the viability of the models.



%%%%%%%%%%%%%%%%%%%%%%%%%%%%%%%%%%%%%%%%%%%%%%%%%%%%%%%%%%%%%%%%%%%%%%%%%%%%%%%
%!TEX root = ../../dissertation.tex
%%%%%%%%%%%%%%%%%%%%%%%%%%%%%%%%%%%%%%%%%%%%%%%%%%%%%%%%%%%%%%%%%%%%%%%%%%%%%%%%
\section{Mobile Core Signaling Evaluation}
\label{c4:sec:evaluations}

Finally, the core network control plane load evaluations can now be tackled. The previously described dataset is thoroughly investigated several approaches to measure load and related factors are iterated.


%%%%%%%%%%%%%%%%%%%%%%%%%%%%%%%%%%%%%%%%%%%%%%%%%%%%%%%%%%%%%%%%%%%%%%%%%%%%%%%%
\subsection{Traffic Ratio Estimations}

\begin{table}
\centering
\caption{Relative device-discriminated traffic statistics extracted from the dataset.}
\label{tab:trafficstats}
	\begin{tabu}{X[1.4]X[r]X[r]X[r]X[r]X[r]}
	\toprule
	& \textbf{Flows} & \textbf{Traffic} & \textbf{Tunnels} & \textbf{\gls{gtp} messages} & \textbf{Devices}\\ 
	\midrule
	\multicolumn{2}{c}{\textbf{By device type}}       &             &             &             &           \\
	% In TAC DB      & $99.72\%$   & $99.97\%$   & $87.57\%$   & $90.95\%$   & $80.93\%$ \\
	Smartphones      & $20.58\%$   & $12.81\%$   & $60.31\%$   & $75.99\%$   & $37.97\%$ \\
	Regular phones   & $0.26\%$    & $0.37\%$    & $5.40\%$    & $0.94\%$    & $9.25\%$  \\
	\gls{3G} dongles & $66.55\%$   & $75.12\%$   & $12.71\%$   & $9.53\%$    & $25.10\%$ \\
	\midrule
	\multicolumn{2}{c}{\textbf{By \gls{os}}}       &             &             &             &           \\
	Android          & $10.82\%$   & $6.48\%$    & $14.33\%$   & $43.33\%$   & $14.01\%$ \\
	iOS              & $7.22\%$    & $4.47\%$    & $18.91\%$   & $20.35\%$   & $7.94\%$  \\
	Symbian          & $1.02\%$    & $1.09\%$    & $21.17\%$   & $4.51\%$    & $12.97\%$ \\
	Blackberry OS    & $0.07\%$    & $0.10\%$    & $2.17\%$    & $2.60\%$    & $1.48\%$  \\
	\bottomrule
	\end{tabu}
\end{table}

To get a first grasp of the dynamics present in the dataset and the core network under investigation, Table~\ref{tab:trafficstats} shows a small survey of the traffic composition split up by device type and \gls{os} categories. The majority of signaling messages originated from smartphones, which in turn generated only a small portion of user traffic when compared to \gls{3G} dongles.

With these numbers, the notion of active devices or tunnels can also be introduced. This only includes entities that, besides signaling, actively generated user traffic during their life cycle. Interestingly, only about \SI{82}{\percent} of all unique devices in the trace were active and could be associated with at least one traffic flow. The remaining \SI{18}{\percent} of devices still had an open \gls{gtp} tunnel but never used it. This is a probably unexpected source of load for the core network, as it causes a significant amount of control plane load without any actual benefit to either the network or the device. The active device distinction will also be used later on in the evaluation.

Unfortunately, the dataset does not contain any hard numbers on the data volume of the signaling messages, which could be a direct indicator of the network load the control plane imposes. Using the estimated upper limit of a \gls{gtp} message from Section~\ref{c4:sec:gtp}, a rough upper limit on the total signaling traffic can also be derived. The following formula is used:

\begin{align}
	\phantom{,}v_s &= 2S(v_{gtp} + v_{udp} + v_{ip})\text{,}\\
	\phantom{.}t_r &= \frac{v_s}{v_t} \approx 0.7\si{\percent}\text{,}
\end{align}
%
with the signaling volume $v_s$, the number of signaling messages $S$ (times two since requests and responses are considered), the estimated size of a \gls{gtp} message $v_{gtp}$, and the length of the \gls{UDP} and \gls{IP} headers. In this scenario, the traffic ratio $t_r$ of $v_s$ compared to the total traffic $v_t$ is calculated to be a minute \SI{0.7}{\percent}. Therefore, it can be assumed that the volume of control plane traffic is not a limiting factor on the core network l non-issue and not the bottleneck. Any impact on \gls{CN} performance thus likely stems from other load factors at the network nodes described earlier, such as the memory profile of the states kept in the gateway nodes, the time required to process the large number of information held in the messages, or the imposed latency through several message round trips during transactions.

Consequently the following evaluations are all intended to find some indirect approach to measure the system's load.



%%%%%%%%%%%%%%%%%%%%%%%%%%%%%%%%%%%%%%%%%%%%%%%%%%%%%%%%%%%%%%%%%%%%%%%%%%%%%%%%
\subsubsection{\texorpdfstring{\acrshort{gtp}}{GTP} Tunnel Duration}

The first indirect evaluation target will be the duration 
% i.S.v. "holding time" I suppose
of the \gls{gtp} tunnels. This duration is directly related to the amount of tunnel management signaling occurring between the \gls{SGSN} and \gls{GGSN}. In turn, each of these signaling interactions causes processing at the two involved nodes, and changes the amount 
% and content?
of state stored in the form of the \gls{PDP} context. In terms of signaling messages, the tunnel duration implies both tunnel create and delete messages, but no update message.

For the purpose of the evaluation the duration is defined as the interval between corresponding \gls{gtp} create and delete messages. As soon as the \gls{GGSN} sends its successful response to the create request, it can be expected that the necessary state has been created throughout the \gls{CN}, and that the network is ready to forward user packets. Similarly, after a delete message, user traffic should not be forwarded anymore. However, state may still exist and could be freed up lazily. But the latter depends entirely on the specific implementation.
% "rather than the standard"?

As a side note, while the trace itself is only one week long, information on tunnels longer than this period can still be obtain when they were closed during the period. The trace's record on delete messages also contains the timestamp of the initial tunnel creation.

All the individual tunnel durations in the dataset are differentiated using two factors based on the presented \gls{TAC} mechanics. The first part of the investigation looks at tunnels from different device types. After that, possible influences from the operating system are investigated. 


%%
\paragraph{Influence of the Device Type}

\begin{figure}[htb]
	\centering
	\includegraphics[width=0.9\textwidth]{images/R-tunnel-duration-device-type.pdf}
	\caption{Tunnel duration distribution, separated for \acrshort{3G} dongles, smartphones and regular phones with medians at \SI{115}{\second} (total), \SI{31}{\second} (regular), \SI{82}{\second} (smartphone), and \SI{1207}{\second} (dongle).}
\label{c4:fig:cdf-duration-device-class}
\end{figure}

Figure~\ref{c4:fig:cdf-duration-device-class} shows the \gls{ECDF} for the user tunnels and their \gls{PDP} Context durations in the dataset. In this first graph, the duration of different device classes is distinguished and put in perspective to the overall duration distribution. The devices classes here are smartphones, regular phones, and \gls{3G} dongles. It can be observed that tunnel durations range between seconds and more than one week.
% \approx 6e5 seconds

The median can be clearly differentiated between device types, being much longer for \gls{3G} dongles than for mobile phones. This reflects expected user behavior very well and gives a first indicator on a possible influence of the user plane on the control plane.

Dongles are usually used with laptops to be able to work while being mobile. Therefore dongle sessions last often for extended periods, longer than a few minutes up to hours. Also, this type of device is usually put into a standby mode after the period, which completely disables any mobile connections --- and therefore any associated tunnel --- instead of switching to low power radio idle modes. This is reflected in the dongle tunnel duration here as well. When compared to the other device category, dongles are more compactly centered around their median of about \SI{20}{\minute}.

A similar behavior can be observed in the regular phone distribution with values arranged tightly around the median of \SI{31}{\second}. Compared to today's smartphones, data connections on regular phones are mostly initiated explicitly by user interaction, for example through starting a browser and viewing a web page. 
% "The device does not generate traffic in the background." o.ä.?
This could also explain the comparatively low durations here.

The picture is rather different in the smartphone tunnel duration. Here, often background tasks are running over long periods of time and devices try to keep connectivity up as long as possible (while still attempting to conserve power). Overall, this could lead to the smoother distribution seen here with no clear center value.

Overall, a relatively high number of tunnels with a duration shorter than \SI{10}{\second} can also be observed. Especially the peak at about \SI{1.5}{\second} --- which is interestingly shifted to \SI{6.8}{\second} in the dongle distribution --- is of note. This is even shorter than the default values for the \gls{RRC} idle state transitions which causes the tunnel to be destroyed. It can be conjectured that these short tunnels have been explicitly removed by the \gls{UE} as no other involved state machine has timers this short.

Another distinct step in the total and smartphone distributions can be observed at the \SI{30}{\minute} mark. 
% 1800 s / 75th percentile
As it is only present in these to categories --- and the total distribution looks to be mostly governed by smartphones --- it is reasonable to assume that the cause for this is a specific behavior observable in some aspect of smartphone related influence factors.


%%
\paragraph{Influence of the \texorpdfstring{\acrshort{os}}{OS}}

\begin{figure}[htb]
	\centering
	\includegraphics[width=0.9\textwidth]{images/R-tunnel-duration-operating-system.pdf}
	\caption{Tunnel duration \acrshort{CDF}, separated for select \acrshortpl{os}; Medians at \SI{115}{\second} (total), \SI{15.5}{\second} (symbian), \SI{104}{\second} (iOS), and \SI{765}{\second} (android).}
\label{c4:fig:cdf-duration-os}
\end{figure}

Next, the smartphone and regular phone categories are further broken down by their \gls{os}. Only the three major systems, Android, iOS, and Symbian, are identified here, the amount of other types was negligible. The smartphone category is almost exclusively represented by Android and iOS devices, while Symbian devices make up most of the regular phones but is also represented in a number of smartphone models.

Figure~\ref{c4:fig:cdf-duration-os} depicts the \gls{ECDF} of the tunnel durations of these categories in relation to the total duration distribution. They immediately exhibit a clear difference between individual \glspl{os}.

The Symbian tunnel durations are similarly distributed to the previously depicted regular phone category, albeit with an even shorter duration median of about \SI{15}{\second}. This is an indicator of the large intersection between these two groups and the explicit user traffic property attributed to regular phones.

The two smartphone-exclusive \glspl{os} have remarkably similar tunnel distributions, with the exception of the Android tunnel distribution shifted to much longer tunnels. This is mostly due to the larger accumulation of iOS tunnels around the previously mentioned \SI{1.5}{\second} mark. Over \SI{20}{\percent} of all tunnels established by iOS devices are shorter than \SI{2}{\second}. A possible explanation is an interaction between the described implicit background traffic happening in intervals and the efforts of iOS phones to preserve as much energy as possible. 

To this end, phones aggressively force their radio connection to the low power idle states or even completely shut off the radio immediately after transmission have ended, circumventing \gls{RRC} timers. To achieve this, iOS devices are known to implement a form of \gls{3GPP} Fast Dormancy~\cite{gsma2011fdbestpract}. It is deemed to improve device battery life, radio signaling and radio spectrum efficiency. Due the more frequent state transitions it also could cause an increase in core network tunnel management signaling, which is probably what happened in the iOS case depicted in the \gls{ECDF}.

Another set of tunnel duration accumulations are also visible in the \gls{os} distributions. Two types of steps should be distinguished here. First are accumulations that occur across multiple or all categories. This points to an influence source outside of the specific category. If the artifact is present in every distribution it is even likely that the source is a behavior of the network's state machines. The second type of accumulation is local to one or some categories, which places the root cause into the region of these categories and their related influence factors. 

In case of the \gls{os} category, additionally, peaks at \SI{30}{\second}, \SI{300}{\second}, and \SI{600}{\second} can be observed. However, whether this behavior can be attributed directly to the operating systems themselves cannot be decided just by looking at these distribution. Other factors, e.g., the device's baseband and user traffic dynamics, also play a role. 


%%
\paragraph{Influence of the Time of Day}

\begin{figure}[htb]
	\centering
	\includegraphics[width=0.9\textwidth]{images/R-duration-timeofday-ecdf.pdf}
	\caption{Tunnel duration of all active tunnels by time of day.}
\label{c4:fig:duration-timeofday-ecdf}
\end{figure}

In addition to device factors, diurnal effects could also play a role in the duration of tunnels. Figure~\ref{c4:fig:duration-timeofday-ecdf} depicts individual \glspl{ECDF} of the tunnel duration for four six-hour intervals of a day, starting at midnight. While no clear distinctions are visible, there is a trend towards shorter tunnels in the early morning hours. 
% Außerdem stärker ausgeprägte "automatische" Timeouts bei 30 min etc.?
The early afternoon hours tend to produce tunnels more centered around the middle duration range. Even longer tunnels should be treated with reservation, as they exceed the length of their assigned time slot in the \gls{ECDF} and span a larger time frame. Only the tunnel creation point is guaranteed to be in the slot.


%%
\paragraph{Influence of Other Factors}

Due to the nature of the trace dataset at hand many other influence factors are hard or outright impossible to distinguish. Some factors are unknown from the \gls{CN} perspective, as the mentioned device baseband, while others have not been recorded in the trace.

For example, it would theoretically be possible to investigate the influence of the \gls{RAT} as it is an \gls{IE} in the \gls{gtp} messages recorded in the trace. The radio access parts of \gls{GSM} and \gls{UMTS} are completely different --- including the \gls{RRC} state machines which were depicted in Figure~\ref{c4:fig:mmstatemodel} --- and therefore could also differ in their control plane load impact on the core. However, the \gls{RAT} \gls{IE} is optional and only set in less than \SI{1}{\percent} of the available records. As the radio access can change even during an existing tunnel --- in which case the \gls{GGSN} receives a \gls{gtp} update request informing the node about the change --- a complete picture without gaps would be required to do any investigation on this.
% Was sind die "gaps"? Die fehlenden Updates? Oder die 99%?


%%
\paragraph{Influence Strength of the Categories}

% Hä? Überschrift --> Bild --> Text?

\begin{figure}[htb]
	\centering
	\includegraphics[width=0.9\textwidth]{images/R-duration-qq-category-comparison.pdf}
	\caption{Q-Q Plots of the tunnel duration distributions in comparison to device classification categories.}
\label{c4:fig:qq-plots}
\end{figure}


To ascertain which of the investigated device categories influences the total duration distribution most, Q-Q plots are created and investigated. It is conjectured that the amount of influence on the duration distribution is correlated to the influence on the control plane load. In theory, if both durations follow the same distribution, one expects a straight diagonal $y=x$ line through the origin. A steeper incline indicates more compact regions in the distribution plotted on the $x$ axis and vice versa.

The Q-Q plots in Figure~\ref{c4:fig:qq-plots} compare the total tunnel duration distribution to the duration distribution of the dongle, smartphone, Android, and iOS classification cateogries. It can be observed that the smartphone duration distribution is distributed almost equally to the total except for minor variations. However, the \gls{3G} dongle tunnel durations follow a very different distribution. Their effect on the total duration distribution seems to be negligible despite the  large amount of traffic they are causing. This is also a first indicator that smartphones might have a larger impact on signaling than other device types.

Looking closer at the smartphone category, Q-Q plots of the two major \glspl{os} are investigated. With the exception of the large sub-\SI{2}{\second} peak in the lower tail of the distribution, iOS device tunnel durations are very similar to the overall tunnel duration distribution. The same can not be said about the Android distribution, which deviates somewhat in the distribution's center but is similar to the total distribution in the upper tail. Even devices with just a different \gls{os} seem to strongly differ in their influence on the tunnel duration distribution and therefore on signaling.
% Interessant somit, dass total \approx Summe der Smartphones ist, wo doch iOS != Android. Kompensiert das <2s-Knie den Andro-Buckel?

\begin{figure}[htb]
	\centering
	\includegraphics[width=0.9\textwidth]{images/R-duration-classification-density.pdf}
	\caption{Logscale density plot of the tunnel duration with all classifications.}
\label{c4:fig:durations-density}
\end{figure}


Figure~\ref{c4:fig:durations-density} attempts to depict where in their distributions the investigated device categories show the most impact on the total distribution. The plot shows the density of all previously investigated device influence categories.

It is evident that the durations are not evenly distributed, but rather follow sharp spikes. One of largest spike across all categories is the one at a duration of \SI{30}{\minute}, with about \SI{1.8}{\percent} of all tunnels in the network falling into that region. Since this spike happens across all device types, it makes a rather strong case for being induced by the network. 
% Was spräche gegen, sagen wir einmal, TCP?
On the other hand, the bulk of tunnel durations in the short-to-medium range does not seem to be governed by the two major smartphone operation systems but by other devices in the network, which do not show major spikes in other bins.

Besides the long-tailed behavior in the upper tail of the tunnel durations another slight accumulation effect, repeating itself every \SIrange{6}{7}{\day}, is present in the upper tail. 
% Wodurch drückt sich der Effekt aus? Sieht man den in den Graphen?
This phenomenon is as yet of unknown origin and does not coincide with any known timers of the \gls{3G} mobile network.
% DHCP lease time? Schau mal mit adb logcat, was Dir das Netz beim Einbuchen so liefert!

The investigation of this data leads to the conclusion that the planning and dimensioning of the control plane needs to watch the behavior of smartphones more carefully than that device types.


%%%%%%%%%%%%%%%%%%%%%%%%%%%%%%%%%%%%%%%%%%%%%%%%%%%%%%%%%%%%%%%%%%%%%%%%%%%%%%%
\subsubsection{\texorpdfstring{\acrshort{gtp}}{GTP} Tunnel Arrivals}

The duration of \gls{gtp} tunnels is but one aspect of influence on control plane load. The arrival process of these tunnels is also interesting in itself, specifically the arrival of tunnel requests, i.e., \gls{gtp} create requests, at the \gls{GGSN}. 

An arrival process can be described in two distinct ways. First by the number of arrivals in a given time interval. Second, by the \gls{IAT}, the time between two consecutive arrivals. Depending on the choice one has to deal with either a discrete or a continuous distribution.

Here, the tunnel arrival process is investigated with both approaches. This also adds to the foundation of the load model constructed in the next section. Note that the notion of classifying arrivals into influence categories based on device specifics is omitted here. An category-based investigation of this process can not be realistically be conducted and still relate to the total system load.
% Wieso eigentlich nicht, abgesehen vom Aufwand?

\begin{figure}[htb]
	\centering
	\includegraphics[width=0.9\textwidth]{images/R-create-frequency.pdf}
	\caption{Tunnel arrivals histogram overlaid with a density plot.}
\label{c4:fig:freq-arrivals}
\end{figure}

Figure~\ref{c4:fig:freq-arrivals} depicts a histogram of the number of tunnel arrivals per second during the whole trace duration period. Of note is the clear bimodal nature with one peak around twelve arrivals per second and the other in the low thirties. While the distribution is rather compact around these two peaks, there are some outliers reaching $107$ arrivals per second. If the hypothesis of the correlation between signaling load and number of arrivals holds, it can be assumed that load is not constant but rather switches between two modes with some periods of very high load induced by an increased number of arrivals.

\begin{figure}[htb]
	\centering
	\includegraphics[width=0.9\textwidth]{images/R-createspersecond-1h-violin.pdf}
	\caption{Violin plot of tunnel arrivals in one second per time of day.}
\label{c4:fig:freq-arrivals-per-second-violin}
\end{figure}

A reasonable cause for the occurrence of these two modes can be found in the diurnal arrival patterns. Figure~\ref{c4:fig:freq-arrivals-per-second-violin} contains a violin plot of the tunnel arrivals. This type of plot is similar to a box plot but additionally shows the density of the individual items on the vertical axis. Here the arrivals are broken down to hourly slots. 

The nocturnal plateau of arrivals between midnight and \formattime{5}{0}{0} and the longer daytime plateau between \formattime{8}{0}{0} and \formattime{19}{0}{0} match the two modes found in the histogram. In between are short transition phases. The density of the arrivals during daytime indicates a spread of the number of arrivals over a larger range. This could be an indication of load fluctuations in the system.
% Hmm... cause and effect? Willst Du sagen, Load fluctuations erzeugen den spread, oder umgekehrt, der spread könnte zu fluctuations führen? Korrekt wäre meiner Meinung nach zweiteres.

\begin{figure}[htb]
	\centering
	\begin{subfigure}[b]{0.5\textwidth}    
		\includegraphics[width=\textwidth]{images/R-IAT-successful-2h-ecdfs.pdf}
		\caption{All tunnel requests.}
		\label{c4:fig:IAT-ecdf-2h-successful}
	\end{subfigure}%
	~
		\begin{subfigure}[b]{0.5\textwidth}
		\includegraphics[width=\textwidth]{images/R-IAT-fromflows-ecdfs-2h.pdf}
		\caption{Only tunnels with data flows.}
		\label{c4:fig:IAT-ecdf-2h-active}
	\end{subfigure}

	\begin{subfigure}[b]{0.5\textwidth}
		\includegraphics[width=\textwidth]{images/R-IAT-fromflows-gprs-ecdfs-2h.pdf}
		\caption{Tunnels with data flows initiated in \gls{GPRS}.}
		\label{c4:fig:IAT-ecdf-2h-active-gprs}
	\end{subfigure}%
	~
	\begin{subfigure}[b]{0.5\textwidth}
		\includegraphics[width=\textwidth]{images/R-IAT-fromflows-umts-ecdfs-2h.pdf}
		\caption{Tunnels with data flows initiated in \gls{UMTS}.}
		\label{c4:fig:IAT-ecdf-2h-active-umts}
	\end{subfigure}
	\caption{\acrshortpl{ECDF} of the tunnel \acrshort{IAT} in seconds by time of day.}
% ... and an additional parameter (as noted)
\label{c4:fig:IAT-ecdf-2h}
\end{figure}

Complementing the arrival rate evaluation is the investigation of the tunnel \gls{IAT}. This metric is more sensitive to short time fluctuations of arrivals and more suited to describe the arrival process for use in the proposed load model.
% Möchte man erwähnen, dass IAT und # Tunnel pro Zeit reziprok zu einander sind, <<<IAT daher >>>Anzahl Tunnel heißt, und alle "low"s und "high"s dementsprechend in ihren Auswirkungen auf die Last genau zu interpretieren sind?

The overall picture of all arrivals is given in the \gls{ECDF} of Figure~\ref{c4:fig:IAT-ecdf-2h-successful}, again broken down by time of day. Obviously the same previously observed diurnal load variation can again be perceived. The median \glspl{IAT} fall in the range of \SI{20}{\milli\second} and \SI{60}{\milli\second}, enveloped by the distribution for the hours starting at \formattime{16}{0}{0} and \formattime{2}{0}{0} having the lowest and highest \gls{IAT} respectively. Additionally, tunnel arrivals are occurring at an increased frequency with an interval of multiples of \SI{20}{\milli\second}, which generates these wave-like steps in the \gls{ECDF} plot. As this is happening very regularly at every time of the day, a source inside the mobile network is indicated.

A hypothesis as to the origin of this effect is the value of the \gls{TTI}. This property determines the duration of a mobile network's radio transmission slot. In \gls{3GPP} standards up to \gls{UMTS} the default value of the \gls{TTI} is either \SI{10}{\milli\second} or \SI{20}{\milli\second}, newer versions of the specification additionally allow values of \SI{2}{\milli\second} (in \gls{HSPA}) or even \SI{1}{\milli\second} (for \gls{LTE}). The absolute time of every transmission slot is also synchronized across every base station in the whole mobile network, which makes the \gls{TTI} noticeable even when not measuring directly at the radio link. 

The observed step-width of \SI{20}{\milli\second} therefore indicates that the tunnel establishment signaling procedure includes at least one trip from the mobile device over the radio interface. This makes sense, as the tunnel is typically created during the \gls{GPRS} Attach procedure, which is indeed initiated at the user's device. Unfortunately, this gives the arrival process batch properties. As a result the load at the \gls{GGSN} increases momentarily when a batch arrives. The \gls{GGSN} would then need to process more requests simultaneously than if the arrivals followed a smooth stochastic distribution.

This effect becomes more peculiar when the tunnel arrivals are further broken down. Figure~\ref{c4:fig:IAT-ecdf-2h-active} only displays arrivals of tunnels that actually transported user traffic during their lifetime. Here, the influence of the effect is visually unnoticeable. This could be attributed to the fact that most active data connections during the time of the trace recording were already using almost exclusively \gls{HSPA} or better, which sees the much lower \gls{TTI}. Only older, regular phones establish plain \gls{UMTS} connections and often do not even use it.

The discrimination of the \gls{IAT} distribution by \gls{RAT} that was used during the creation of the tunnel reveals no further information. Due to the much lower number of connections the \gls{GPRS} distributions are shifted to much higher intervals than the \gls{UMTS} specific distributions.


%%%%%%%%%%%%%%%%%%%%%%%%%%%%%%%%%%%%%%%%%%%%%%%%%%%%%%%%%%%%%%%%%%%%%%%%%%%%%%%
\subsection{\texorpdfstring{\acrshort{gtp}}{GTP} Tunnel Message Processing Time}

Finally, the \gls{GGSN}'s processing time of \gls{gtp} tunnel management messages is investigated. Potentially, this can be a direct measure of the load at the node. In times of higher load one would expect a higher processing time of signaling messages.

From the network trace the processing time can be calculated by two timestamps in each record.
As the trace is recorded at the Gn interface these timestamps represent the points in time the \gls{gtp} signaling request and subsequent response pass on the link to and from the \gls{GGSN}. Therefore, they can also be interpreted as the start and finish of the involved processing at the \gls{GGSN}.

Generally, the processing time of all three message types --- i.e., creates, deletes and updates --- could be calculated. It would be of special interest to know what influences the setup time of tunnels, as this is one of the \gls{GGSN}'s most time-sensitive jobs and can impact the time a user has to wait before being able to actually transfer data. Unfortunately, due to unrecoverable issues with the recording of the dataset, the timestamps for both the create and delete messages records were completely unreliable and did not allow for an investigation of the processing time.

Only \gls{gtp} update messages were unaffected by this mishap, and gave the opportunity for further investigation. The trace contains roughly two orders of magnitude more update messages than either creates or deletes, spread out almost evenly over the whole observation period. Therefore, a node load investigation should still be possible with just the updates messages.

\begin{figure}[htb]
	\centering
	\includegraphics[width=0.9\textwidth]{images/R-update-time-cdfs.pdf}
	\caption{\acrshortpl{ECDF} of the time in seconds it takes a \acrshort{GGSN} to process a \acrshort{gtp} update event, separately plotted for four time slots each day.}
	\label{c4:fig:update-time}
\end{figure}

Figure~\ref{c4:fig:update-time} depicts a band of \glspl{ECDF} for the processing time of update messages by time of day. The processing time distribution almost perfectly follows a continuous uniform distribution between \SI{2}{\milli\second} and \SI{22}{\milli\second}. Only the upper end displays a slight long-tail behavior. The impact of the time of day is very slim with slightly higher processing times during the evening, the same time frame which also experienced an elevated arrival rate.

The occurrence of a continuous uniform distribution is rather unexpected as these do not usually occur in computing processes. According to the central limit theorem one would rather expect to see a normal distribution influenced by, e.g., process scheduling or other queuing artifacts. The source of this effect is still unknown, and the current dataset does not allow for a more thorough investigation. Still, the fact that a higher update processing time coincides with an increase in the arrival rate points to an influence of tunnel messaging on the load of a \gls{GGSN}.



%%%%%%%%%%%%%%%%%%%%%%%%%%%%%%%%%%%%%%%%%%%%%%%%%%%%%%%%%%%%%%%%%%%%%%%%%%%%%%%
\subsection{Statistical Evaluation and Data Fitting}
\label{c4:sec:statistical_evaluation}

The uncovered empirical distributions for both the tunnel duration and the tunnel \gls{IAT} are now to be matched against theoretical probability distributions. Therefore, a univariate distribution fit to the experimental data was conducted. Having a concise representation for the empirical data will help in creating a model of the core network, which is the task conducted in the sections following after this.


%%
\paragraph{\gls{IAT} Fitting}

\begin{figure}[htb]
	\centering
	\includegraphics[width=0.9\textwidth]{images/R-IAT-ecdfs.pdf}
	\caption{Sampled inter-arrival time \acrshort{CDF} and fitted theoretical distributions.}
\label{c4:fig:IAT-cdfs}
\end{figure}

In order to investigate the tunnel \gls{IAT}, Figure~\ref{c4:fig:IAT-cdfs} displays the overall \gls{ECDF} with fits for various basic probability distributions. Each of the fits was generated through the method of moments matching.

The goodness of these fits was checked both visually using the \glspl{CDF} plots and numerically with goodness of fit measures, using Pearson's correlation coefficient and Pearson's $\chi^2$ test. Unfortunately, none of the probability distributions reaches the significance level for $\chi^2$. This can probably be largely attributed to the various previously described artifacts in the data. Matching them visually, the exponential fit seems to be reasonably close to the experimental data.

\begin{figure}[htb]
	\centering
	\includegraphics[width=0.9\textwidth]{images/R-IAT-active-fit-cdf-facets.pdf}
	\caption{Empirical and exponentially fitted \acrshortpl{CDF} of the tunnel \acrshort{IAT} by time of day. \acrshortpl{CDF} are overlapping as the coefficient of determination is close to $1$.}
\label{c4:fig:pdparrivalsecdf}
\end{figure}

To improve the fits, two modifications were made to the process. First, to remove the \SI{20}{\milli\second} steps, only the active tunnels were taken into consideration. Second, the overall \gls{IAT} distribution was once again split up into time of day slots. 
% Kann man den Unterschied, den das macht, sinnvoll beziffern? 3.8(b) schaut ja doch deutlich anders aus als (a).
The overall distribution is just a superimposition of the individual slots anyway. Therefore, this should further improve the fidelity of the fits.

\begin{table}[htb]
\caption{Parameters for the exponentially distributed inter-arrival times and corresponding Pearson correlation coefficients.}
\label{c4:tab:IAT-fits}
	\centering
	\begin{tabu}{X[0.9,l]X[r]X[r]} 
	\toprule
	\textbf{Time of Day} & $\mathbf{\lambda}$ & $\mathbf{R_{arrival}}$\\ 
	\midrule
	0h-5h   & $10.67477$ & $0.995$ \\
	6h-11h  & $24.53298$ & $0.992$ \\
	12h-17h & $29.2504$  & $0.993$ \\
	18h-23h & $23.49983$ & $0.986$ \\
	\bottomrule
	\end{tabu}
\end{table}
% Blöde Frage, aber wo sind die Mittdreißiger, die im Histogramm vorher vertreten waren?

The results are depicted in Figure~\ref{c4:fig:pdparrivalsecdf}. To improve plot visibility only four larger time slots are displayed here while the actual fits were conducted for one-hour slots. Parameters for the exponential distribution $F(x) = 1- e^{-\lambda x}, x \geq 0$ and the corresponding correlation coefficients to the original data for the four time slots are given in Table~\ref{c4:tab:IAT-fits}. The fitted functions match the empirical data quite well, with some deviation present at the left tail but an overall positive correlation coefficient approaching $1$.


%%
\paragraph{Duration Fitting}

The second fitting effort surrounds the empirical data concerning the tunnel durations. However, none of the basic probability distributions (including exponential, gamma, and Weibull distributions) fit the tunnel duration even remotely. One of the reasons for this is probably that the tunnel duration is influenced by an overwhelming amount of factors, which were previously described. This superposition, especially with the user behavior, will result in unpredictable results that does not follow any basic probability distribution.

\begin{figure}[htb]
	\centering
	\includegraphics[width=0.9\textwidth]{images/R-duration-fit-cdf-facets.pdf}
	\caption{Empirical and fitted \acrshortpl{CDF} of the tunnel duration by time of day with fitted rational functions.}
\label{c4:fig:fittedsdurationlots}
\end{figure}

Instead, rational functions are fitted to the \glspl{ECDF} using the proprietary third-party tool \textit{Eureqa}~\cite{eureqa_paper, eureqa_software}. This allows for a much closer fit as seen in Figure~\ref{c4:fig:fittedsdurationlots}, but limits its application in the statistical evaluation.

\begin{table}[htb]
\caption{Inverse rational functions fitted to the \acrshortpl{ECDF} of the tunnel duration by time of day and correlation coefficients of the fit.}
\label{c4:tab:fits-duration}
	\centering
	\begin{tabu}{X[1.1,l]X[4.5,r]X[r]} 
	\toprule
	\textbf{Time of Day} & \textbf{Inverse Serving Time \gls{CDF} Representation} & $\mathbf{R_{dur}}$\\ 
	\midrule
	0h-5h & $0.919 - 60.614y - 3498.78y^3 - \frac{110.707y + 2289.94y^3}{y - 1.005}$ &  $0.999$ \\
	6h-11h & $1 + 117.484y - 368.643y^2 - \frac{1720.13y^4}{y - 1.004}$ & $0.999$ \\
	12h-17h & $0.953 + 69.491y + \frac{81146.1y^3 + 1.086\times10^6y^5}{805 - 802.01y}$ & $0.999$ \\
	18h-23h & $0.912 + 82.056y - \frac{2936.93y^4}{1.945y - 1.953}$ & $0.999$ \\
	\bottomrule
	\end{tabu}
\end{table}

	% high precision:
	% 0h-5h & $0.919208 - 60.6136y - 3498.78y^3 - \frac{110.707y + 2289.94y^3}{y - 1.00469}$ &  $0.999$ \\
	% 6h-11h & $1 + 117.484y - 368.643y^2 - \frac{1720.13y^4}{y - 1.0041}$ & $0.999$ \\
	% 12h-17h & $0.952566 + 69.4907y + \frac{81146.1y^3 + 1.08572\times10^6y^5}{805 - 802.01y}$ & $0.999$ \\
	% 18h-23h & $0.911924 + 82.0562y - \frac{2936.93y^4}{1.94468y - 1.9532}$ & $0.999$ \\

Table~\ref{c4:tab:fits-duration} contains the functions which were fitted to the \textit{inverse} \gls{CDF}. The inverse was chosen here to simplify the modeling and simulation process coming afterwards. The functions can be easily inverted again for other purposes. Both the \glspl{CDF} in the plot as well as the Pearson correlation coefficient, which again approach $1$, confirm the goodness of the fitted functions.


% \begin{table}
% \centering
% \caption{TAC Statistics}
% \begin{tabu}{|X|X|X[1.5]|X|X|X|} \hline
% & \textbf{\# of Flows} & \textbf{Total Traffic (Bytes)} &  \textbf{\# of Tunnels} & \textbf{\# of GTP Signalling Msgs} & \textbf{\# of Distinct IMSIs}\\ \hline
% Total          & 2234659247 & 122758578593993 (112TB)    & 16632094 & 409733865 & 1255293 (all) / 1030895 (with flows) \\ \hline
% In TAC DB      & 2228315260 & 122716712007150 (111.61TB) & 14565430 & 372662108 & 1015891 \\ \hline
% Smartphones    & 459990512  & 15721818747754 (14.30TB)   & 10030734 & 311342846 & 476675  \\ \hline
% Regular phones & 5705832    & 448140315058 (0.41TB)      & 897529   & 3860162   & 116124  \\ \hline
% 3G dongles     & 1487230062 & 92215931895630 (83.87TB)   & 2114756  & 39053819  & 315003  \\ \hline
% Android        & 241973565  & 7953178401958 (7.2TB)      & 2383255  & 177537567 & 175919  \\ \hline
% iOS            & 161408903  & 5481693567152 (5TB)        & 3145384  & 83374590  & 99679   \\ \hline
% Symbian        & 22827418   & 1332996529271 (1.21TB)     & 3520242  & 18479002  & 162790  \\ \hline
% Blackberry OS  &            & 128074907884 (0.12TB)      &          &           &         \\ \hline
% \end{tabu}
% \end{table}


%Devices with GTP signaling but no user plane traffic: (\#distinct imsis gtp db)-(\#distinct imsis flow db):
% $255293-1030895=224398\text{ or }17.88\%$

%%%%%%%%%%%%%%%%%%%%%%%%%%%%%%%%%%%%%%%%%%%%%%%%%%%%%%%%%%%%%%%%%%%%%%%%%%%%%%%%
%\subsection{Correlations to User Traffic}
% TODO, incl. measurements



%%%
% Direct signaling traffic overhead in relation to user traffic and induced network load

% GTP Header: 12 Byte
% IE header and footer: 2 Byte
% Maximum minimum data size including all \glspl{IE}: 221 Byte + 12 Byte Header + 2*37 Extension Header = 307 Byte
% Minimum size of message with just mandatory \glspl{IE}: 12 + 30 + 2*5 = 52 Byte

% 307 Bytes:
% calculation from our dataset
% Total maximum signaling traffic with this calculation: 117.15GB
% Ratio: 0.10\%
% 52 Bytes:
% Total maximum signaling traffic with this calculation: 19.84GB
% Ratio: 0.02\%
% Total traffic: 122758578593993


% signaling calc:
% gtp signaling traffic volume estimation $v_s$ = (1059B gtp message + 8B udp header + 20B ipv4 header) = 1087B * 409733865 number of request/response pairs * 2 (2 messages per pair) = 8,9076142E11B
% ratio to total $r=\frac{v_s}{v_t}=0.72\%$  $v_t=122758578593993B$


%%
%TODO: radio access type plots, if we have the data
% We do not.
%
%\begin{figure}
%\centering
%\includegraphics[width=\columnwidth]{figures/tunnel-dur-radio-cdf-mod.pdf}
%\caption{Tunnel duration distribution, separated for UMTS and GPRS radio access [NOTE: only in the last tunnel segment; and majority of radio types %is unknown anyway.}
%\label{fig:cdf-duration-radio}
%\end{figure}


%%%%%%%%%%%%%%%%%%%%%%%%%%%%%%%%%%%%%%%%%%%%%%%%%%%%%%%%%%%%%%%%%%%%%%%%%%%%%%%%
%!TEX root = ../../dissertation.tex
%%%%%%%%%%%%%%%%%%%%%%%%%%%%%%%%%%%%%%%%%%%%%%%%%%%%%%%%%%%%%%%%%%%%%%%%%%%%%%%
\section{Modeling Mobile Network Load}
\label{c4:sec:modeling}

The next logical step after the collection of empirical data and the execution of a statistical analysis lies in the creation of models abstracting this real system. While some loss of precision is incurred, models are much more flexible and can have numerous applications. Load models and the derived information on the network \gls{QoS} parameters can serve as a basis for the video measurement framework of Section~\ref{c3:sec:measurements}, which uses arbitrarily chosen values for its latency and loss experiments. Using the load model a more realistic mobile network could be emulated. Additionally, network operators can also be supported in predicting the signaling load in their core network with the benefit of improved network engineering and correctly scaling core components.

On the basis of the tunnel distributions attained in Section~\ref{c4:sec:evaluations}, models for both a traditional \gls{GGSN} as well as a virtualized \gls{GGSN} are introduced. The performance trade-offs when using a virtual \gls{GGSN} are further studied, discussing different options to consider when using the virtual node.

The modeling and simulation of the resulting models was conducted in cooperation with the University of Würzburg and partially published in \cite{metzger2014lossmodel}.

%\cite{trangia-lbvs}

%%%%%%%%%%%%%%%%%%%%%%%%%%%%%%%%%%%%%%%%%%%%%%%%%%%%%%%%%%%%%%%%%%%%%%%%%%%%%%%
\subsection{Queuing Theory Basics}

To understand the modeling process some knowledge on queuing theory is required. The next few sections give a short overview on the definitions used in the subsequent sections.


%%
\subsubsection{Little's Law}

A basic queuing system can be expressed as a stream of customers arriving at an arbitrary system with a rate $\lambda$. This system then processes the customers, taking an average time of $W$ on a number of processors until the customers depart again. On average $L$ customers will be in the system. The representation --- and queuing theory in general for that matter --- was originally devised for telephone networks by Erlang~\cite{erlang1917solution}. From that, Little's~Law~\cite{little1961proof} can be formulated as

\begin{equation}
	\phantom{,}L = \lambda W\text{,}
\end{equation}

which holds universally, independent of any specific arrival or service time process.


%%
\subsubsection{Kendall's Notation}

To distinguish the variations of a queuing system's parameter a simple convention and naming scheme was devised by Kendall in 1953~\cite{kendall1953stochastic} and later extended on.  In its simplest form the notation reads $A/S/s$ with $A$ denoting the arrival distribution, $S$ the service time distribution, and $s$ the number of servers. Here, an extended notation will be used, 

\begin{equation}
	A/S/s/q
\end{equation}

which additionally describes the queue length $q$. With this naming scheme, a queuing system ($q=\infty$) can be easily distinguished from a blocking or loss system ($q=0$). The most commonly used arrival processes and service time distributions are summarized in Table~\ref{c4:tbl:kendalldistributions}.

\begin{table}[htb]
\caption{Typical abbreviation of processes in Kendall's notation.}
\label{c4:tbl:kendalldistributions}
	\begin{tabu}{X[l]X[7]}
	\toprule
	\textbf{Symbol} & \textbf{Description} \\
	\midrule
	$M$ & Markovian, i.e., Poisson, arrival process or exponential service time distribution \\
	$D$ & Deterministic arrival process or service time distribution \\
	$G$ & General arrival process or service time distribution with no special assumptions \\
	$GI$ & General arrival process with independent arrivals; also called regenerative \\
	\bottomrule
	\end{tabu}
\end{table}


%%
\subsubsection{Information Gain}

Depending on the complexity of the specific queuing system model, much information can be gained from an analysis of the given model. In simple cases the state probability can be mathematically determined, i.e., the probability that exactly $m$ customers are in the system concurrently. If this number is higher than the number of processors, this also determines the queue length, or the blocking probability $p_B$ if there is no queue. Other properties include for example the waiting time of customers.

\begin{figure}[htb]
	\centering
	\includegraphics[width=1.0\textwidth]{images/markovchain.pdf}
	\caption{$M/M/c/\infty$ Markov chain model.}
\label{c4:fig:markovchain}
\end{figure}

One such basic queuing system is $M/M/1/\infty$~\cite[pp.~94-99]{Kleinrock:1975:TVQ:1096491}, on which stationary analysis can be applied upon. Both the one processor queue and $M/M/c/\infty$ can also be easily expressed as a Markov chain due to their memoryless property. Figure~\ref{c4:fig:markovchain} depicts the state transitions of a system with $c$ processors and a queue length of $i-c$.
% Bin mit der Queue Length nicht ganz einverstanden. Für i<c gibt's doch keine Queue. Außerdem fehlen mir für ein .../.../.../\infty-System noch \dots rechts vom State i.

More complex models are often not tractable by stationary analysis or other mathematical tools any more and no general solution is known. This is especially true for the class of $G/G/c$ systems, which can only be directly solved under certain conditions. System parameters may still be investigated using numerical queuing simulation. Here both the arrival and the serving process are implemented in a \gls{DES} using random numbers drawn from the desired distributions in order to determine the system load and blocking probability.


%%%%%%%%%%%%%%%%%%%%%%%%%%%%%%%%%%%%%%%%%%%%%%%%%%%%%%%%%%%%%%%%%%%%%%%%%%%%%%%
\subsection{GGSN Model Rationale and General Queuing Theoretic Representation}

The \gls{GGSN} was already determined to be critical to the \gls{CN}'s load. Therefore, the network will be represented by this node in the model. Additionally, most of the load influencing factors are at least to a degree related to the \gls{gtp} tunnels. So, to dimension a mobile network based on its control plane load, the number of supported tunnels has to be modeled. 

\begin{figure}[htb]
	\centering
	\includegraphics[width=0.6\textwidth]{images/GGn-model.pdf}
	\caption{Queuing system representation of a mobile network's \acrshort{GGSN}.}
\label{c4:fig:ggn-model}
\end{figure}

Figure~\ref{c4:fig:ggn-model} shows this model for the proposed tunnel load metric. It is in its generic form a $G/G/c/0$ system. Tunnels enter the system governed by a general random distribution, and are served at the \gls{GGSN} for the duration of their existence. This duration also follows a general distribution. Afterwards, tunnels leave the system again through the reception of a \gls{gtp} tunnel delete message. If all $c$ serving units are filled, blocking occurs, and arriving tunnel requests are rejected.

The number of serving units corresponds to available resources at the \gls{GGSN}. The maximum supported number of concurrent tunnels is hard to estimate as it depends on a number of factors, most of which are unknown for this modeling process. This could include soft-limits like the specific configuration, and hard-limits, e.g., the \gls{GGSN}'s processing and memory constraints.

For the purpose of creating an initial toy model the generic $G/G/c/0$ is simplified to a $M/M/\infty$ system. As stated, no actual limit to the number of virtual servers is known and the data also does not indicate any obvious limits. Thus, an unlimited system with neither blocking nor queuing is assumed for this simple model.

Now, assuming both a Poisson arrival and an exponential serving process (temporarily neglecting the fact that no basic function matched the \gls{GGSN}'s serving process), a stationary analysis can be conducted. As seen in the statistical evaluation, the former condition may hold, but the serving time is definitely not exponentially distributed. However, for the toy model this assumption is still made to get an initial grasp of the model.

The diurnal influences seen in the tunnel arrivals in the trace data are also temporarily ignored and only the overall empirical distribution is taken into account. Through distribution fitting with moment matching the overall arrival rate is set to be $\lambda\approx 25.641$ in the trace. The exponential service time distribution is calculated to have the parameter $\mu\approx \num{1.587e-4}$. Using Little's Law this gives an estimate for the mean number of concurrent tunnels at the \gls{GGSN} in a $M/M/\infty$ system of 

\begin{equation}
	\phantom{.}L=\frac{\lambda}{\mu}\approx 161.6\text{.} %=161598.14.
\end{equation}

As stated, the amount of state held at the node and propagated through the network is directly related to the number of tunnels. Therefore, this metric can serve as an initial estimate of the load at the \gls{GGSN}.


%%%%%%%%%%%%%%%%%%%%%%%%%%%%%%%%%%%%%%%%%%%%%%%%%%%%%%%%%%%%%%%%%%%%%%%%%%%%%%%
\subsection{Representative GGSN Models} 

With the experience from the toy model at hand more appropriate models can now be constructed to better accommodate for the core network's properties. Two models are provided here.
The first describes a monolithic version of a \gls{GGSN}, closely resembling the system used traditionally in the network. The second model is that of a hypothetical virtualized \gls{GGSN} using \gls{NFV}. In \gls{NFV}~\cite{nfv_whitepaper} custom monolithic network nodes are replaced by commodity hardware. The tasks solved by the original hardware is migrated to a pure software implementation.


%%
\subsubsection{Monolithic \texorpdfstring{\acrshort{GGSN}}{GGSN}}

The \gls{3gpp} architecture considers the \gls{GGSN} to be one fixed monolithic entity, even if in reality it often consists of multiple servers. The entire \gls{GGSN} is purchased from a vendor as a single entity, they do not integrate well with other existing network infrastructure. 
% Was soll uns der zweite Halbsatz sagen?
Nor can idle instances be deactivated or reused for other purposes.

\begin{figure}[htb]
	\centering
	\includegraphics[width=0.3\textwidth]{images/ggsn-monolithic.pdf}
	\caption{Traditional control plane load modeling approach to a \acrshort{GGSN}.}
\label{c4:fig:model-ggsn-monolithic}
\end{figure}

The queuing theoretic equivalent is displayed in Figure~\ref{c4:fig:model-ggsn-monolithic} and is very similar to the basic toy model. New tunnels requests arrive according to a Poisson distribution with a rate of $\lambda(t)$ at the \gls{GGSN}. The periodic time-of-day dependence of these exponentially distributed \gls{IAT} and the corresponding distribution fits were extrapolated from the trace data.

Furthermore, the model has a maximum tunnel capacity of $c$. When this capacity is reached, blocking will occur and further incoming tunnels are rejected. The governing factors of the capacity are mostly the node's available memory and processing capabilities. Monolithic \glspl{GGSN} need to be preemptively dimensioned in such a way that blocking rarely happens, often resulting in gross overdimensioning as the node can not be easily scaled after it has been deployed.

When an incoming tunnel request is accepted one of the \gls{GGSN}'s serving units will be occupied for the tunnel's duration $x(t)$. Following the trace data, this duration is assumed to be of an arbitrary, non-Markovian service time distribution, again with a slight time-of-day dependence.

Combining the model with the exponential and rational function fits functions previously depicted in Tables~\ref{c4:tab:IAT-fits} and \ref{c4:tab:fits-duration} this results in a \textbf{non-stationary Erlang loss model}, or more precisely

\begin{equation}
	\phantom{.}M(t)/G(t)/c/0\text{.}
\end{equation}

With this model, high control plane load can be indirectly described as the system's blocking probability $p_B$. The peak load can be ascertained by looking at the busy hour period where the arrival rate is the highest. No exact mathematical solution is known for this type of model. Only if the service time distribution can be confined to certain specific distributions, e.g. Phase-type distributions, some approximations can be made \cite{davis1995nonstationaryerlang}. 
However, the evaluated trace data does not give any indication of the presence of such a Phase-type in the the service time distribution. Therefore, the model falls back to a non-stationary general distribution  and a simulative approach to evaluate the model will be taken in Section~\ref{c4:sec:simulation}.


%%
\subsubsection{Virtualized \texorpdfstring{\acrshort{GGSN}}{GGSN}}
\label{c4:sec:virtual_ggsn}

\begin{figure}[htb]
	\centering
	\includegraphics[width=0.6\textwidth]{images/ggsn-virtualized.pdf}
	\caption{Model of a \acrshort{GGSN} using \acrshort{NFV}.}
% Hmmm.... ein Kastl gefüllt mit schwarzen Kastln?
\label{c4:fig:model-ggsn-virtualized}
\end{figure}

In the second model virtualization concepts are introduced. The assumptions of the non-stationary Markov arrival process $\lambda(t)$ and the serving time distributions $x(t)$ are carried over. However, instead of one server processing every tunnel, the system is now partitioned into individual server instances coordinated by a load balancer in Figure~\ref{c4:fig:model-ggsn-virtualized}. 

One virtual \gls{GGSN} has up to $s$ servers instances $s_i$. Each of the individual instances can be much smaller than the monolithic \gls{GGSN}, having a concurrent tunnel serving capacity of $c_i \ll c$ and a total system capacity of $c_{virt} = \sum_{i=1}^{s} c_i = \| \overrightarrow{c}\|_1 \text{ with } \overrightarrow{c} = \{c_1, c_2, \ldots ,c_i, \ldots ,c_s\}$. % 1-norm des Kapazitätsvektors
The complete model now reads:

\begin{equation}
	\phantom{.}M(t)/G(t)/\|\overrightarrow{c}\|_1/0\text{.}
\end{equation}

The instances do not have a static uptime. Instead, their life cycle is managed by a \gls{VMM} and adjusted to the current load of the network. New tunnels are either placed on running instances or new ones are provisioned on demand. The \gls{VMM} can have multiple optimization goals. A prominent example is the minimization of server instance and energy usage.  Another set of example provisioning rules is discussed in the implementation of the model simulation in Section~\ref{c4:sec:simulation}. 

A target criterion could be to keep the blocking probability inside a certain target range. If the \gls{VMM} decision rules are not carefully selected additional blocking could occur. Despite not having reached its maximum capacity, this system will still reject tunnel requests during the provisioning phase when no tunnel slots are free. This could be remedied by a request queue. However, this makes the system more complex without providing real benefit, as failed tunnel requests are retransmitted by the network control plane or another attempt might also be made directly by the mobile device after a timeout.

To place incoming tunnel state on one of the available servers and manage the servers a load balancer is required. To ensure that the system can scale down to its actual needs, the balancer should place tunnels on servers that are the fullest, keeping the reserve free. It may even migrate tunnel state from almost empty servers away so that these can be shut down, when certain conditions are fulfilled. Keeping instance close to their capacity should also have no impact on the performance a mobile device associated to a specific tunnel experiences. Adequate strategies for both load balancing and migration should be considered in subsequent research.

Through this virtualized model, which suggests to use technologies from cloud computing in the network and replace specialized nodes with commodity hardware, network operators can scale the \gls{GGSN} out instead of only up. Today, these network components are typically sold in a static and monolithic form and can not be easily extended with off-the-shelf hardware in order to accommodate to a changing environment. The system in this model can however be easily scaled out to additional low cost machines instead of completely replacing the existing \gls{GGSN} with a more powerful version. 

It is also entirely possible that the described single-arrival-process approaches might not the best way to describe control plane load. Several load influencing factors discussed earlier have direct influence on the tunnel arrivals and duration, e.g., the device type or the radio access technology. Therefore, amongst others, multidimensional queuing networks or fluid flow models could be more appropriate in subsequent research. Still the non-stationary Erlang loss model described here should be sufficient for basic core network control plane load estimations.




%%%%%%%%%%%%%%%%%%%%%%%%%%%%%%%%%%%%%%%%%%%%%%%%%%%%%%%%%%%%%%%%%%%%%%%%%%%%%%%%
%!TEX root = ../../dissertation.tex
%%%%%%%%%%%%%%%%%%%%%%%%%%%%%%%%%%%%%%%%%%%%%%%%%%%%%%%%%%%%%%%%%%%%%%%%%%%%%%%
\section{Load Model Queuing Simulation} 
\label{c4:sec:simulation}

As discussed, the solvability of a non-stationary Erlang loss system is very limited. To better tackle this, a simulative approach can be taken. Depending on the level of detail, different types of simulations are available.

Here, a queuing simulation is used to ascertain the blocking probability and tunnel serving slot utilization from the model using the fitted distributions from the trace.


%%%%%%%%%%%%%%%%%%%%%%%%%%%%%%%%%%%%%%%%%%%%%%%%%%%%%%%%%%%%%%%%%%%%%%%%%%%%%%%
\subsection{Queuing Simulation Implementation}

The queuing simulation is implemented on the basis of a \gls{DES}. Instead of reproducing continuous time, this simulation is a series of discrete events. Time is advanced only at these events. 

A queuing model can be easily represented in a \gls{DES}.  Each tunnel request arrival is modeled as a discrete event. When such an event occurs, three processes are executed. The first process draws a random number from a \gls{PRNG} mapped to \gls{IAT} exponential distribution to schedule the next arrival event. Secondly, the serving units are checked for any free units. If one is found, it will now be occupied. Else, this arrival will be marked as rejected and the third action skipped. This third process now determines the length of the tunnel using another \gls{PRNG} adjusted to the serving time distribution to schedule the event in which the tunnel exits the system.

This model was implemented on the basis of version 3.0 of the \textit{SimPy}\footnote{\url{https://simpy.readthedocs.org/}} package, which is a Python \gls{DES} framework that provides the basic event and scheduling infrastructure. On top of this a base \gls{GGSN} class was constructed, managing the arrival of tunnel events and the scheduling of the service ending events. Specific classes for the traditional (i.e., monolithic) and virtualized (called ``multiserver'' in the code) nodes respectively exist.\footnote{The implementation is also publicly available at \url{https://github.com/fmetzger/ggsn-simulation/} as a reference.}


%%%%%%%%%%%%%%%%%%%%%%%%%%%%%%%%%%%%%%%%%%%%%%%%%%%%%%%%%%%%%%%%%%%%%%%%%%%%%%%
\subsection{Description and Design of the Individual Experiments}

To match the measurement data the simulation time is set to be \SI{7}{\day} in all simulation scenarios. The initial \SI{60}{\minute} of each experiment are considered to be the transient phase and are afterwards excluded from the results. Ten replications of each scenario were performed. All depicted error bars show the \SI{95}{\percent} confidence intervals across the experiments.

The first experiment was conducted to investigate the normalized 
% Was bedeutet das nochmal?
baseline load a monolithic \gls{GGSN} experiences using the presented model. Using this, an upper limit to the number of concurrent tunnels and its correlation to the blocking probability and tunnel rejection rate can be established. The effects of scaling up, improving the hardware capabilities of the single node, can thus be investigated afterwards.

Based on these results, the virtualization and scaling out effects in the virtualized, \gls{GGSN} model are examined. In order to study the feasibility of this approach the performance indicators of the virtual \gls{GGSN} are compared to the baseline established in the first experiment. To this end, the virtual \gls{GGSN} is simulated in several configurations, which vary the number of instances and supported concurrent tunnels per instance.

In a final experiment the startup and shutdown duration of virtual instances and the life cycle management of these instances are additionally taken into account. Although the boot duration of modern \glspl{os} and \glspl{VM}, especially on current hardware with flash storage, 
% \gls{SSD} anyone?
is significantly lower than what past systems could achieve, there is still a delay. This could cause further blocking if the load balancer does not account for this. On the other hand, the more generously in advance the balancer starts instances the smaller the virtualization efficiency gain, especially the energy savings, will be become. For this reason, the number of active instance is a relevant performance metric in the virtual \gls{GGSN} model.

The experiment varies the boot delay and implements a very simple load balancer rule as baseline. The rule keeps at least one empty instance running in reserve at all times, and deactivates instances when two running instances are completely unused. As this is very generous, virtualization blocking should only occur in cases of instances limited to handling a low maximum number of tunnels, or very rapid arrivals. Realistic provisioning rules can improve on this quite easily. But even this simplistic approach already serves to demonstrate potential benefits.


%%
\subsubsection{GGSN Load, Capacity, and Scaling}

First, with the help of the \glspl{IAT} and duration of tunnels calculated in the dataset evaluation, the monolithic \gls{GGSN} model is studied. 
% Ist die Annahme, dass das echte Ding nicht blockt, oder muss man davon ausgehen, dass was in diesem Datensatz ist, nur der Anteil (1-p_{B, in echt}) ist?
While these passive measurement traces provided information on the frequency of new tunnel arrivals and the duration they remain active, no reliable information on the number of required supported concurrent tunnels for a given arrival rate could be deduced. 
This experiment evaluates arbitrary values for the \gls{GGSN} tunnel capacity and determines the resulting blocking probability such that a reasonable value can be found, given desired limits on the blocking probability. This is a typical task in a dimensioning process.

\begin{figure}[htb]
	\centering
	\includegraphics[width=0.9\textwidth]{images/R-monolithic-blocking.pdf}
	\caption{Impact of the number of supported parallel tunnels on the blocking probability for the traditional \acrshort{GGSN} model.}
\label{c4:fig:traditional_blocking}
\end{figure}

Figure~\ref{c4:fig:traditional_blocking} studies this impact of the maximum supported number of concurrent tunnels $n$ on the blocking probability $p_B$, where $n$ is incrementally increased in steps of \numprint{100} tunnels from \numprint{0} to \numprint{5500}. As expected, the blocking probability decreases with the number of supported tunnels. An almost linear correlation can be observed in the larger part of the graph with a small convergence phase shortly before reaching $p_B=0$. For the normalized inter-arrival rate no blocking is occurring if a capacity of \numprint{5000} concurrent tunnels is allocated to the \gls{GGSN}.

\begin{figure}[htb]
	\centering
	\includegraphics[width=0.9\textwidth]{images/R-monolithic-tunnelusage.pdf}
	\caption{Mean number of tunnels concurrently served by the \acrshort{GGSN} for incrementally increasing capacity.}
\label{c4:fig:traditional_tunnelusage}
\end{figure}

A similar picture is also evident in the number of tunnels served by this \gls{GGSN} in the same scenario as shown in Figure~\ref{c4:fig:traditional_tunnelusage}. For the first half of the experiments the \gls{GGSN} is loaded to its limit. Only when the capacity reaches \numprint{4600} can the normalized arrival rate be fully served, which surmounts to about \numprint{3820} tunnels on average in the system. Both results are stable across all simulation runs as the confidence intervals display. For the purpose of network dimensioning the results can be easily scaled up from the normalized arrival rates to the actual ones in the network in question.


%%
\subsubsection{Virtualization Impact and Gain}

A similar experiment can be set up for the virtual \gls{GGSN} model. Learning from the monolithic model, these follow-up simulations can be tuned to the same total tunnel capacity in advance. The only difference is that the tunnel capacity is now spread out evenly between the virtual \gls{GGSN} instances. The experiment tests different amounts for the total number of virtual instances, ranging from \numprint{1}, which represents the monolithic architecture, up to \numprint{100} instances in steps of \numprint{10}.

\begin{figure}[htb]
	\centering
	\includegraphics[width=0.9\textwidth]{images/R-virtualized-blocking.pdf}
	\caption{Comparison of the mean blocking probability of various virtual instance configurations. The horizontal axis depicts the aggregate capacity of all instances in the experiment.}
\label{c4:fig:virtualized_blocking}
\end{figure}

\begin{figure}[htb]
	\centering
	\includegraphics[width=0.9\textwidth]{images/R-virtualized-tunnelusage.pdf}
	\caption{Comparison of the mean tunnel capacity usage of the individual virtual instance configurations.}
\label{c4:fig:virtualized_tunnelusage}
\end{figure}

Figures~\ref{c4:fig:virtualized_blocking} and \ref{c4:fig:virtualized_tunnelusage} demonstrate the results in terms of $p_B$ and concurrent tunnels served overlaid onto the base monolithic scenario's results. No large difference in the results can be seen and the virtualized \gls{GGSN} model behaves no worse than a single large node model.

\begin{figure}[htb]
	\centering
	\includegraphics[width=0.9\textwidth]{images/R-virtualized-mean-instanceusage.pdf}
	\caption{Mean instance usage of various virtualization configurations. A higher number of total instances results in a finer granularity of scaling and energy efficiency as more instances can be kept shut down.}
% Die unterschiedlichen x- und y-Maßstäbe verursachen mir immer noch Interpretationskopfschmerzen...
\label{c4:fig:res-instance-usage-mean}
\end{figure}

But the possible effects of an increased number of instances need to be investigated further. One goal in virtualization is the increase of energy efficiency. This can be achieved by having turned on just as many instances as needed and not more, thus scaling the system to its current load. 

Therefore, Figure~\ref{c4:fig:res-instance-usage-mean} takes a look at scenarios with nine different instance pools and varying tunnel capacities for each instance. Each setup is compared by the mean number of active instances during the one-week course. The bigger the capacity of each instance becomes, the smaller the number of instances required active. An actual \gls{GGSN}, even a virtualized one, would need to be dimensioned in such a way to keep the total overhead low. It was already determined that with the assumed normalized arrival rate, a capacity of \numprint{5000} tunnels is sufficient in order to achieve a blocking probability close to zero. Keeping the setup at this minimum capacity and taking a look at the results in the figure, a good portion of the instances, usually around \SI{20}{\percent}, can still be kept turned off.
% Irgendwie find ich das nicht leicht, da durchzusteigen. Scheinbar geht es so: "Ich habe sagenwireinmal 10 Instanzen, und brauche 5 kTunnel. Nominell bedingt das 500 Tunnel pro Instanz. Der Mittelwert der aktiven Instanzen für diese Pro-Instanz-Kapazität ist aber nur ... (ablesen) ... 8 und ein bissl was, also ,,spare'' ich mir möglicherweise Strom etc. für knapp 2 der Instanzen." Ist das sinnvoll als Beispiel anzuführen, oder bin nur ich so langsam? :-)
% Sehe gerade weiter unten, dass da 30 bzw. 50 instances verwendet werden, was also auch für ein Beispiel sinnvoller wäre als meine 10, zumal man dann vergleichen könnte.

\begin{figure}[htbp]
	\centering
	\includegraphics[width=0.9\textwidth]{images/R-virtualized-instanceuse.pdf}
	\caption{Impact of the maximum number of tunnels and number of instances on the number of active instances in the virtual \acrshort{GGSN} model.}
\label{c4:fig:virtualized_instanceuse}
\end{figure}

To get into more detail, Figure~\ref{c4:fig:virtualized_instanceuse} displays the distribution of the portion of time a specific number of instances was active. Depicted are four configurations that differ in their total number of instances and their tunnel capacity. The setup with \numprint{30} instances with \numprint{100} capacity was clearly overwhelmed with the arrival rate and all \numprint{30} instances were active over \SI{70}{\percent} of the time. Only when the capacity was increased to \numprint{150} tunnels the virtualization benefits come into effect and more instances are able to sleep. Similar observations can be made in the \numprint{50} instance case.  Here, the \numprint{100} tunnel scenario is already equipped to handle the tunnel arrival rate and can scale back its active instances quite well, below \numprint{40} instances half the time. The final configuration with a \numprint{150} tunnel capacity is clearly overdimensioned here with no more than \numprint{33} of the \numprint{50} instances ever being active.
% Die Auswertung taugt mir!

\begin{figure}[htbp]
	\centering
	\includegraphics[width=0.9\textwidth]{images/R-virtualized-instanceuse-barplot.pdf}
	\caption{Resource usage from a select maximum instances and tunnel capacity combination, displaying the capability to scale up and out.}
\label{c4:fig:res-usage-barplot}
\end{figure}


Looking at these scenarios and additionally Figure~\ref{c4:fig:res-usage-barplot} from a network dimensioning perspective, two distinct pathways to scale in the virtualized \gls{GGSN} model are revealed. To reach the desired tunnel capacity either the number of instances or each instance's tunnel capacity can be increased. The latter represents the classical \textit{scaling up}. But virtualization also opens up the new path of \textit{scaling out} by increasing the number of instances. Through this, scaling can become easier and cheaper as existing machines need not be replaced any more.


%%
\subsubsection{Virtual Instance Life Cycle Management Impact}

A final aspect to be investigated in the experiments 
% "simulations", no?
is the potential increase of the blocking probability in virtualized scenarios when compared to the monolithic base. In theory, virtualization can incur additional overhead which would represent itself as an increase in $p_B$. In the given model the overhead can stem from the hypervisor and its scheduling  and lifecycle management strategies in conjunction with the instances' boot delay. 

The somewhat simplistic hypervisor strategy in this simulation was already discussed above 
% Genau! Wieso steht die eigentlich nicht hier?
and should give an upper limit on the impact on the blocking probability. This strategy is fixed, but the instance boot duration gets changed to analyze the impact. Values between \SI{20}{\second} and \SI{5}{\minute} are considered and should reasonably represent real-life systems of a wide variety.

\begin{figure}[htb]
	\centering
	\includegraphics[width=0.9\textwidth]{images/R-virtualized-startstop-blocking-barchart.pdf}
	\caption{Influence of the boot and shutdown time on the blocking probability.}
\label{c4:fig:blockprob-startstop-barchart}
\end{figure}

Figure~\ref{c4:fig:blockprob-startstop-barchart} compares a number of instance and tunnel capacity scenarios on basis of their instance boot duration. In most scenarios there is almost no increase in the tunnel blocking probability. Only in cases with very many but small instances, where a lot of instance churn will occur, an increase can be noticed at higher boot durations.

\begin{figure}[htbp]
	\centering
	\includegraphics[width=0.9\textwidth]{images/compare-maxinstances-block.pdf}
	\caption{Influence of start up and shut down time on blocking probability with regard to different numbers of instances.}
\label{c4:fig:compare_maxinstances_block}
\end{figure}

Figure~\ref{c4:fig:compare_maxinstances_block} investigates this increase in more detail and shows the blocking probability of one select scenario. Here, the system supports \numprint{5000} tunnels in total with differing individual instance capacity of \numprint{50}, \numprint{150}, and \numprint{500}. In each case the start and stop down duration is changed between \SI{1}{\minute} and \SI{5}{\minute}. The increase in blocking probability in relation to both the instance capacity as well as the start duration can be easily observed.

This can be partially attributed to the assumed hypervisor and its simplistic scheduling  and lifecycle management strategies in the simulation. If a low capacity instance with a long start time is activated, there is a high probability that the system will quickly expend its capacity again.
A potential conclusion is that choosing larger instance capacities decreases the blocking probability at the cost of energy efficiency (because less instances can stay turned off).

\begin{figure}[htbp]
	\centering
	\includegraphics[width=0.9\textwidth]{images/R-virtualized-startstop-tunnelusage-blocking-comparison.pdf}
	\caption{Trade-off between blocking probability and mean resource utilization with regard to maximum number of instances, instance tunnel capacity, and start and stop time.}
\label{c4:fig:compare_util_block}
\end{figure}

Finally, Figure~\ref{c4:fig:compare_util_block} shows two scenarios with \numprint{40} and \numprint{100} virtual \gls{GGSN} instances respectively, ranging from \numprint{1000} up to \numprint{5000} served tunnels. For each scenario, the combined impact of different individual instance tunnel capacities as well as start up and shutdown time on blocking probability and mean resource utilization is studied. The first observation is that by increasing the number of instances, i.e., scaling out, the blocking probability can be decreased, while maintaining a relatively low mean resource utilization. 

In addition to the previous effects, it can be noticed that a higher start up and shut down time causes a slight increase in blocking probability for instances with low tunnel capacity. Therefore, if smaller instances are to be used, for example due to price considerations, both the start up and shut down duration should be kept at a minimum. This could for example be achieved by using purely virtual instances in combination with fast storage.


%%
\subsubsection{Significance and Effect Sizes}

In order to analyze the influence of the different model parameters on the resulting metrics a one-way \gls{ANOVA} is performed. The effect size measures calculated here are the F-test, $\eta^2$, as well as $\omega^2$~\cite{stats,field2012discovering}. All are applied pairwise to each independent and derived variable combination. The results are depicted in Table~\ref{c4:tab:anova}. The two derived values and simulation metrics are the blocking probability and the mean tunnel usage.

\begin{table}[htbp]
\caption{Effect sizes of the simulation parameters based on a one-way \acrshort{ANOVA}.}
\label{c4:tab:anova}
	\centering
	\begin{tabu}{X[2.5,l]X[1.0,r]X[0.6,r]X[0.6,r]X[0.6r]}
	\toprule
	& $\mathbf{F-ratio}$ & $\mathbf{p-value}$ & $\mathbf{\eta^2}$ & $\mathbf{{\omega}^2}$\\ 
	\midrule
	\multicolumn{2}{c}{\textbf{Blocking probability}} & & & \\ 
	Individual instance tunnel capacity & $104$ & $<0.001$ & $0.468$ & $0.463$\\ %F(12, 1417)=
	Number of instances & $9.29$ & $<0.001$ & $0.056$ & $0.050$\\ %F(9,1420)=
	Start/stop duration & $0.21$ & $0.931$ & $<0.001$ & $0.002$\\ %F(4,1425)=
	Total tunnel capacity & $317257$ & $<0.001$ & $0.999$ & $0.999$ \\ %F(12,1417=
	\midrule
	\multicolumn{2}{c}{\textbf{Mean number of tunnels}}& & & \\ 
	Individual instance tunnel capacity & $105.7$ & $<0.001$ & $0.472$ & $0.467$\\ %F(12,1417)=
	Number of instances & $9.39$ & $<0.001$ & $0.056$ & $0.050$\\ %F(9,1420)=
	Start/stop duration & $0.25$ & $0.912$ & $<0.001$ & $0.002$\\ %F(4,1425)=
	Total tunnel capacity & $365753$ & $<0.001$ & $0.999$ & $0.999$ \\ %F(12,1417)=
	\bottomrule
	\end{tabu}
\end{table}

Both of these metrics yield very similar results as they are also --- by design --- strongly related to each other. The F-ratio computed by the F-test and the corresponding significance level $p$ indicate a large influence of the individual instance capacity on the metrics with a minor influence of the number of instances and no measurable impact of the start/stop duration. This is also confirmed by both $\eta^2$ and $\omega^2$. Interestingly, only the compound variable, which describes the total tunnel capacity, i.e., the product of the individual instance tunnel capacity and the number of instances, is an almost perfect match in its variance to the derived metrics.




%https://www.msu.edu/~levinet/eta%20squared%20hcr.pdf
% https://en.wikipedia.org/wiki/F-test
% F(2,1275): degrees of freedom (DF) for boh variables
% F-ratio: ratio of the model to its error; Note F(x, y) denotes an F-distribution with x degrees of freedom in the numerator and y degrees of freedom in the denominator.
% eta-squared, omega-squared, f-squared and effect sizes: https://en.wikiversity.org/wiki/Effect_size , https://en.wikipedia.org/wiki/Effect_size
% see also: https://en.wikipedia.org/wiki/Manipulation_checks

% Cohen's f is an effect size measure.  It is handy for power analysis as Cohen describes in "Statistical Power Analysis for the Behavioral Sciences."
% It is a "pure number to index the degree of departure from no effect."  
%The computation/use of f depends on the type of analysis used; fixed effects are an assumption (p. 273).  Cohen indeed gives f as f = (sqrt(eta^2 / (1 - eta^2)) for one-way fixed factor designs.  Since "there is no need to adjust one's conception of f for a set of k means when one moves from the one-way ANOVA to the case where additional bases of partitioning of the data exists," there doesn't seem to be any computational difference across the ANOVA designs. 

% https://en.wikiversity.org/wiki/Eta-squared

% https://stats.stackexchange.com/questions/15958/how-to-interpret-and-report-eta-squared-partial-eta-squared-in-statistically




% \begin{table}[htb]
%   \caption{Effect sizes of the simulation parameters based on one-way \acrshort{ANOVA}.}
%   \centering
%   \label{c4:tab:manipulation2color}
%   \begin{tabu}{X[1.6,l]X[r]X[r]X[r]X[1.1r]X[1.1,r]}
%   \toprule
%   & $\mathbf{F(2,1275)}$ & $\mathbf{\eta^2_p}$ & $\mathbf{p}$ & \textbf{Cohen's} $\mathbf{f^2}$ & \textbf{Cohen's} $\mathbf{\hat{\omega}^2}$\\ 
%   \midrule
%   \multicolumn{2}{l}{\textit{blocking probability}} & & & &\\ 
%   maxTunnels &  $15601.534$ & $\color{red}0.99$ & $<0.001$ & $\color{red} 26.739$ & $0.964$\\ 
%   maxInstances &  $10218.173$ & $\color{red} 0.986$ & $<0.001$ & $\color{red} 1.068$ & $0.516$\\ 
%   startstopDuration & $0.868$ & $\color{black} 0.003$ & $0.482$ & $\color{black} 0.000$ & $0.000$\\
%   \midrule
%   \multicolumn{2}{l}{\textit{mean number of tunnels}}& & & &\\ 
%   maxTunnels & $20448.347$ & $\color{red} 0.994$ & $<0.001$ & $\color{red} 27.712$ & $0.965$\\ 
%   maxInstances & $13348.251$ & $\color{red} 0.989$ & $<0.001$ & $\color{red} 1.064$ & $0.515$\\ 
%   startstopDuration & $2.872$ & $\color{black} 0.009$ & $0.022$ & $\color{black} 0.000$ & $0.000$\\
%   \bottomrule
%   \end{tabu}
% \end{table}



% \begin{figure}[htb]
%   \centering
%   \includegraphics[width=1.0\textwidth]{images/blocking-comparison.pdf}
%   \caption{Relative increase of blocking probability on the number of servers compared to the traditional \gls{GGSN}; with the $4500$ maximum tunnels per server being on a single server, $150$ on $30$, and $75$ on $60$ servers.}
% \label{c4:fig:blocking-comparison}
% \end{figure}

%%%%%%%%%%%%%%%%%%%%%%%%%%%%%%%%%%%%%%%%%%%%%%%%%%%%%%%%%%%%%%%%%%%%%%%%%%%%%%%%
\section{Core Network Evaluation Summary}
\label{c4:sec:conclusion}


The investigation of a week-long measurement trace recorded in an operational core network revealed some interesting signaling characteristics especially regarding the interdependency between user plane and control plane. Additionally, \gls{gtp} tunnel properties were determined to be a worthwhile measure for control plane load at the \gls{GGSN}, one of the central nodes in a \gls{3G} core network.

The investigation showed that the control plane is easily influenced by several device-based --- as far as they can be distinguished in a core network trace --- and time-of-day related features. The overall diurnal tunnel signaling load closely resembles the progression of the user plane. Most of the control plane's procedures are still triggered, either directly or indirectly, by user devices, of which the offered load is much smaller during night time. The trace evaluation also confirms 
% Wer hat denn diese Dominanz vorgeschlagen? "shows ... currently dominating"?
the dominating influence of smartphones compared to other devices types, even when looking at the control plane.

But this also means that sheer traffic volume is not a good measure to determine load, as the per-device traffic volume of a smartphone is rather low when compared to devices like pure \gls{3G} modems attached to a notebook. In this aspect, the findings also support the stories of signaling storms in mobile networks caused by applications regularly causing small amounts of network traffic. Each application interaction results in disproportionate amounts of signaling load being generated. Even worse, measures taken to improve the radio interface control plane such as Fast Dormancy could possibly have adverse effects to core signaling as they might increase the tunnel churn.

But the load investigation should not stop here. The presented approaches were just the ones that could be conducted with the available data. If one were to have access to a mobile network monitoring system or more detailed data records from such a system, it would open up many more angles in the investigation. For example, recording every individual signaling message with all \glspl{IE} would give hard numbers on the direct signaling overhead, as could measurement probes located inside the network nodes report on the CPU and memory load in order to determine the control plane's processing overhead. A closer investigation of control plane load in relation to mobility behavior should also prove very interesting, as this is one of the central motifs in every mobile network.

Learning from this historical data, queuing theoretic models were created that can describe the control plane load in such networks. These models can be easily used in network dimensioning and planning processes by means of, e.g., stationary analyses. The novel baseline control plane load model presented here is a $M(t)/G(t)/c/0$ non-stationary Erlang loss model. When used in conjunction with parameters derived from the measurement traces it can easily be used for network dimensioning. To improve scaling in the future a further \gls{GGSN} load model with features used in virtualization was also proposed.

Due to general solvability issues of non-stationary Erlang models the model is evaluated and validated using a queuing simulation in terms of blocking and tunnel state probability as well as overall resource utilization. The virtual model provided the added benefit of being more flexible in its scaling properties and energy efficiency. This might even lead to new \gls{GGSN}-as-a-Service business models, removing the need to provide and operate large amounts of infrastructure for rare cases of peak load. 

All of these properties serve to show the complexity of current mobile network systems even without running media streaming on top of it. Streaming in itself, while not being a real-time communication protocol, is relatively sensitive to timings and influences from lower network layers, which can make streaming over mobile networks rather problematic. The next chapters investigate these issues and methods to evaluate them more closely.

%In the future we would like to deepen our modeling efforts to provide more dimensioning options for a core network. Also, we want to further investigate the correlation of user traffic and signaling and take a look at the implications specific traffic types bring for the core network. 

%All these investigations and modeling efforts combined could lead to a more informed approach of network planning: Being more aware of the control plane provides the necessary tools to identify probable causes for control plane activity. 

%We would also like to expand our evaluations, as there are several angles not investigated so far that could prove worthwhile. This includes an examination of the exact number and size of signaling messages flowing through the core, a more detailed picture of the processing load these messages induce at the \gls{GGSN}, and an evolved model. Furthermore, a differential analysis of our data compared to a newer dataset (potentially including \gls{LTE} access) could really prove worthwhile.

%We also look forward to searching for multiple active tunnels per device. As discussed previously, the \textit{Secondary PDP Context Activation Procedure} enables devices to establish up to ten additional tunnels attributed with a different, higher QoS level, if the network supports this. The additional load of managing and holding multiple tunnels plus the displacement of other, ``lower-quality'' traffic could prove to be an interesting investigation. Initial observations indicate that this feature is rarely used today by very few types of devices, but it will be of increased interest in the face of ongoing LTE/EPS deployments, whose specifications expand upon this secondary tunnel concept.