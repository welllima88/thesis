%!TEX root = ../../dissertation.tex
%%%%%%%%%%%%%%%%%%%%%%%%%%%%%%%%%%%%%%%%%%%%%%%%%%%%%%%%%%%%%%%%%%%%%%%%%%%%%%%
\section{Load Model Queuing Simulation} 
\label{c4:sec:simulation}

As discussed, the solvability of a non-stationary Erlang loss system is very limited. To better tackle this, a simulative approach can be taken. Depending on the level of detail, different types of simulations are available.

Here, a queuing simulation is used to ascertain the blocking probability and tunnel serving slot utilization from the model using the fitted distributions from the trace.


%%%%%%%%%%%%%%%%%%%%%%%%%%%%%%%%%%%%%%%%%%%%%%%%%%%%%%%%%%%%%%%%%%%%%%%%%%%%%%%
\subsection{Queuing Simulation Implementation}

The queuing simulation is implemented on the basis of a \gls{DES}. Instead of reproducing continuous time, this simulation is a series of discrete events. Time is advanced only at these events. 

A queuing model can be easily represented in a \gls{DES}.  Each tunnel request arrival is modeled as a discrete event. When such an event occurs, three processes are executed. The first process draws a random number from a \gls{PRNG} mapped to \gls{IAT} exponential distribution to schedule the next arrival event. Secondly, the serving units are checked for any free units. If one is found, it will now be occupied. Else, this arrival will be marked as rejected and the third action skipped. This third process now determines the length of the tunnel using another \gls{PRNG} adjusted to the serving time distribution to schedule the event in which the tunnel exits the system.

This model was implemented on the basis of version 3.0 of the \textit{SimPy}\footnote{\url{https://simpy.readthedocs.org/}} package, which is a Python \gls{DES} framework that provides the basic event and scheduling infrastructure. On top of this a base \gls{GGSN} class was constructed, managing the arrival of tunnel events and the scheduling of the service ending events. Specific classes for the traditional (i.e., monolithic) and virtualized (called ``multiserver'' in the code) nodes respectively exist.\footnote{The implementation is also publicly available at \url{https://github.com/fmetzger/ggsn-simulation/} as a reference.}


%%%%%%%%%%%%%%%%%%%%%%%%%%%%%%%%%%%%%%%%%%%%%%%%%%%%%%%%%%%%%%%%%%%%%%%%%%%%%%%
\subsection{Description and Design of the Individual Experiments}

To match the measurement data the simulation time is set to be \SI{7}{\day} in all simulation scenarios. The initial \SI{60}{\minute} of each experiment are considered to be the transient phase and are afterwards excluded from the results. Ten replications of each scenario were performed. All depicted error bars show the \SI{95}{\percent} confidence intervals across the experiments.

The first experiment was conducted to investigate the normalized 
% Was bedeutet das nochmal?
baseline load a monolithic \gls{GGSN} experiences using the presented model. Using this, an upper limit to the number of concurrent tunnels and its correlation to the blocking probability and tunnel rejection rate can be established. The effects of scaling up, improving the hardware capabilities of the single node, can thus be investigated afterwards.

Based on these results, the virtualization and scaling out effects in the virtualized, \gls{GGSN} model are examined. In order to study the feasibility of this approach the performance indicators of the virtual \gls{GGSN} are compared to the baseline established in the first experiment. To this end, the virtual \gls{GGSN} is simulated in several configurations, which vary the number of instances and supported concurrent tunnels per instance.

In a final experiment the startup and shutdown duration of virtual instances and the life cycle management of these instances are additionally taken into account. Although the boot duration of modern \glspl{os} and \glspl{VM}, especially on current hardware with flash storage, 
% \gls{SSD} anyone?
is significantly lower than what past systems could achieve, there is still a delay. This could cause further blocking if the load balancer does not account for this. On the other hand, the more generously in advance the balancer starts instances the smaller the virtualization efficiency gain, especially the energy savings, will be become. For this reason, the number of active instance is a relevant performance metric in the virtual \gls{GGSN} model.

The experiment varies the boot delay and implements a very simple load balancer rule as baseline. The rule keeps at least one empty instance running in reserve at all times, and deactivates instances when two running instances are completely unused. As this is very generous, virtualization blocking should only occur in cases of instances limited to handling a low maximum number of tunnels, or very rapid arrivals. Realistic provisioning rules can improve on this quite easily. But even this simplistic approach already serves to demonstrate potential benefits.


%%
\subsubsection{GGSN Load, Capacity, and Scaling}

First, with the help of the \glspl{IAT} and duration of tunnels calculated in the dataset evaluation, the monolithic \gls{GGSN} model is studied. 
% Ist die Annahme, dass das echte Ding nicht blockt, oder muss man davon ausgehen, dass was in diesem Datensatz ist, nur der Anteil (1-p_{B, in echt}) ist?
While these passive measurement traces provided information on the frequency of new tunnel arrivals and the duration they remain active, no reliable information on the number of required supported concurrent tunnels for a given arrival rate could be deduced. 
This experiment evaluates arbitrary values for the \gls{GGSN} tunnel capacity and determines the resulting blocking probability such that a reasonable value can be found, given desired limits on the blocking probability. This is a typical task in a dimensioning process.

\begin{figure}[htb]
	\centering
	\includegraphics[width=0.9\textwidth]{images/R-monolithic-blocking.pdf}
	\caption{Impact of the number of supported parallel tunnels on the blocking probability for the traditional \acrshort{GGSN} model.}
\label{c4:fig:traditional_blocking}
\end{figure}

Figure~\ref{c4:fig:traditional_blocking} studies this impact of the maximum supported number of concurrent tunnels $n$ on the blocking probability $p_B$, where $n$ is incrementally increased in steps of \numprint{100} tunnels from \numprint{0} to \numprint{5500}. As expected, the blocking probability decreases with the number of supported tunnels. An almost linear correlation can be observed in the larger part of the graph with a small convergence phase shortly before reaching $p_B=0$. For the normalized inter-arrival rate no blocking is occurring if a capacity of \numprint{5000} concurrent tunnels is allocated to the \gls{GGSN}.

\begin{figure}[htb]
	\centering
	\includegraphics[width=0.9\textwidth]{images/R-monolithic-tunnelusage.pdf}
	\caption{Mean number of tunnels concurrently served by the \acrshort{GGSN} for incrementally increasing capacity.}
\label{c4:fig:traditional_tunnelusage}
\end{figure}

A similar picture is also evident in the number of tunnels served by this \gls{GGSN} in the same scenario as shown in Figure~\ref{c4:fig:traditional_tunnelusage}. For the first half of the experiments the \gls{GGSN} is loaded to its limit. Only when the capacity reaches \numprint{4600} can the normalized arrival rate be fully served, which surmounts to about \numprint{3820} tunnels on average in the system. Both results are stable across all simulation runs as the confidence intervals display. For the purpose of network dimensioning the results can be easily scaled up from the normalized arrival rates to the actual ones in the network in question.


%%
\subsubsection{Virtualization Impact and Gain}

A similar experiment can be set up for the virtual \gls{GGSN} model. Learning from the monolithic model, these follow-up simulations can be tuned to the same total tunnel capacity in advance. The only difference is that the tunnel capacity is now spread out evenly between the virtual \gls{GGSN} instances. The experiment tests different amounts for the total number of virtual instances, ranging from \numprint{1}, which represents the monolithic architecture, up to \numprint{100} instances in steps of \numprint{10}.

\begin{figure}[htb]
	\centering
	\includegraphics[width=0.9\textwidth]{images/R-virtualized-blocking.pdf}
	\caption{Comparison of the mean blocking probability of various virtual instance configurations. The horizontal axis depicts the aggregate capacity of all instances in the experiment.}
\label{c4:fig:virtualized_blocking}
\end{figure}

\begin{figure}[htb]
	\centering
	\includegraphics[width=0.9\textwidth]{images/R-virtualized-tunnelusage.pdf}
	\caption{Comparison of the mean tunnel capacity usage of the individual virtual instance configurations.}
\label{c4:fig:virtualized_tunnelusage}
\end{figure}

Figures~\ref{c4:fig:virtualized_blocking} and \ref{c4:fig:virtualized_tunnelusage} demonstrate the results in terms of $p_B$ and concurrent tunnels served overlaid onto the base monolithic scenario's results. No large difference in the results can be seen and the virtualized \gls{GGSN} model behaves no worse than a single large node model.

\begin{figure}[htb]
	\centering
	\includegraphics[width=0.9\textwidth]{images/R-virtualized-mean-instanceusage.pdf}
	\caption{Mean instance usage of various virtualization configurations. A higher number of total instances results in a finer granularity of scaling and energy efficiency as more instances can be kept shut down.}
% Die unterschiedlichen x- und y-Maßstäbe verursachen mir immer noch Interpretationskopfschmerzen...
\label{c4:fig:res-instance-usage-mean}
\end{figure}

But the possible effects of an increased number of instances need to be investigated further. One goal in virtualization is the increase of energy efficiency. This can be achieved by having turned on just as many instances as needed and not more, thus scaling the system to its current load. 

Therefore, Figure~\ref{c4:fig:res-instance-usage-mean} takes a look at scenarios with nine different instance pools and varying tunnel capacities for each instance. Each setup is compared by the mean number of active instances during the one-week course. The bigger the capacity of each instance becomes, the smaller the number of instances required active. An actual \gls{GGSN}, even a virtualized one, would need to be dimensioned in such a way to keep the total overhead low. It was already determined that with the assumed normalized arrival rate, a capacity of \numprint{5000} tunnels is sufficient in order to achieve a blocking probability close to zero. Keeping the setup at this minimum capacity and taking a look at the results in the figure, a good portion of the instances, usually around \SI{20}{\percent}, can still be kept turned off.
% Irgendwie find ich das nicht leicht, da durchzusteigen. Scheinbar geht es so: "Ich habe sagenwireinmal 10 Instanzen, und brauche 5 kTunnel. Nominell bedingt das 500 Tunnel pro Instanz. Der Mittelwert der aktiven Instanzen für diese Pro-Instanz-Kapazität ist aber nur ... (ablesen) ... 8 und ein bissl was, also ,,spare'' ich mir möglicherweise Strom etc. für knapp 2 der Instanzen." Ist das sinnvoll als Beispiel anzuführen, oder bin nur ich so langsam? :-)
% Sehe gerade weiter unten, dass da 30 bzw. 50 instances verwendet werden, was also auch für ein Beispiel sinnvoller wäre als meine 10, zumal man dann vergleichen könnte.

\begin{figure}[htbp]
	\centering
	\includegraphics[width=0.9\textwidth]{images/R-virtualized-instanceuse.pdf}
	\caption{Impact of the maximum number of tunnels and number of instances on the number of active instances in the virtual \acrshort{GGSN} model.}
\label{c4:fig:virtualized_instanceuse}
\end{figure}

To get into more detail, Figure~\ref{c4:fig:virtualized_instanceuse} displays the distribution of the portion of time a specific number of instances was active. Depicted are four configurations that differ in their total number of instances and their tunnel capacity. The setup with \numprint{30} instances with \numprint{100} capacity was clearly overwhelmed with the arrival rate and all \numprint{30} instances were active over \SI{70}{\percent} of the time. Only when the capacity was increased to \numprint{150} tunnels the virtualization benefits come into effect and more instances are able to sleep. Similar observations can be made in the \numprint{50} instance case.  Here, the \numprint{100} tunnel scenario is already equipped to handle the tunnel arrival rate and can scale back its active instances quite well, below \numprint{40} instances half the time. The final configuration with a \numprint{150} tunnel capacity is clearly overdimensioned here with no more than \numprint{33} of the \numprint{50} instances ever being active.
% Die Auswertung taugt mir!

\begin{figure}[htbp]
	\centering
	\includegraphics[width=0.9\textwidth]{images/R-virtualized-instanceuse-barplot.pdf}
	\caption{Resource usage from a select maximum instances and tunnel capacity combination, displaying the capability to scale up and out.}
\label{c4:fig:res-usage-barplot}
\end{figure}


Looking at these scenarios and additionally Figure~\ref{c4:fig:res-usage-barplot} from a network dimensioning perspective, two distinct pathways to scale in the virtualized \gls{GGSN} model are revealed. To reach the desired tunnel capacity either the number of instances or each instance's tunnel capacity can be increased. The latter represents the classical \textit{scaling up}. But virtualization also opens up the new path of \textit{scaling out} by increasing the number of instances. Through this, scaling can become easier and cheaper as existing machines need not be replaced any more.


%%
\subsubsection{Virtual Instance Life Cycle Management Impact}

A final aspect to be investigated in the experiments 
% "simulations", no?
is the potential increase of the blocking probability in virtualized scenarios when compared to the monolithic base. In theory, virtualization can incur additional overhead which would represent itself as an increase in $p_B$. In the given model the overhead can stem from the hypervisor and its scheduling  and lifecycle management strategies in conjunction with the instances' boot delay. 

The somewhat simplistic hypervisor strategy in this simulation was already discussed above 
% Genau! Wieso steht die eigentlich nicht hier?
and should give an upper limit on the impact on the blocking probability. This strategy is fixed, but the instance boot duration gets changed to analyze the impact. Values between \SI{20}{\second} and \SI{5}{\minute} are considered and should reasonably represent real-life systems of a wide variety.

\begin{figure}[htb]
	\centering
	\includegraphics[width=0.9\textwidth]{images/R-virtualized-startstop-blocking-barchart.pdf}
	\caption{Influence of the boot and shutdown time on the blocking probability.}
\label{c4:fig:blockprob-startstop-barchart}
\end{figure}

Figure~\ref{c4:fig:blockprob-startstop-barchart} compares a number of instance and tunnel capacity scenarios on basis of their instance boot duration. In most scenarios there is almost no increase in the tunnel blocking probability. Only in cases with very many but small instances, where a lot of instance churn will occur, an increase can be noticed at higher boot durations.

\begin{figure}[htbp]
	\centering
	\includegraphics[width=0.9\textwidth]{images/compare-maxinstances-block.pdf}
	\caption{Influence of start up and shut down time on blocking probability with regard to different numbers of instances.}
\label{c4:fig:compare_maxinstances_block}
\end{figure}

Figure~\ref{c4:fig:compare_maxinstances_block} investigates this increase in more detail and shows the blocking probability of one select scenario. Here, the system supports \numprint{5000} tunnels in total with differing individual instance capacity of \numprint{50}, \numprint{150}, and \numprint{500}. In each case the start and stop down duration is changed between \SI{1}{\minute} and \SI{5}{\minute}. The increase in blocking probability in relation to both the instance capacity as well as the start duration can be easily observed.

This can be partially attributed to the assumed hypervisor and its simplistic scheduling  and lifecycle management strategies in the simulation. If a low capacity instance with a long start time is activated, there is a high probability that the system will quickly expend its capacity again.
A potential conclusion is that choosing larger instance capacities decreases the blocking probability at the cost of energy efficiency (because less instances can stay turned off).

\begin{figure}[htbp]
	\centering
	\includegraphics[width=0.9\textwidth]{images/R-virtualized-startstop-tunnelusage-blocking-comparison.pdf}
	\caption{Trade-off between blocking probability and mean resource utilization with regard to maximum number of instances, instance tunnel capacity, and start and stop time.}
\label{c4:fig:compare_util_block}
\end{figure}

Finally, Figure~\ref{c4:fig:compare_util_block} shows two scenarios with \numprint{40} and \numprint{100} virtual \gls{GGSN} instances respectively, ranging from \numprint{1000} up to \numprint{5000} served tunnels. For each scenario, the combined impact of different individual instance tunnel capacities as well as start up and shutdown time on blocking probability and mean resource utilization is studied. The first observation is that by increasing the number of instances, i.e., scaling out, the blocking probability can be decreased, while maintaining a relatively low mean resource utilization. 

In addition to the previous effects, it can be noticed that a higher start up and shut down time causes a slight increase in blocking probability for instances with low tunnel capacity. Therefore, if smaller instances are to be used, for example due to price considerations, both the start up and shut down duration should be kept at a minimum. This could for example be achieved by using purely virtual instances in combination with fast storage.


%%
\subsubsection{Significance and Effect Sizes}

In order to analyze the influence of the different model parameters on the resulting metrics a one-way \gls{ANOVA} is performed. The effect size measures calculated here are the F-test, $\eta^2$, as well as $\omega^2$~\cite{stats,field2012discovering}. All are applied pairwise to each independent and derived variable combination. The results are depicted in Table~\ref{c4:tab:anova}. The two derived values and simulation metrics are the blocking probability and the mean tunnel usage.

\begin{table}[htbp]
\caption{Effect sizes of the simulation parameters based on a one-way \acrshort{ANOVA}.}
\label{c4:tab:anova}
	\centering
	\begin{tabu}{X[2.5,l]X[1.0,r]X[0.6,r]X[0.6,r]X[0.6r]}
	\toprule
	& $\mathbf{F-ratio}$ & $\mathbf{p-value}$ & $\mathbf{\eta^2}$ & $\mathbf{{\omega}^2}$\\ 
	\midrule
	\multicolumn{2}{c}{\textbf{Blocking probability}} & & & \\ 
	Individual instance tunnel capacity & $104$ & $<0.001$ & $0.468$ & $0.463$\\ %F(12, 1417)=
	Number of instances & $9.29$ & $<0.001$ & $0.056$ & $0.050$\\ %F(9,1420)=
	Start/stop duration & $0.21$ & $0.931$ & $<0.001$ & $0.002$\\ %F(4,1425)=
	Total tunnel capacity & $317257$ & $<0.001$ & $0.999$ & $0.999$ \\ %F(12,1417=
	\midrule
	\multicolumn{2}{c}{\textbf{Mean number of tunnels}}& & & \\ 
	Individual instance tunnel capacity & $105.7$ & $<0.001$ & $0.472$ & $0.467$\\ %F(12,1417)=
	Number of instances & $9.39$ & $<0.001$ & $0.056$ & $0.050$\\ %F(9,1420)=
	Start/stop duration & $0.25$ & $0.912$ & $<0.001$ & $0.002$\\ %F(4,1425)=
	Total tunnel capacity & $365753$ & $<0.001$ & $0.999$ & $0.999$ \\ %F(12,1417)=
	\bottomrule
	\end{tabu}
\end{table}

Both of these metrics yield very similar results as they are also --- by design --- strongly related to each other. The F-ratio computed by the F-test and the corresponding significance level $p$ indicate a large influence of the individual instance capacity on the metrics with a minor influence of the number of instances and no measurable impact of the start/stop duration. This is also confirmed by both $\eta^2$ and $\omega^2$. Interestingly, only the compound variable, which describes the total tunnel capacity, i.e., the product of the individual instance tunnel capacity and the number of instances, is an almost perfect match in its variance to the derived metrics.




%https://www.msu.edu/~levinet/eta%20squared%20hcr.pdf
% https://en.wikipedia.org/wiki/F-test
% F(2,1275): degrees of freedom (DF) for boh variables
% F-ratio: ratio of the model to its error; Note F(x, y) denotes an F-distribution with x degrees of freedom in the numerator and y degrees of freedom in the denominator.
% eta-squared, omega-squared, f-squared and effect sizes: https://en.wikiversity.org/wiki/Effect_size , https://en.wikipedia.org/wiki/Effect_size
% see also: https://en.wikipedia.org/wiki/Manipulation_checks

% Cohen's f is an effect size measure.  It is handy for power analysis as Cohen describes in "Statistical Power Analysis for the Behavioral Sciences."
% It is a "pure number to index the degree of departure from no effect."  
%The computation/use of f depends on the type of analysis used; fixed effects are an assumption (p. 273).  Cohen indeed gives f as f = (sqrt(eta^2 / (1 - eta^2)) for one-way fixed factor designs.  Since "there is no need to adjust one's conception of f for a set of k means when one moves from the one-way ANOVA to the case where additional bases of partitioning of the data exists," there doesn't seem to be any computational difference across the ANOVA designs. 

% https://en.wikiversity.org/wiki/Eta-squared

% https://stats.stackexchange.com/questions/15958/how-to-interpret-and-report-eta-squared-partial-eta-squared-in-statistically




% \begin{table}[htb]
%   \caption{Effect sizes of the simulation parameters based on one-way \acrshort{ANOVA}.}
%   \centering
%   \label{c4:tab:manipulation2color}
%   \begin{tabu}{X[1.6,l]X[r]X[r]X[r]X[1.1r]X[1.1,r]}
%   \toprule
%   & $\mathbf{F(2,1275)}$ & $\mathbf{\eta^2_p}$ & $\mathbf{p}$ & \textbf{Cohen's} $\mathbf{f^2}$ & \textbf{Cohen's} $\mathbf{\hat{\omega}^2}$\\ 
%   \midrule
%   \multicolumn{2}{l}{\textit{blocking probability}} & & & &\\ 
%   maxTunnels &  $15601.534$ & $\color{red}0.99$ & $<0.001$ & $\color{red} 26.739$ & $0.964$\\ 
%   maxInstances &  $10218.173$ & $\color{red} 0.986$ & $<0.001$ & $\color{red} 1.068$ & $0.516$\\ 
%   startstopDuration & $0.868$ & $\color{black} 0.003$ & $0.482$ & $\color{black} 0.000$ & $0.000$\\
%   \midrule
%   \multicolumn{2}{l}{\textit{mean number of tunnels}}& & & &\\ 
%   maxTunnels & $20448.347$ & $\color{red} 0.994$ & $<0.001$ & $\color{red} 27.712$ & $0.965$\\ 
%   maxInstances & $13348.251$ & $\color{red} 0.989$ & $<0.001$ & $\color{red} 1.064$ & $0.515$\\ 
%   startstopDuration & $2.872$ & $\color{black} 0.009$ & $0.022$ & $\color{black} 0.000$ & $0.000$\\
%   \bottomrule
%   \end{tabu}
% \end{table}



% \begin{figure}[htb]
%   \centering
%   \includegraphics[width=1.0\textwidth]{images/blocking-comparison.pdf}
%   \caption{Relative increase of blocking probability on the number of servers compared to the traditional \gls{GGSN}; with the $4500$ maximum tunnels per server being on a single server, $150$ on $30$, and $75$ on $60$ servers.}
% \label{c4:fig:blocking-comparison}
% \end{figure}