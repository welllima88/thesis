%!TEX root = ../../dissertation.tex
%%%%%%%%%%%%%%%%%%%%%%%%%%%%%%%%%%%%%%%%%%%%%%%%%%%%%%%%%%%%%%%%%%%%%%%%%%%%%%%%
\chapter{Mobile Core Network Architecture}
\label{chap:mobilenets}

The rise of reliable video streaming is not happening completely independent of other developments in computer networking. 
% Nein eh nicht, aber was spielt das hier für eine Rolle?
Internet access is shifting towards mobile smartphones and mobile networks, sometimes even completely replacing fixed dial-up access  with stationary mobile \gls{3G} and \gls{LTE} modems. And all the ``over-the-top'' media streaming services are now more and more being used within mobile networks, bringing a new set of traffic patterns as well as possible issues and influence factors along with them.

Today's cellular mobile networks are usually based on \gls{3GPP} specifications which have evolved from the circuit switched \gls{GSM} network into the fully packet switched \gls{LTE}, the latter still being in its unrolling phase. Yet being packet switched does not mean that mobile networks share a lot of similarity with a typical wireline \gls{IP} stack and network access infrastructure, for example a \gls{VDSL} or \gls{DOCSIS} dial-up connected to an \gls{IP} network. 

A \gls{3G} network (a term synonymous for the typical combined \gls{GSM} and \gls{UMTS} cellular network in use today) is very distinct from typical wired networks as it provides, amongst others, mobility and authentication in its lower network layers through the core specifications, rather than implementing respective optional on-top services as is typically the case in the Internet. To achieve this, large amounts of state have to be kept at each network node, and state is explicitly communicated among the network nodes in the \gls{CN}. Also, the lines separating layers and functions are blurry, making it very hard to grasp parts of the architecture without understanding the whole. Many specialized protocols are involved to communicate intents and states in the network. This causes processing overhead and signaling traffic on the network paths on top of the actually useful user traffic. 

It is a known fact that radio spectral resources are a scarce resource that need to be well managed. But it might not seem immediately obvious to think the same about control plane resources in the infrastructure of a mobile network. Yet, there have been accounts\footnote{``Docomo Counts Cost of Signaling Storm'',\url{http://www.lightreading.com/d/d-id/693779}}\footnote{``Android Signaling Storm Rises in Japan'', \url{http://www.lightreading.com/a/d-id/693138}}\footnote{``Angry Birds + Android + ads = network overload'', \url{http://www.itwire.com/business-it-news/47823}}~\cite{huawei2011storm} 
% Beistriche zwischen den hochgestellten Zahlen wären der Hit! --- http://ctan.org/pkg/footmisc
of situations where the core network has been flooded with signaling, despite only a negligible amount of actual user traffic being transported. This resulted in an event dubbed ``signaling storm'', causing an unintentional \gls{DDoS} attack, disrupting user-plane connectivity on its way. 
Similar patterns can also occur with some segmented transmission strategies in streaming.
% "as will later be shown", oder wird das hier nur als Kuriosum erwähnt?


The state of research on the architecture and especially the control plane behavior of \gls{3G} mobile cellular networks is lean in comparison to the huge volume of research conducted on other wired Internet structures. Most of the available research on \gls{3G} focuses on user-oriented metrics such as traffic statistics and mobility patterns, or only investigates properties of the radio interface. Little has been published about activity within the core network, and yet less about core signaling. 

Conducting \gls{IP} traffic research at the network's edge, independent of the access technology, is usually relatively easy. Writing simple tests and measurement scripts that record the user's traffic is all that is needed. But mobile phones do not allow one to peek inside its layer 1 and 2 implementation and interaction, where the control plane is implemented. Any information on this black box can only be indirectly inferred from layers above --- by forcing behavior known from specification --- or below --- through spectrum analysis using \gls{SDR} approaches. 

However, to directly investigate the core as well as the control plane, some kind of a measurement infrastructure inside the network is needed. With this, researchers could not just look into user traffic flowing through the network but also investigate the signaling-heavy mobile network control plane. Yet to gain this kind of access to the network  would require the cooperation of a mobile network operator, which they grant very reluctantly due to privacy and competitive concerns.

Another important aspect related to signaling is network dimensioning. Operators mostly dimension their networks in accordance to the expected user traffic. But in such a signaling-dependent architecture this might not be the most fitting approach, as the mentioned signaling storms seem to demonstrate. 

Both this and the next chapter attempt to remedy parts of the problem. The current chapter contains material required for a basic understanding of mobile networks and the evaluation methodology. Section~\ref{c4:sec:3gpparchitecture} discusses the required details of the mobile network architecture under investigation, followed by a survey of existing literature in Section~\ref{c4:sec:relwork}. Next, an attempt on defining control plane load is made in Section~\ref{c4:sec:loaddefinition}. This is used to find viable targets for a load evaluation. The evaluation methodology is then given in Section~\ref{c4:sec:methodology}.

Chapter~\ref{chap:mobilenetsmeasuring} then exclusively discusses the actual control plane modeling and evaluations conducted in an existing mobile network. Both chapters also use and extend on material previously published in \cite{metzger2012research}, 
% Der Quelle metzger2012research fehlen ein paar Eigenschaften
\cite{metzger2014jcnc}, and \cite{metzger2014lossmodel}.


%%%%%%%%%%%%%%%%%%%%%%%%%%%%%%%%%%%%%%%%%%%%%%%%%%%%%%%%%%%%%%%%%%%%%%%%%%%%%%%
%!TEX root = ../../dissertation.tex
%%%%%%%%%%%%%%%%%%%%%%%%%%%%%%%%%%%%%%%%%%%%%%%%%%%%%%%%%%%%%%%%%%%%%%%%%%%%%%%%
\section{Core Architecture Overview}
\label{c4:sec:3gpparchitecture}

Today's dominating commercial mobile network system, which combines \gls{GSM}, \gls{UMTS}, and now often also \gls{LTE}, is designed and specified by the \gls{3GPP}. This group is an umbrella organization for several standardization bodies, including the European \gls{ETSI} and their individual members --- in this case mostly telecommunication companies. Unlike the \gls{IETF}, the Internet's de-facto standards body, natural persons cannot participate in the \gls{3GPP} on their own but only through the organizational members.

Specifications are not released individually, but are instead grouped together into larger releases once every or every other year. \gls{GSM} was first specified in the \textit{Phase 1} release in 1992 with \gls{GPRS} added in \textit{Release 97} (1998). \gls{UMTS} followed with \textit{Release 99} (2000), but most \gls{3G} networks operate at least with \textit{Release 5} (2002), \textit{Release 6} (2004), or \textit{Release 7} (2007) as they introduced \gls{HSDPA}, \gls{HSUPA}, and \gls{HSPA+} likewise. 
% likewise --> respectively?
\gls{LTE} first found its way into the specifications in 2008 with \textit{Release 8}. \textit{Release 12} is scheduled to be published in March of 2015. This background section mostly describes the \gls{UMTS}-based \glspl{TS} 
% Plural!
with some comparisons being made to the older \gls{GPRS} and the latest \gls{LTE} specification versions where available.

\begin{figure}[htbp]
	\centering
	\includegraphics[width=1.0\textwidth]{images/3gpp-physical-arch.pdf}
	\caption{Overview of a combined \acrshort{CS}/\acrshort{PS} \acrshort{GSM}/\acrshort{UMTS}/\acrshort{LTE} architecture.}
\label{c4:fig:psdomain}
\end{figure}

Today's most commonly deployed version of the mobile network architecture is depicted in Figure~\ref{c4:fig:psdomain} and based on \gls{3GPP} \gls{TS} 23.002~\cite{3gpp.23.002}, with some minor nodes and network paths omitted\footnote{For a complete reference of all the acronyms and addressing schemes used in \gls{3GPP} specs please refer to \gls{TS} 21.905~\cite{3gpp.21.905} and \gls{TS} 23.003~\cite{3gpp.23.003}}. The displayed architecture combines all three access technologies as well as the \gls{CS} and the \gls{PS} domains of the core. 

Concerning the \gls{PS} domain, one has to further distinguish between links and nodes used solely for control plane tasks on one hand, and links and nodes that are in path of the actual user \gls{IP} traffic. For \gls{UMTS} and \gls{GPRS}\footnote{\gls{GPRS} provides \gls{PS} data services for \gls{GSM} radio access. The same core infrastructure is also used in the \gls{UMTS} \gls{PS} domain.} the \gls{SGSN} and the \gls{GGSN} are the core elements on the user traffic path. Elements other than these solve control plane tasks only.

One architectural detail to note is the strict separation between user plane and control plane tasks in the \gls{3GPP} architecture. Completely separate protocol stacks are used, and signaling is mostly conducted in an explicit and out-of-band manner. This is in contrast to the typical approach of the Internet's \gls{TCP}/\gls{IP} stack, especially its upper layers, where most state is implicitly inferred and only some signaled in-band. 

The following sections give a short description of nodes and their tasks as well as used protocols stacks and signaling procedures for both the user and the control plane. The description will be mostly focused on the \gls{UMTS} parts of the architecture which is also overviewed in \cite{3gpp.23.101}.


%%
\subsection{\texorpdfstring{\acrshort{3GPP}}{3GPP} Radio Network}

The architecture has three distinct, separate radio networks, one for each access technology: \gls{GSM}'s \gls{BSS} (or more complete: \gls{GERAN}), \gls{UTRAN}, and \gls{E-UTRAN} in \gls{LTE}.

Essential to the radio network is a base station, a radio transceiver providing the physical connection to the user's mobile device\footnote{Mobile devices are called \gls{MS} in \gls{GSM} networks and \gls{UE} in \gls{3G} and later.}. \gls{3G}'s base station is called \textit{Node B}. The used radio spectrum is divided into a number of channels, with various shared channels responsible for management and control plane signaling, and one or more dedicated channels for each active mobile device~\cite{3gpp.25.201,3gpp.25.301}. Layer 2 consists of several protocols managing and multiplexing transmissions on the link. These are \gls{MAC}~\cite{3gpp.25.321} and \gls{RLC}~\cite{3gpp.25.322} with an additional user plane broadcast service provided by \gls{BMC}~\cite{3gpp.25.324}. 

On layer 3, the actual radio control plane signaling protocol \gls{RRC} \cite{3gpp.25.331} resides, managing the device's state and the radio connection. Some of the signaling procedures 
% Hmm? Ein Phänomen der Specs, dass da nicht alles drinnensteht?
are detailed in \cite{3gpp.25.931}. Additionally, \gls{PDCP}~\cite{3gpp.25.323} provides the connection to the  usual Internet \gls{TCP}/\gls{IP} user plane stack atop. Thus, all user traffic is encapsulated into so called \textit{radio bearers}, which tunnels traffic from the mobile device directly into the core network.

Each base station acts as an independent radio cell. Mobile devices can be seamlessly handed over between cells without higher layers being able to notice this.\footnote{However, there are still many ways to detect a handover in the upper layers of a mobile device.} 
% "Noticing" ist die eine Sache (die eben nicht unmöglich ist); relevanter finde ich, dass sich die höheren Schichten nicht drum *kümmern* müssen.
The handover process is fully controlled and conducted by the network through a node that shares the old and new path to the device. For \gls{UMTS}, in most cases this can be handled by the \gls{RNC}, while the core \gls{SGSN} is responsible for handovers between larger regions.

In \gls{UMTS} multiple base stations are concentrated into one \gls{RNC}. Most of the functions of a \gls{RNC}, including \gls{MM}, are defined by the \gls{RANAP} control plane protocol defined in \cite{3gpp.25.413}. \gls{RANAP} is used at the Iu interface between the \gls{RNC} and the core network, i.e., the \gls{SGSN}. Today, all non-radio links of the network are usually \gls{IP}-based. But in the past all interfaces have also been explicitly defined for \gls{ATM} and exhibited some differences to their \gls{IP}-based counterparts. The connection of the decentralized parts of the radio network to either the core network itself or a \gls{RNC} is called \textit{backhaul}. This term usually subsumes the bulk transport of data over dedicated links to a central location. Often, optical fiber or microwave transmission links are used.


%%
\subsection{\texorpdfstring{\acrshort{3GPP}}{3GPP} Core Network}

The \gls{PS} domain of a mobile core network manages most of the aspects of the connected devices and acts as the bridge and gateway 
% Wieso bridge? Wieso nicht "gateway router"?
to the Internet. It consists of nodes that are directly in the path of the user plane traffic, plus additional nodes that only exist in the control plane. Each of the nodes can consist of any number of actual physical machines, but is considered as a monolithic unit in the architecture and related specifications.

\gls{GPRS} and \gls{UMTS} use the exact same \gls{PS} core network architecture. Only \gls{LTE} introduced a new core network concept, the \gls{EPC}. If one core has to simultaneously provide support for both \gls{UMTS} and \gls{LTE} access, core nodes for both architectures have to be present. The exception are some \gls{EPC} nodes that provide legacy interfaces to supplant their \gls{UMTS} predecessors.

The two central elements in the user plane's path are the \gls{SGSN} and the \gls{GGSN}~\cite{3gpp.22.060,3gpp.23.060}. The \gls{SGSN} is the endpoint of the \gls{RAB}, tunneling user traffic from the \gls{UE} to the core, and an endpoint for the \gls{gtp} based core tunnel, further transporting user plane traffic to the \gls{GGSN}. \Gls{gtp} will be described in detail in Section~\ref{c4:sec:gtp}. The \gls{GGSN}'s control plane tasks include mobility and connection management via \gls{RANAP} to the \gls{RNC}. The necessary information is cached and retrieved from the \gls{HLR} using \gls{MAP}~\cite{3gpp.29.002}. The \gls{HLR} or, respectively, the \gls{HSS} in an \gls{EPC}, acts as the central storage of all the operator's subscriber information.

The \gls{GGSN} represents the gateway to the public Internet, and therefore typically is the only node in the network that has a publicly routable \gls{IP} address. It filters incoming traffic into the corresponding \gls{gtp} tunnel and routes packets to the correct \gls{SGSN}. State about every active tunnel and device has to be kept locally and is initially retrieved from the \gls{SGSN}.

In \gls{EPC} user traffic in the core is handled by the \gls{SGW} and \gls{PGW}, having similar functions as their \gls{GPRS} counterparts \gls{SGSN} and \gls{GGSN}. Depending on the specific version of the implemented infrastructure these two nodes can also be combined into one, eliminating the S5 interface and the \gls{gtp} signaling between them. In the \gls{EPC} many of the control plane tasks previously maintained by the \gls{SGSN} have been offloaded to a new node, the \gls{MME}, which maintains its own logical connection to the radio network using the \gls{S1AP} signaling protocol~\cite{3gpp.36.413}. Instead of \gls{MAP} to retrieve user data from the central storage, Diameter~\cite{rfc6733} is now used to communicate with the \gls{HSS}. Of note is also a further addition to the \gls{EPS}: the \gls{PCRF} in conjunction with the \gls{PCEF} which is integrated into the \gls{PGW}. They act as a \gls{DPI} entity, inspecting all user plane traffic, and enable arbitrary filtering of traffic and traffic-based billing. Both entities are described in \cite{3gpp.23.203}.

\begin{figure}[htbp]
	\centering
	\includegraphics[width=0.9\textwidth]{images/umts-userpath-stack.pdf}
	\caption{Simplified control plane and user plane \acrshort{IP}-based protocol stacks on the user traffic path through the mobile network.}
% Der Pfeil "Uu" sollte meiner Meinung nach rechts am PHY-Kastl des UE weggehen.
% Weiters: warum übergibt das GGSN via Gi PHY ans Internet und nicht IP?
\label{c4:fig:protocolstacks}
\end{figure}

Figure~\ref{c4:fig:protocolstacks} overviews the complete protocol stack 
% "complete"? Vergleiche Caption: "Simplified control plane and user plane...."
on the path of the user traffic through the whole network from the \gls{UE} to an external network.


%%
\subsection{Core Network Concepts}

Most of the discussed upcoming research deals with the core network. Therefore, this next sections will explain in detail the concepts behind the \gls{3G} core network control plane and the \gls{gtp} protocol family.

As mentioned, there is a strict separation between control instances and instances that carry the actual user traffic. Looking at the control plane side of things, management is conducted in a completely stateful way. Nodes keep track of every \gls{UE} they are managing, and need to locally store any state information they might need for management purposes. Of particular interest is the state associated with the \textit{\gls{PDP} Context}. For each open data connection a device has, both the \gls{GGSN} and \gls{SGSN} must keep such a \gls{PDP} Context, which identifies both the connection and the device belonging to it.

Additionally, a number of state machines are maintained. Transitions between machine states trigger a signaling message to specific network neighbors. Any one of these signaling interactions belongs to one or more larger control plane procedures. Most of the procedures that happen inside the core network \gls{PS} domain or affect it are defined in \gls{TS}~24.008~\cite{3gpp.24.008} and \gls{TS}~23.060~\cite{3gpp.23.060}.

\begin{figure}[htbp]
	\centering
	\includegraphics[width=0.7\textwidth]{images/pdp-context-activation-procedure.pdf}
	\caption{\acrshort{PDP} Context activation procedure signaling interaction diagram for \acrshort{UMTS}, including involved signaling protocols.}
% MS --> UE
\label{c4:fig:pdpcontextactivationinteraction}
\end{figure}

A very simple example is demonstrated in Figure~\ref{c4:fig:pdpcontextactivationinteraction}. Initiated by the \gls{UE}'s session management state machine a data connection to the device is requested to be set up, triggering signaling with various protocols throughout the network. Additional secondary \gls{CAMEL} procedures, defined in \cite{3gpp.23.078}, are also triggered and conducted.

The general motif of control in \gls{3G} networks is rather different to that of the typical \gls{TCP}/\gls{IP} Internet stack. Whereas plain \gls{IP} stacks rely on the end-to-end principle and put most of the control inside the devices at the edge, in \gls{3G} control procedures are spread across the network. Coordinating this requires the discussed control plane and all of the explicit signaling interactions. This results in a rather high complexity but also gives the opportunity to investigate these mechanisms and implications the structures have on the performance.


%%
\subsection{Tunneling Concept: Bearers and \texorpdfstring{\acrshort{PDP}}{PDP} Contexts}

To enact the aforementioned user and control plane separation, a custom tunneling concept is used. Beginning at the \gls{UE}, the actual user \gls{IP} stack is never directly carried over the link's layer 1 and 2 protocols but always further encapsulated into so called \textit{bearer}. The specifications distinguish between the \textit{radio bearer} on the air interface, the \gls{RAB} denominates the path between the \gls{UE} and the \gls{CN}, where it is called \gls{CN} bearer.
% Der Satz sollte sich irgendwie zweiteilen (und damit leichter verständlich machen) lassen. Speziell fehlt mir, wovon der "radio bearer" denn jetzt genau unterschieden wird.

Technically, different protocols with tunneling capability are used on each link. \gls{PDCP} is used on the radio link, \gls{GTP-U} is employed between the \gls{RAN} and the \gls{CN} and in the core between \gls{SGSN} and \gls{GGSN}. Closely related to the core network bearer are a number of information records stored and maintained at the two core nodes, the aforementioned \gls{PDP} Context. For every active bearer a context is stored, containing identification and management information about it. Amongst other data this includes device identifiers, e.g., \gls{IMSI} and \gls{IMEI}, tunnel identifiers (\gls{TEID}), information about the intended public network through the \gls{APN} field, charging, and \gls{QoS} \cite[Section~13]{3gpp.23.060}.

Commonly, any device with an active data connection has at least one bearer and an associated \gls{PDP} Context, the \textit{default bearer}. In terms of \gls{QoS} this represents a best-effort tunnel carrying all user traffic that is not further differentiated. Additionally, the device can request a number of secondary tunnels, i.e., \textit{dedicated bearers}, with certain \gls{QoS} guarantees. Traffic matching a specified \gls{TFT} will then be carried over this dedicated bearer. However, this concept is scarcely used in \gls{3G} networks. All in all, one \gls{UE} can be associated with up to eleven bearers, consisting of one default bearer and additional dedicated bearers.


%%
\subsection{\texorpdfstring{\acrshort{gtp}}{GTP} and \texorpdfstring{\acrshort{gtp}}{GTP}-based Core Network Signaling}
\label{c4:sec:gtp}

A large part of core network communication is conducted by \gls{gtp}. In \gls{3G} networks version 1 of the protocol, defined in \gls{TS}~29.060~\cite{3gpp.29.060} and \gls{TS}~29.281~\cite{3gpp.29.281}\footnote{For a more concise description of the protocol, one should actually read the \gls{gtp} implementation in the community \gls{FOSS} project OpenGGSN at \url{https://github.com/osmobuntu/openggsn}.}, is used. For \gls{EPC} some changes were made to \gls{gtp} bringing it to version 2, which is specified in \gls{TS}~29.274~\cite{3gpp.29.274}. The latter will not be further discussed as all presented evaluations will have a \gls{3G} network as basis.

\gls{gtp} is mainly used on the Gn interface that connects the two main \gls{GPRS} nodes, the \gls{GGSN} and \gls{SGSN}. Its functionality is split up between a user plane and a control plane part, named respectively \gls{GTP-U} and \gls{GTP-C}. \gls{gtp} can best be described as an application layer signaling protocol and is intended to be transported by \gls{UDP}.

\begin{figure}[htb]
	\begin{tabu}{X[1]|X[1]|X[1]|X[1]|X[1]|X[1]|X[1]|X[1]|X[1]|}
	\multicolumn{1}{c}{} & \multicolumn{8}{c}{\textbf{Bits}} \\
	\textbf{Octets} & \textbf{8} & \textbf{7} & \textbf{6} & \textbf{5} & \textbf{4} & \textbf{3} & \textbf{2} & \textbf{1} \\ 
	\cline{2-9} \textbf{1} & \multicolumn{3}{c|}{Version}  & 1 & 0 & E & S & PN \\ 
	\cline{2-9} \textbf{2} & \multicolumn{8}{c|}{Message Type}  \\ 
	\cline{2-9} \textbf{3} & \multicolumn{8}{c|}{\multirow{2}{*}{Length}}  \\ 
				\textbf{4} & \multicolumn{8}{c|}{}  \\ 
	\cline{2-9} \textbf{5} & \multicolumn{8}{c|}{\multirow{4}{*}{Tunnel Endpoint Identifier}} \\ 
% Acro/gls für TEID verwenden?
				\textbf{6} & \multicolumn{8}{c|}{} \\ 
				\textbf{7} & \multicolumn{8}{c|}{} \\ 
				\textbf{8} & \multicolumn{8}{c|}{} \\ 
	\cline{2-9} \textbf{9} & \multicolumn{8}{c|}{\multirow{2}{*}{Sequence Number}} \\
				\textbf{10} & \multicolumn{8}{c|}{} \\
	\cline{2-9}	\textbf{11} & \multicolumn{8}{c|}{N-PDU} \\
	\cline{2-9} \textbf{12} & \multicolumn{8}{c|}{Next Extension Header Type} \\
	\cline{2-9}
	\end{tabu} 
	\caption{General \SI{12}{\byte} \acrshort{gtp} header format.}
\label{c4:fig:gtpheader}
\end{figure}

Conceptually, \gls{gtp} is structured through a base packet header, a number of extension headers and a message body consisting of a series of \gls{IE}. The base header is depicted in Figure~\ref{c4:fig:gtpheader} and has a total length of \SI{12}{\byte}. Essential to the header are the \gls{TEID} to identify the corresponding user plane tunnel and the \SI{8}{\bit} message type field. Each of the message types corresponds to a specific signaling interaction from the overarching control plane procedures. The procedures that \gls{gtp} concerns itself with on the Gn path belong either to path management, tunnel management, or mobility management. Messages also usually come in request and response pairs, thus involving message sent from the requester to the receiving node and back.

Each messages is defined as a specific set of \glspl{IE}, each of which is either mandatory, conditional on some external factor, or optional. These \glspl{IE} are of fixed or variable length depending on their type. They convey the actual state to be signaled, and always relate to a specific \gls{UE} and tunnel, for example the device's \gls{IMSI} or the configured \gls{APN}.

\begin{table}[htbp]
\caption{All \acrshort{IE} in a Create \acrshort{PDP} Context request, and respective sizes, for \acrshort{IPv4} network and user traffic only. The denoted sizes exclude the first message type byte.}
\label{c4:tab:createrequestelements}
	\begin{tabu} to 0.49\textwidth{X[2.5]X[1.2]X[0.7]}
		\toprule
		\textbf{\gls{IE}} & \textbf{Presence} & \textbf{Size}\\
		\midrule
		\gls{IMSI} & cond. & \SI{8}{\byte} \\ 
		\acrshort{RAI} & opt. & \SI{6}{\byte} \\
		Recovery & opt. & \SI{1}{\byte} \\
		Selection mode	& cond. & \SI{1}{\byte} \\
		\gls{TEID} Data I & mand. & \SI{4}{\byte} \\
		\gls{TEID} Control Plane & cond. & \SI{4}{\byte} \\
		\gls{NSAPI} & mand. & \SI{1}{\byte} \\
		Linked \gls{NSAPI} & cond. & \SI{1}{\byte} \\
		Charging Characteristics & cond. & \SI{2}{\byte} \\
		Trace Reference & opt. & \SI{2}{\byte} \\
		Trace Type & opt. & \SI{2}{\byte} \\
		End User Address & cond. & \SI{8}{\byte} \\
		\gls{APN} & cond. & max \SI{102}{\byte} \\ % APN format defined in 23.003 section 9
		\acrshort{PCO} & opt. & max \SI{255}{\byte} \\ % defined in 24.008 section 10.5.6.3
		\gls{SGSN} signaling address & mand.  & \SI{6}{\byte} \\ % defined in 23.003 section 5 (without address type and length fields)
		\gls{SGSN} user traffic address & mand. & \SI{6}{\byte} \\ % same as above
		\gls{MSISDN} & cond. & max \SI{17}{\byte} \\ % ITU-T E.164 msisdn format recommendation of max 15 chars
		\gls{QoS} Profile & mand. & max \SI{257}{\byte} \\
		\bottomrule
	\end{tabu}%
	\raisebox{0.95mm}{\begin{tabu} to 0.49\textwidth{X[2.5]X[1.2]X[0.7]}
		\toprule
		\textbf{\gls{IE}} & \textbf{Presence} & \textbf{Size} \\
		\midrule
		\gls{TFT} & cond. & max \SI{257}{\byte} \\ % defined in 24.008 section 10.5.6.12
		Trigger Id & opt. & var. \\ % no definition found
		\acrshort{OMC} Identity & opt. & var. \\ % maybe in MAP 29.002, however not further definition found
		Common Flags & opt. & \SI{3}{\byte} \\
		\gls{APN} Restriction & opt. & \SI{3}{\byte} \\
		\gls{RAT} & opt. & \SI{3}{\byte} \\
		User Location Information & opt. & \SI{10}{\byte} \\
		\gls{MS} Time Zone & opt. & \SI{4}{\byte} \\
		\gls{IMEI} (\acrshort{SV}) & cond. & \SI{10}{\byte} \\
		\gls{CAMEL} Charging Information Container & opt. & var. \\
		Additional Trace Info & opt. & \SI{11}{\byte} \\
		Correlation-ID & opt. & \SI{3}{\byte} \\
		Evolved Allocation Retention Priority I & opt. & \SI{3}{\byte} \\
		Extended Common Flags & opt. & \SI{3}{\byte} \\ % might be more, but seems unused 
		User \acrshort{CSG} Information & opt. & \SI{10}{\byte} \\
		\gls{APN}-\acrshort{AMBR} & opt.  & \SI{11}{\byte} \\
		Signaling Priority Indication & opt. & \SI{3}{\byte} \\ % might be more, but seems unused
		Private Extension & opt. & var. \\
		\bottomrule
	\end{tabu}}
	%\vfill
	%\null
\end{table}

Coming back to the described \gls{PDP} Context activation procedure of Figure~\ref{c4:fig:pdpcontextactivationinteraction}, it contains both the \gls{gtp} message \textit{Create \gls{PDP} Context request} as well as the \textit{response} twice. Such a Create request consists of the 36 \gls{IE} depicted in Table~\ref{c4:tab:createrequestelements}. Neglecting the elements which have no predefined upper length bound (besides the default \SI{16}{\bit} \gls{IE} length field) and assuming a maximum length for the other variable elements this results in a message size of \SI{1059}{\byte}. The complexity of the other message types is comparable.

One of the investigations that will be conducted is that of the core network load, which will be defined and discussed later in detail. \gls{gtp} messages could play an interesting role here as they may directly or indirectly contribute to this load or at least be an indicator of load existing otherwise. Load could be caused by the generated network traffic as well as the assembly, processing, and storage of the involved state in form of the \gls{IE}.

The following sections detail the three \gls{gtp} tunnel management message pairs involved in the maintenance of \gls{PDP} Contexts. These are the \textit{Create, Update,} and \textit{Delete \gls{PDP} Context requests} and \textit{responses}. They represent the basis for the core network investigations.


%%
\subsubsection{Create \gls{PDP} Context Message}

This message type is part of procedures that enable the \gls{PS} data connection and the \gls{gtp} tunnel for a mobile device. These are the \textbf{\gls{PDP} Context activation procedure} (already depicted in Figure~\ref{c4:fig:pdpcontextactivationinteraction}) and the \textbf{Secondary \gls{PDP} Context activation} for additional \gls{gtp} tunnels to the device with specific \gls{QoS} levels set. They are triggered by the mobile device through \gls{RRC}/\gls{RANAP} signaling during or following a \gls{GPRS} attach procedure.

When a \gls{GGSN} receives a create request from an \gls{SGSN}, it has to allocate the necessary resources for a \gls{PDP} Context. Depending on the outcome, a response is sent back, indicating the success or failure of the operation. Typical failures include failed user authentication, a lack of resource, or unrecoverable system failures and malformed or corrupted request.


%%
\subsubsection{Delete \gls{PDP} Context Message}

Similar to to Create Context messages, a \textbf{Delete \gls{PDP} Context request} and \textbf{response} always coincides with the termination of a \gls{gtp} tunnel and the removal of the associated \gls{PDP} Context. Create and Delete requests together mark the beginning and the end of every user traffic tunnel, making them very interesting in determining tunnel properties and perfect candidates to indirectly identify tunnel durations in a core network investigation.

Deletes are either created through explicit \gls{PDP} Context deactivation procedures or play a part in \gls{GPRS} attach and detach procedures. Contrary to creates, they can be initiated from the \gls{MS} as well as from a core network node, depending on the kind of procedure.


%%
\subsubsection{Update \gls{PDP} Context Messages}

Several procedures also emit \textbf{Update \gls{PDP} Context requests} and \textbf{responses}, usually in relation to some aspect of the tunnel or the device changing. Possible causes for an \textit{Update Context request} are:

\begin{itemize}
	\item The mobile devices moves between \glspl{SGSN}, causing a \textbf{\gls{GPRS} inter-\gls{SGSN} Routing Area Update} procedure.
	\item Parameters belonging to the context such as the assigned \gls{QoS} are altered using the \textit{\gls{PDP} Context Modification} procedure.
	\item As part of \textbf{Context redistribution and load balancing} procedures.
	\item The \gls{MS} switches between \gls{UMTS} and \gls{GPRS} access technologies, causing a \textit{Inter-system intra-\gls{SGSN} Update} procedure. Note that the same tunnel can be used regardless of the radio technology.
	\item As part of a direct \gls{RNC} to \gls{GGSN} \gls{GTP-U} tunnel activation procedure, thereby circumventing the \gls{SGSN}. Or, finally, 
	\item to activate secondary \gls{PDP} Contexts using the \textbf{Secondary PDP Context Activation} as previously described. 
\end{itemize}

However, the appearance of update message signaling in some of these procedures is conditional or even optional. This often depends on the specific implementation and is not known without in-depth knowledge of it. Only for mobility management procedures the updates are mandatory.

By observing update messages one could capture most forms of mobility happening in the network, and get a good picture of potential correlation between mobility and tunneling characteristics. 
By distinguishing portions of tunnels which were associated with a \gls{UMTS} \gls{RAB} from \gls{2G} radio access through the related update message, one could also study any influence of the access technology on the core.

Nowadays \gls{GSM}/\gls{GPRS} is either used in older models or feature phones, or in mobile scenarios in rural areas where \gls{GSM} still is prevalent due to its usage of lower frequency bands and thus larger-scale coverage. Both could indicate that the data session will be rather short because of limited device capabilities or low throughput rates of \gls{GPRS}.


%%
\subsection{\texorpdfstring{\acrshort{gtp}}{GTP} Influencing State Machines}

To understand the occurrence of these signaling procedures one should look at the state machines that govern these. Involved in the tunnel management aspects are three distinct \glspl{FSM}, namely the \gls{MM} and \gls{RRC} state machines and the actual \gls{PDP} state model. The \gls{MM} and \gls{RRC} describe the current state of the mobile device and its radio and data connection. They are both maintained at the \gls{MS} and mirrored at the \gls{SGSN}.

\begin{figure}[htb]
	\centering
	\begin{subfigure}[b]{0.60\textwidth}
		\includegraphics[width=\textwidth]{images/mm-2g-state-model.pdf}
		\caption{State machine for \acrshort{2G} radio access.}
		\label{c4:fig:2g-mmstatemodel}
	\end{subfigure}%
% Ein wenig \vskip täte hier gut :-)
	\begin{subfigure}[b]{0.70\textwidth}
		\includegraphics[width=\textwidth]{images/mm-3g-state-model.pdf}
		\caption{State machine for \acrshort{3G} radio access.}
		\label{c4:fig:3g-mmstatemodel}
	\end{subfigure}%
	\caption{\acrshort{SGSN} \acrshort{MM} state models and machines as defined in \cite[Section~6.1]{3gpp.23.060}.}
\label{c4:fig:mmstatemodel}
\end{figure}

The \gls{MM} model, defined in \cite[Section~6.1]{3gpp.23.060}, describes the general state of the data connection. State switches occur either based on an idle timer or when new packets arrive for the mobile device. The specific model depends on the currently used \gls{RAT} with only slight differences between \gls{GSM} (Figure~\ref{c4:fig:2g-mmstatemodel}) and \gls{UMTS} (Figure~\ref{c4:fig:3g-mmstatemodel}) access. The \gls{LTE} related model brings larger changes but is omitted here, as it will not be relevant for the investigation. With the transition to and from the \textbf{IDLE} state in the \gls{2G} model (or \textbf{DETACHED} in \gls{3G}) \textbf{GPRS Attach/Detach} procedures are triggered, also resulting in the transmission of \gls{PDP} Create and Delete Context messages. Likewise, other state transition procedures indicate mobility and location changes, which usually include update messages.

\begin{figure}[htb] 
	\centering
	\includegraphics[width=0.4\textwidth]{images/rrc-state-model.pdf}
	\caption{\acrshort{RRC} State Model as per \cite[Section~7.1]{3gpp.25.331}.}
	\label{c4:fig:rrcstatemodel}
\end{figure}

 The \gls{RRC} state machine given in \gls{TS}~25.331~\cite[Section~7.1]{3gpp.25.331} and depicted in Figure~\ref{c4:fig:rrcstatemodel} governs the usage of radio channels and therefore power states of the \gls{MS}. State changes happen depending on user and radio activity and inactivity which is determined by timers. Only in the \gls{CELLDCH} state is the \gls{MS} assigned a dedicated channel for its data connection and can transmit at full rate bidirectionally. However this consumes the most device power and radio resources, both of which are scarce. The goal of the state machine is to minimize resource usage with intermediary states --- \gls{CELLFACH}, \gls{URAPCH}, and \gls{CELLPCH}, and  
% and what?
 --- that successively require less power and radio channels, before completely turning off the \gls{RRC} connection by transitioning to the idle state. Coinciding with the \gls{RRC}, the \gls{CN} \gls{gtp} tunnel can also be released or needs to be reestablished. However, this is implementation specific and not precisely specified.

\begin{figure}[htb]
	\centering
	\includegraphics[width=0.35\textwidth]{images/pdp-state-model.pdf}
	\caption{\acrshort{PDP} State Model defined in \cite[Section~9]{3gpp.23.060}.}
\label{c4:fig:pdpstatemodel}
\end{figure}

The final state machine of relevance is the \textbf{\gls{PDP} State Model} from \gls{TS}~23.060~\cite[Section~9]{3gpp.23.060} in Figure~\ref{c4:fig:pdpstatemodel}. It reflects the actual state of the \gls{PDP} context and associated tunnel, and is synchronized with the \gls{MM} state machine.


%%
\subsection{Signaling Discussion}

This section on the basics of current mobile network architectures serves a critical purpose: In order to measure and evaluate network traffic one has to first understand its architecture and needs to grasp how certain traffic patterns can occur. Unfortunately, the \gls{3GPP} specifications do not make this task very easy. Typical \gls{IETF} protocols and architectures adhere to the fundamental principles of protocol layering, function separation and end-to-end. One can read a single \gls{RFC} and understand its function and directly implement it independent of the knowledge of any other specification. This is not the case in a \gls{3G} network. Functions are often spread over several protocols or nodes, necessary details essential to an implementation are spread out over several specifications without direct reference, or are even completely omitted. This circumstance makes it very hard to attribute certain observed phenomena to a specific feature in the specification.

The protocols used in \gls{3G} networks are also very heavy in terms of state and signaling inside the network. This can be the cause of unintended and hard-to-predict load, which will be defined and discussed in Section~\ref{c4:sec:loaddefinition}. For now, some possibly load information in relation to the Create, Update and Delete \gls{PDP} Context Request and Reply message pairs can already be deduced. Measuring the time delta between corresponding Create and Delete events obviously results in the total duration a tunnel was established. Having shorter tunnels often also means having a greater number of tunnels and therefore a higher volume of signaling messages and an increase in processing and state-keeping efforts due to the signaling. 

Conversely, longer tunnel durations cause an increased overall memory footprint in the involved nodes to store the \gls{PDP} Contexts. Large numbers of update messages, especially combined with frequent \gls{RAT} switches, are usually an indicator for highly mobile devices switching their routing area. The time between a request and its corresponding response could also be an indicator for the amount of processing involved for this message as well as the current general processing load at the \gls{GGSN}. Most of the actions in the network as well as in the mobile devices are reflected in the presented tunnel management messaging. Therefore, taking a look at the dynamics of this control aspect in real networks gives valuable insights on the influence of many of the networks' aspects.





%This section starts with a primer on cellular data network basics, and then moves on to describe relevant details of \gls{gtp}, the tunneling protocol under investigation.


% calculation: 36 element lengths + 36 * 1 Byte IE type fields + 11 byte header with no extensions
% 1012 + 36 + 11 = 1059B



% http://3gppinterview.blogspot.co.at/p/why-is-cellfach-and-cellpch-not.html <- this may be easier to understand


% \subsubsection{Information Elements Wire Format}

% \paragraph{IMSI}

% \begin{table}[htb]
% 	\caption{IMSI Information Element Format.}
% 	\label{c4:tbl:imsiieformat}
% 	\begin{tabu}{X[c]|X|X|X|X|X|X|X|X|}
% 	\multicolumn{1}{c}{} & \multicolumn{8}{c}{\textbf{Bits}} \\
% 	\cline{2-9} \textbf{Octets} & 8 & 7 & 6 & 5 & 4 & 3 & 2 & 1 \\ 
% 	\cline{2-9} 1 & \multicolumn{8}{c|}{Type = 1 (decimal)} \\ 
% 	\cline{2-9} 2 to 3 & \multicolumn{8}{c|}{Length = n}  \\ 
% 	\cline{2-9} 4 & \multicolumn{4}{c|}{Spare} & \multicolumn{4}{c|}{Instance} \\ 
% 	\cline{2-9} 5 & \multicolumn{4}{c|}{Number digit 2} & \multicolumn{4}{c|}{Number digit 1} \\ 
% 	\cline{2-9} 6 & \multicolumn{4}{c|}{Number digit 4} & \multicolumn{4}{c|}{Number digit 3} \\ 
% 	\cline{2-9} ... & \multicolumn{4}{c|}{...} & \multicolumn{4}{c|}{...} \\ 
% 	\cline{2-9} n+4 & \multicolumn{4}{c|}{Number digit m} & \multicolumn{4}{c|}{Number digit m-1} \\ 
% 	\cline{2-9}
% 	\end{tabu}
% \end{table}

% Decimals coded as TBCD; if odd number fill last nibble with 1; max digits is 15.\\
% Max IE size 12 Byte.

% \paragraph{APN}

% \begin{table}[htb]
% 	\caption{APN Information Element Format.}
% 	\label{c4:tbl:apnieformat}
% 	\begin{tabu}{X[c]|X|X|X|X|X|X|X|X|}
% 	\multicolumn{1}{c}{} & \multicolumn{8}{c}{\textbf{Bits}} \\
% 	\cline{2-9} \textbf{Octets} & 8 & 7 & 6 & 5 & 4 & 3 & 2 & 1 \\ 
% 	\cline{2-9} 1 & \multicolumn{8}{c|}{Type = 71 (decimal)} \\ 
% 	\cline{2-9} 2 to 3 & \multicolumn{8}{c|}{Length = n}  \\ 
% 	\cline{2-9} 4 & \multicolumn{4}{c|}{Spare} & \multicolumn{4}{c|}{Instance} \\ 
% 	\cline{2-9} 5 to (n+4) & \multicolumn{8}{c|}{Access Point Name} \\ 
% 	\cline{2-9}
% 	\end{tabu} 
% \end{table}

% Full APN name including APN Network Identifier and APN Operator Identifier.
% Network Identifier: max length 63 bytes.
% Operator Identifier: mnc<3digits>.mcc<3digits>.gprs; 16 bytes (18 incl dots).
% (Ex: ggsn-cluster-A.provinceB.mnc012.mcc345.gprs)

% Max total $4+63+16=83$

% \paragraph{AMBR}

% \begin{table}[htb]
% 	\caption{APN Information Element Format.}
% 	\label{c4:tbl:abmrieformat}
% 	\begin{tabu}{X[c]|X|X|X|X|X|X|X|X|}
% 	\multicolumn{1}{c}{} & \multicolumn{8}{c}{\textbf{Bits}} \\
% 	\cline{2-9} \textbf{Octets} & 8 & 7 & 6 & 5 & 4 & 3 & 2 & 1 \\ 
% 	\cline{2-9} 1 & \multicolumn{8}{c|}{Type = 72 (decimal)} \\ 
% 	\cline{2-9} 2 to 3 & \multicolumn{8}{c|}{Length = n}  \\ 
% 	\cline{2-9} 4 & \multicolumn{4}{c|}{Spare} & \multicolumn{4}{c|}{Instance} \\ 
% 	\cline{2-9} 5 to 8 & \multicolumn{8}{c|}{APN-AMBR for uplink} \\ 
% 	\cline{2-9} 9 to 12 & \multicolumn{8}{c|}{APN-AMBR for downlink} \\ 
% 	\cline{2-9}
% 	\end{tabu} 
% \end{table}

% Total size 12 bytes.


% \paragraph{Recovery}

% \begin{table}[htb]
% 	\caption{Recovery Information Element Format.}
% 	\label{c4:tbl:recoveryieformat}
% 	\begin{tabu}{X[c]|X|X|X|X|X|X|X|X|}
% 	\multicolumn{1}{c}{} & \multicolumn{8}{c}{\textbf{Bits}} \\
% 	\cline{2-9} \textbf{Octets} & 8 & 7 & 6 & 5 & 4 & 3 & 2 & 1 \\ 
% 	\cline{2-9} 1 & \multicolumn{8}{c|}{Type = 3 (decimal)} \\ 
% 	\cline{2-9} 2 to 3 & \multicolumn{8}{c|}{Length = n}  \\ 
% 	\cline{2-9} 4 & \multicolumn{4}{c|}{Spare} & \multicolumn{4}{c|}{Instance} \\ 
% 	\cline{2-9} 5 to (n+4) & \multicolumn{8}{c|}{Recovery (Restart Counter} \\ 
% 	\cline{2-9}
% 	\end{tabu} 
% \end{table}

% IN GTPv2 first release IE length is 5 bytes. May be longer in the future.


% \paragraph{MEI}

% \begin{table}[htb]
% 	\caption{MEI Information Element Format.}
% 	\label{c4:tbl:meiieformat}
% 	\begin{tabu}{X[c]|X|X|X|X|X|X|X|X|}
% 	\multicolumn{1}{c}{} & \multicolumn{8}{c}{\textbf{Bits}} \\
% 	\cline{2-9} \textbf{Octets} & 8 & 7 & 6 & 5 & 4 & 3 & 2 & 1 \\ 
% 	\cline{2-9} 1 & \multicolumn{8}{c|}{Type = 75 (decimal)} \\ 
% 	\cline{2-9} 2 to 3 & \multicolumn{8}{c|}{Length = n}  \\ 
% 	\cline{2-9} 4 & \multicolumn{4}{c|}{Spare} & \multicolumn{4}{c|}{Instance} \\ 
% 	\cline{2-9} 5 to (n+4) & \multicolumn{8}{c|}{Mobile Equipment (ME) Identity} \\ 
% 	\cline{2-9}
% 	\end{tabu}
% \end{table}

% 15 (IMEI) or 16 (IMEISV) BCD digits filled with 1 to full octet. Size is 12 bytes.

% \paragraph{MSISDN}

% \begin{table}[htb]
% 	\caption{MSISDN Information Element Format.}
% 	\label{c4:tbl:msisdnieformat}
% 	\begin{tabu}{X[c]|X|X|X|X|X|X|X|X|}
% 	\multicolumn{1}{c}{} & \multicolumn{8}{c}{\textbf{Bits}} \\
% 	\cline{2-9} \textbf{Octets} & 8 & 7 & 6 & 5 & 4 & 3 & 2 & 1 \\ 
% 	\cline{2-9} 1 & \multicolumn{8}{c|}{Type = 76 (decimal)} \\ 
% 	\cline{2-9} 2 to 3 & \multicolumn{8}{c|}{Length = n}  \\ 
% 	\cline{2-9} 4 & \multicolumn{4}{c|}{Spare} & \multicolumn{4}{c|}{Instance} \\ 
% 	\cline{2-9} 5 & \multicolumn{4}{c|}{Number digit 2} & \multicolumn{4}{c|}{Number digit 1} \\ 
% 	\cline{2-9} 6 & \multicolumn{4}{c|}{Number digit 4} & \multicolumn{4}{c|}{Number digit 3} \\ 
% 	\cline{2-9} ... & \multicolumn{4}{c|}{...} & \multicolumn{4}{c|}{...} \\ 
% 	\cline{2-9} n+4 & \multicolumn{4}{c|}{Number digit m} & \multicolumn{4}{c|}{Number digit m-1} \\ 
% 	\cline{2-9}
% 	\end{tabu}
% \end{table}

% MSISDN limited to 15 digits. Max total size 12 bytes.


% \paragraph{Indication}

% \begin{table}[htb]
% 	\caption{Indication Information Element Format.}
% 	\label{c4:tbl:indicationieformat}
% 	\begin{tabu}{X[c]|X|X|X|X|X|X|X|X|}
% 	\multicolumn{1}{c}{} & \multicolumn{8}{c}{\textbf{Bits}} \\
% 	\cline{2-9} \textbf{Octets} & 8 & 7 & 6 & 5 & 4 & 3 & 2 & 1 \\ 
% 	\cline{2-9} 1 & \multicolumn{8}{c|}{Type = 77 (decimal)} \\ 
% 	\cline{2-9} 2 to 3 & \multicolumn{8}{c|}{Length = n}  \\ 
% 	\cline{2-9} 4 & \multicolumn{4}{c|}{Spare} & \multicolumn{4}{c|}{Instance} \\ 
% 	\cline{2-9} 5 & DAF & DTF & HI & DFI & OI & ISRSI & ISRAI & SGWCI \\ 
% 	\cline{2-9} 6 & Spare & UIMSI & CFSI & CRSI & P & PT & SI & MSV \\ 
% 	\cline{2-9} 7 to (n+4) & \multicolumn{8}{c|}{These octet(s) is/are present only if explicitly specified} \\ 
% 	\cline{2-9}
% 	\end{tabu}
% \end{table}

% Size is 7 bytes.

% \paragraph{PCO}


% \begin{table}[htb]
% 	\caption{PCO Information Element Format.}
% 	\label{c4:tbl:pcoieformat}
% 	\begin{tabu}{X[c]|X|X|X|X|X|X|X|X|}
% 	\multicolumn{1}{c}{} & \multicolumn{8}{c}{\textbf{Bits}} \\
% 	\cline{2-9} \textbf{Octets} & 8 & 7 & 6 & 5 & 4 & 3 & 2 & 1 \\ 
% 	\cline{2-9} 1 & \multicolumn{8}{c|}{Type = 78 (decimal)} \\ 
% 	\cline{2-9} 2 to 3 & \multicolumn{8}{c|}{Length = n}  \\ 
% 	\cline{2-9} 4 & \multicolumn{4}{c|}{Spare} & \multicolumn{4}{c|}{Instance} \\ 
% 	\cline{2-9} 5 to (n+4) & \multicolumn{8}{c|}{Protocol Configuration Options} \\
% 	\cline{2-9}
% 	\end{tabu}
% \end{table}

% Minimum length 4+3-3, maximum length 4+253-3; average?


% \paragraph{PAA}

% \begin{table}[htb]
% 	\caption{PAA Information Element Format.}
% 	\label{c4:tbl:paaieformat}
% 	\begin{tabu}{X[c]|X|X|X|X|X|X|X|X|}
% 	\multicolumn{1}{c}{} & \multicolumn{8}{c}{\textbf{Bits}} \\
% 	\cline{2-9} \textbf{Octets} & 8 & 7 & 6 & 5 & 4 & 3 & 2 & 1 \\ 
% 	\cline{2-9} 1 & \multicolumn{8}{c|}{Type = 79 (decimal)} \\ 
% 	\cline{2-9} 2 to 3 & \multicolumn{8}{c|}{Length = n}  \\ 
% 	\cline{2-9} 4 & \multicolumn{4}{c|}{Spare} & \multicolumn{4}{c|}{Instance} \\ 
% 	\cline{2-9} 5 & \multicolumn{5}{c|}{Spare} & \multicolumn{3}{c|}{PDN Type} \\
% 	\cline{2-9} 6 to (n+4) & \multicolumn{8}{c|}{PDN Adress and Prefix} \\
% 	\cline{2-9}
% 	\end{tabu} 
% \end{table}

% Either 9 (IPv4), 22 (IPv6), or 26 (IPv4v6).


% \paragraph{RAT Type}


% \begin{table}[htb]
% 	\caption{RAT Information Element Format.}
% 	\label{c4:tbl:ratieformat}
% 	\begin{tabu}{X[c]|X|X|X|X|X|X|X|X|}
% 	\multicolumn{1}{c}{} & \multicolumn{8}{c}{\textbf{Bits}} \\
% 	\cline{2-9} \textbf{Octets} & 8 & 7 & 6 & 5 & 4 & 3 & 2 & 1 \\ 
% 	\cline{2-9} 1 & \multicolumn{8}{c|}{Type = 82 (decimal)} \\ 
% 	\cline{2-9} 2 to 3 & \multicolumn{8}{c|}{Length = n}  \\ 
% 	\cline{2-9} 4 & \multicolumn{4}{c|}{Spare} & \multicolumn{4}{c|}{Instance} \\ 
% 	\cline{2-9} 5 & \multicolumn{8}{c|}{RAT Type} \\
% 	\cline{2-9} 6 to (n+4) & \multicolumn{8}{c|}{These octet(s) is/are present only if explicitly specified} \\
% 	\cline{2-9}
% 	\end{tabu} 
% \end{table}

% Maximum length 5 to ?.

% \paragraph{Serving Network}

% \begin{table}[htb]
% 	\caption{Serving Network Information Element Format.}
% 	\label{c4:tbl:servingnetieformat}
% 	\begin{tabu}{X[c]|X|X|X|X|X|X|X|X|}
% 	\multicolumn{1}{c}{} & \multicolumn{8}{c}{\textbf{Bits}} \\
% 	\cline{2-9} \textbf{Octets} & 8 & 7 & 6 & 5 & 4 & 3 & 2 & 1 \\ 
% 	\cline{2-9} 1 & \multicolumn{8}{c|}{Type = 83 (decimal)} \\ 
% 	\cline{2-9} 2 to 3 & \multicolumn{8}{c|}{Length = n}  \\ 
% 	\cline{2-9} 4 & \multicolumn{4}{c|}{Spare} & \multicolumn{4}{c|}{Instance} \\ 
% 	\cline{2-9} 5 & \multicolumn{4}{c|}{MCC digit 2} & \multicolumn{4}{c|}{MCC digit 1} \\ 
% 	\cline{2-9} 6 & \multicolumn{4}{c|}{MNC digit 3} & \multicolumn{4}{c|}{MCC digit 3} \\ 
% 	\cline{2-9} 7 & \multicolumn{4}{c|}{MNC digit 2} & \multicolumn{4}{c|}{MNC digit 1} \\ 
% 	\cline{2-9} 8 to (n+4) & \multicolumn{8}{c|}{These octet(s) is/are present only if explicitly specified} \\
% 	\cline{2-9}
% 	\end{tabu}
% \end{table} 

% Maximum length 7 to ?.


% \paragraph{User Location Information}

% \begin{table}[htb]
% 	\caption{User Location Information Element Format.}
% 	\label{c4:tbl:userlocieformat}
% 	\begin{tabu}{X[c]|X|X|X|X|X|X|X|X|}
% 	\multicolumn{1}{c}{} & \multicolumn{8}{c}{\textbf{Bits}} \\
% 	\cline{2-9} \textbf{Octets} & 8 & 7 & 6 & 5 & 4 & 3 & 2 & 1 \\ 
% 	\cline{2-9} 1 & \multicolumn{8}{c|}{Type = 86 (decimal)} \\ 
% 	\cline{2-9} 2 to 3 & \multicolumn{8}{c|}{Length = n}  \\ 
% 	\cline{2-9} 4 & \multicolumn{4}{c|}{Spare} & \multicolumn{4}{c|}{Instance} \\ 
% 	\cline{2-9} 5 & \multicolumn{3}{c|}{Spare} & ECGI & TAI & RAI & SAI & CGI \\ 
% 	\cline{2-9} a to a+6 & \multicolumn{8}{c|}{CGI} \\ 
% 	\cline{2-9} 7 & \multicolumn{8}{c|}{SAI} \\ 
% 	\cline{2-9} 7 & \multicolumn{8}{c|}{RAI} \\ 
% 	\cline{2-9} 7 & \multicolumn{8}{c|}{TAI} \\ 
% 	\cline{2-9} 7 & \multicolumn{8}{c|}{ECGI} \\ 
% 	\cline{2-9} 8 to (n+4) & \multicolumn{8}{c|}{These octet(s) is/are present only if explicitly specified} \\
% 	\cline{2-9}
% 	\end{tabu} 
% \end{table}



% Information Elements Table for PDP Context Activation Case only for GTPv2 (LTE)
% \begin{longtabu} to\linewidth{| X[2,l] | X[2,c] | X[l] | X[4] |}
% \hline
% Information Element 						& IE Type 					& Max Wire Size (Bytes)	& Comment \\ \hline
% \gls{IMSI} 										& IMSI 						& 12					& \\ \hline
% \gls{MSISDN} 										& MSISDN					& 12					& On S11 Interface if provided by HSS; In case of UE requested connectivity if MME has it stored. \\ \hline
% MEI Identity 								& MEI 						& 12					& If available at MME. \\ \hline
% User Location Information 					& ULI						& 						& E-UTRAN initial attach \&  UE requested connectivity only; included by S-GW if received from MME via S5/S8; included on S4 and S5/S8 for PDP context activation, either CGI, SAI, or RAI. \\ \hline
% Serving Network								& Serving Network			& 						& Initial E-UTRAN attach, context activation and UE requested connectivity \\ \hline
% \gls{RAT} Type									& RAT Type					& 5						& \\ \hline
% Indication Flags							& Indication				& 6						& Flags: S5/S8 Protocol Type; Dual Address Bearer Flag; Handover Indication; Direct Tunnel Flag; Piggybacking Supported; Change Reporting Support Indication \\ \hline
% Sender F-TEID for Control Plane				& F-TEID					& 						& \\ \hline
% P-G S5/S8 Address for Control Plane or PMIP	& F-TEID					& 						& On S11/S4 interfaces; 0 if initial attach, context activation or PDN connectivity \\ \hline
% Access Point Name							& APN						& 83					& \\ \hline
% Selection Mode								& Selection Mode			& 						& Indicate whether subscribed or non-subscribed, chosen by MME, was selected \\ \hline
% PDN Type									& PDN Type					& 						& IPv4, IPv6 or IPv4v6. \\ \hline
% PDN Address Allocation						& PAA						& 26					& Set to static IP address; else (dynamic) to 0.0.0.0 or IPv6 Prefix Length 0. \\ \hline
% Maximum APN Restriction						& APN Restriction			& 						& Set to most stringent restriction of any active bearer. \\ \hline
% Aggregate Maximum Bit Rate					& ABMR						& 12					& \\ \hline
% Protocol Configuration Options				& PCO						& 254					& Forwarded from UE to P-GW via S-GW via MME. \\ \hline
% Bearer Contexts to be created				& Bearer Context			& 						& present multiple times to represent list of bearers \\ \hline
% Trace Information							& Trace Information 		& 						& If S-GW / P-GW is activated. \\ \hline
% Recovery									& Recovery					& 5						& If peer node contacted for the first time. \\ \hline
% MME-FQ-CSID									& FQ-CSID					& 						& Included by MME on S11 \\ \hline
% SGW-FQ-CSID									& FQ-CSID					& 						& Included by SGW on S5/S8 \\ \hline
% UE Time Zone								& UE Time Zone 				& 						& Can be included by MME on S11; forwarded to P-GW via S-GW \\ \hline
% User CSG Information						& UCI						& 						& If \gls{UE} accessed via CSG cell or hybrid cell \\ \hline
% Charging Characteristics					& Charging Characteristics	&						& \\ \hline
% Private Extensions							& Private Extensions		&						& \\ \hline
% \end{longtabu}





% \begin{figure}[htb]
% 	\centering
%  	\includegraphics[width=0.9\textwidth]{images/eNB-MME-layers.pdf}
%  	\caption{Control plane protocol stack at the S1-MME interface between eNodeB and MME.}
%  	\label{c4:fig:stack-enbmme}
% \end{figure}

% \begin{figure}[htb]
% 	 \centering
% 	 \includegraphics[width=0.9\textwidth]{images/SGSN-MME-layers.pdf}
% 	 \caption{Control plane protocol stack at the S3 interface between SGSN and MME.}
% 	 \label{c4:fig:stack-sgsnmme}
% \end{figure}

% \begin{figure}[htb]
% 	\centering
% 	\includegraphics[width=0.9\textwidth]{images/S-GW-P-GW-layers.pdf}
% 	\caption{Optional control plane protocol stack at the S5 interface between SGW and PGW.}
% 	\label{c4:fig:stack-sgwpgw}
% \end{figure}

% \begin{figure}[htb]
% 	\centering
% 	\includegraphics[width=0.9\textwidth]{images/MME-S-GW-layers.pdf}
% 	\caption{Control plane protocol stack at the S11 interface between MME and SGW.}
% 	\label{c4:fig:stack-mmesgw}
% \end{figure}



%List of interfaces in the 3G/LTE PS network
%\begin{itemize}
%\item \textbf{Uu}: Interface between the mobile station (MS) and the fixed network part in Iu mode. The Uu interface is the Iu mode network interface for providing packet data services over the radio to the MS. The MT part of the MS is used to access the UMTS services through this interface.
%\item \textbf{Iub}: Interface between a NodeB and a RNC.
%\item \textbf{IuPS}: Interface between a RNC and a SGSN.
%\item \textbf{S1-U}: Interface between a eNodeB and a S-GW. User plane bearer tunneling.
% \item \textbf{S1-MME}: Interface between a eNodeB and a MME.
% \item \textbf{S3}: Interface between a SGSN and a MME. User/bearer information exchange for active/idle state 3g network access mobility.
% \item \textbf{S4}: Interface between a SGSN and a S-GW.	 2G user plane tunneling. GPRS mobility and control.
% \item \textbf{S5}: Interface between a S-GW and a P-GW within the same PLMN. User plane tunneling; S-GW relocation due to mobility.
% \item \textbf{S6a}: Interface between a MME and a HSS. Auth/auth data transfer to evolved system.
% \item \textbf{Gr/S6d}: Interface between a SGSN and a HSS. 
% \item \textbf{S8}: Interface between a S-GW and a P-GW in different PLMNs. Inter-PLMN variant to S5.
% \item \textbf{S9}: Interface between a PRCF and the packet data network. Data exchange to visited PCRF PLMN.
% \item \textbf{S11}: Interface between a S-GW and a MME.
% \item \textbf{S12}: UTRAN to S-GW reference point. Based on Iu-u/Gn-u. Direct Tunnel via GTP-U.
% \item \textbf{S13}: Interface between a MME and a EIR. UE identity check.
% \item \textbf{SGi}: The reference point between the EPC based PLMN and the packet data network. Same as Gi for 3gpp.

% \item \textbf{GC}: Interface between a HSS and a GGSN.
% \item \textbf{Gf}: Interface between a SGSN and a EIR.
% \item \textbf{Gi}: Reference point between Packet Domain and an external packet data network.
% \item \textbf{Gn}: Interface between two GSNs within the same PLMN.
% \item \textbf{Gp}: Interface between two GSNs in different PLMNs. The Gp interface allows support of Packet Domain network services across areas served by the co-operating PLMNs.
% \item \textbf{Gx}: Interface between a PCRF and a P-GW/GGSN. QoS policy and charging rules transfer.
% \item \textbf{Gxc}: Interface between a PCRF and a S-GW.

% \item \textbf{Rx}: Interface between a PRCF and the packet data network.
% \end{itemize}


%EMM Service request procedure
% \begin{figure}[htb]
% 	\centering
% 	\includegraphics[width=1.0\textwidth]{images/UE-service-request.pdf}
% 	\caption{EMM service request procedure sequence diagram.}
% 	\label{c4:fig:3gpp-ueservicereq}
% \end{figure}

% Annotations:
% 1. Encapsulated in RRC message.
% 2. Forwarded in S1-AP Initial UE Message.
% 3. Various security procedures.

%	25.401 \cite{3gpp.25.401} UTRAN overall description
%	25.931 \cite{3gpp.25.931} UTRAN functions, examples on signalling procedures
%	23.401 \cite{3gpp.23.401} \gls{E-UTRAN} procedures (LTE only)
%	24.007 \cite{3gpp.24.007} radio interface signaling % only Um interface in plain GSM/GPRS
%	36.300 \cite{3gpp.36.300} \gls{E-UTRAN} description (LTE only) 
%	36.414 \cite{3gpp.36.414} Evolved Universal Terrestrial Radio Access Network (E-UTRAN); S1 data transport (radio bearer)
%	22.060 \cite{3gpp.22.060} basic and short \gls{GPRS} service description; unchanged since Release 6 (2004)
%	23.060 \cite{3gpp.23.060} \gls{GPRS} description : \gls{GPRS} specific procedures, interfaces and nodes; mobility management; radio management; packet routing; operational aspects



	%23.402 \cite{3gpp.23.402} (LTE only) non-\gls{3GPP} accesses
	%24.301 \cite{3gpp.24.301} EPS Non-Access-Stratum protocol between UE and MME on Uu

% Relevant protocols and interfaces between nodes:
% \begin{itemize}
% 	\item \gls{gtp}, \gls{gtpv2}, GTP-u, GTP-c, \gls{SGSN} to \gls{GGSN} and others (specify!), on top of \gls{UDP}, GTPv1 will be described in detail in a separate section as it is at the core of the upcoming investigations.
% 	\item \gls{MAP} / \gls{SS7}: \gls{SGSN} - \gls{HLR}, \gls{GGSN} - \gls{HLR} and others; subscriber management and information exchange
% 	\item Diameter \cite{rfc6733}: \gls{MME} - \gls{HSS}, \gls{SGSN} - \gls{HSS}, replacement for \gls{MAP}, subscriber management

% 	\item PMIPv6 on \gls{SGW} - \gls{PGW}, alternative tunneling protocol to GTPv2
% \end{itemize}


% \begin{figure}[htb]
% 	\centering
% 	\includegraphics[width=1.0\textwidth]{images/3g-userplane.pdf}
% 	\caption{User plane protocol stack in an UMTS network.}
% 	\label{c4:fig:3gpp-umtsuserplane}
% \end{figure}

% \begin{figure}[htb]
% 	\centering
% 	\includegraphics[width=1.2\textwidth]{images/LTE-userplane.pdf}
% 	\caption{User plane protocol stack in an LTE/EPC network.}
% 	\label{c4:fig:3gpp-lteuserplane}
% \end{figure}

% \begin{figure}[htb]
% 	\centering
% 	\includegraphics[width=1.0\textwidth]{images/bearers.pdf}
% 	\caption{3GPP bearer model.}
% 	\label{c4:fig:3gpp-bearers}
% \end{figure}


% \begin{figure}[htb]
% 	\centering
% 	\includegraphics[width=1.2\textwidth]{images/ECM-states.pdf}
% 	\caption{\gls{ECM} state machine.}
% 	\label{c4:fig:3gpp-ecmstates}
% \end{figure}


% \begin{figure}[htb]
% 	\centering
% 	\includegraphics[width=0.8\columnwidth]{images/pdp-context-activation.pdf}
% 	\caption{PDP Context Activation Procedure in a UMTS network.}
% 	\label{c4:fig:pdpcontextactivation}
% \end{figure}



% \begin{itemize}
% \item 	Every bearer has a predefined QoS level between UE and P-GW.
% 		==> Level of Granularity for QoS control.
% \item	Initial bearer QoS level assigned by network based on subscription data.
% \item	Guaranteed Bit Rate (GBR) bearers: dedicated network resources permanently allocated at est/mod. Otherwise Non-GBR.
% \item	The Traffic Flow Template (TFT) belonging to a bearer is a set of packet filters that assign traffic flows to the bearer.
% \item	UL-TFT at UE, DL-TFT at \gls{PCEF} (P-GW).
% \item 	default bearer: always-on IP connectivity for the UE to a PDN
% \item	dedicated bearer:   
% 			\begin{itemize}
% 				\item any additional bearer for the same PDN
% 				\item \gls{TFT} associated with every ded. bearer
% 				\item establishment/modification decision only by \gls{EPC}
% 				\item QoS level assignment only by \gls{EPC}
% 			\end{itemize}

% \item	default bearer may be used as {m,c}atch-all traffic bearer for everything that does not match any filter
% \item	Every bearer associated with QCI and ARP.

% QoS class identifier (QCI): standardized scalar as reference for node-specific QoS parameters
% Allocation and Retention Policy (ARP): priority level preemption capability, preemption vulnerability.

% \item	All simultaneously active bearers by one UE are provided are provided by the same P-GW.
% \end{itemize}

% \begin{figure}[htb]
% 	\centering
% 	\includegraphics[width=0.6\columnwidth]{images/signalling-stack.pdf}
% 	\caption{Typical signaling protocol stack at the Gn interface between \gls{SGSN} and \gls{GGSN}.}
% 	\label{c4:fig:signallingstack}
% \end{figure}

% \begin{table}[htb]
% 	\caption{GTP header format (TODO: which one exactly. This looks like v2-U but should actually better be v1!)}
% 	\label{c4:tbl:gtpheader}
% 	\begin{tabu}{c|c|c|c|c|c|c|c|c|}
% 	\multicolumn{1}{c}{} & \multicolumn{8}{c}{\textbf{Bits}} \\
% 	\cline{2-9} \textbf{Octets} & 8 & 7 & 6 & 5 & 4 & 3 & 2 & 1 \\ 
% 	\cline{2-9} 1 & \multicolumn{3}{c|}{Version}  & P & T & Spare & Spare & Spare \\ 
% 	\cline{2-9} 2 & \multicolumn{8}{c|}{Message Type}  \\ 
% 	\cline{2-9} 3 & \multicolumn{8}{c|}{Message Length (1st Octet)}  \\ 
% 	\cline{2-9} 4 & \multicolumn{8}{c|}{Message Length (2nd Octet)}  \\ 
% 	\cline{2-9} m to & \multicolumn{8}{c|}{\multirow{2}{10cm}{If T flag is set to 1, then TEID shall be placed into octets 5-8. Otherwise, TEID field is not present at all.}} \\ 
% 	 k(m+3) & \multicolumn{8}{c|}{} \\ 
% 	\cline{2-9} n to (n+2) & \multicolumn{8}{c|}{Sequence Number} \\ 
% 	\cline{2-9} (n+3) & \multicolumn{8}{c|}{Spare} \\ 
% 	\cline{2-9} 
% 	\end{tabu} 
% \end{table}


%LTE: mention also \gls{PMIPv6} as \gls{gtp} alternative, and the option to have a combined \gls{SGW} \gls{PGW}


% request/response messaging
% response types:
% Possible context response types and which request types they answer:
% \begin{itemize}
% \item 192: ``non-existent'' UPDATE \& DELETE ONLY
% \item 193: ``invalid message format'' UPDATE \& DELETE ONLY
% \item 199: ``no resources available'' CREATE ONLY anywhere in the network to allocate context
% \item 200: ``service not supported'' UPDATE ONLY
% \item 201: ``mandatory IE incorrect''
% \item 202: ``mandatory IE missing''
% \item 204: ``system failure'' CREATE \& UPDATE ONLY
% \item 209: ``user authentication failed'' CREATE ONLY rejected for various reasons
% \end{itemize}

%---
%NSAPI {0;15} Integer
%linked \gls{NSAPI}: indicates the \gls{NSAPI} assigned to any one of the already activated \gls{PDP} contexts for this address/phone ("foreign key"?)

% TODO: convert to UMTS

% \begin{figure}[htb]
% 	\centering
% 	\includegraphics[width=1.0\textwidth]{images/UE-requested-PDN-connectivity.pdf}
% mdcm	\caption{\gls{PDN} connectivity request by the UE procedure sequence diagram.}
% 	\label{c4:fig:3gpp-uepdnreq}
% \end{figure}


%% creates
%IPv6 Stateless Address Autoconfiguration Procedure
%PDP Context Activation Procedure for A/Gb mode (additional (optional?) update pair)
%PDP Context Activation Procedure for Iu mode (additional (optional?) update pair)
%Secondary PDP Context Activation for Iu mode (additional (optional?) update pair)
%Secondary PDP Context Activation for A/Gb mode (additional (optional?) update pair)

%% deletes
%MS Initiated PDP Context Deactivation Procedure for A/Gb mode
%MS Initiated PDP Context Deactivation Procedure for Iu mode
%SGSN-initiated PDP Context Deactivation Procedure
%GGSN-initiated PDP Context Deactivation Procedure
%Combined GPRS/IMSI Attach Procedure (2x request+response, 1 to old, 1 to new)
%MS-Initiated Detach Procedure (1x request/response)
%Network-Initiated Detach Procedures (SGSN or HLR)


%% updates
%Inter SGSN Routeing Area Update (1x)
%Combined Inter SGSN RA/LA Update (1x)
%Iu mode RA Update Procedure (1x)
%SRNS Serving Radio Network Subsystem Relocation Procedure (1x)
%Combined Hard Handover and SRNS Relocation Procedure
%Combined Cell/URA (UTRAN Registration Area) Update and SRNS Relocation Procedure
%Enhanced Serving RNS Relocation
%Iu mode to A/Gb mode Intra SGSN Change
%Iu mode to A/Gb mode Inter SGSN Change
%A/Gb mode to Iu mode Inter SGSN Change
%SGSN-Initiated PDP Context Modification Procedure, A/Gb mode (2x pair (1 optional?))
%SGSN-Initiated PDP Context Modification Procedure, Iu mode (2x pair (1 optional?))
%GGSN-Initiated PDP Context Modification Procedure, Iu mode
%GGSN-Initiated PDP Context Modification Procedure, A/Gb mode
%MS-Initiated PDP Context Modification Procedure, A/Gb mode (2x, 1 optional/conditional)
%MS-Initiated PDP Context Modification Procedure, Iu mode (2x, 1 optional/conditional)
%PDP Context Activation Procedure for A/Gb mode (additional (optional?) update pair)
%PDP Context Activation Procedure for Iu mode (additional (optional?) update pair)
%Secondary PDP Context Activation for Iu mode (additional (optional?) update pair)
%Secondary PDP Context Activation for A/Gb mode (additional (optional?) update pair)
%RAB Release Procedure Using Gn/Gp (conditional if direct tunnel and context to be preserved)
%Iu Release Procedure Using Gn/Gp (conditional if direct tunnel and context to be preserved)
%RAB Assignment Procedure Using Gn/Gp (conditional direct tunnel)

%%%%%%%%%%%%%%%%%%%%%%%%%%%%%%%%%%%%%%%%%%%%%%%%%%%%%%%%%%%%%%%%%%%%%%%%%%%%%%%%
%\input{3gppdesc.tex}


% \begin{multicols}{2}
% \setbox\ltmcbox\vbox{
% \makeatletter\col@number\@ne
% 	\begin{longtabu}{|X[2.5]|X[1.2]|X[0.7]|} \hline
% 		\textbf{\gls{IE}} & \textbf{Presence} & \textbf{Size}\\ \hline
% 		\gls{IMSI} & cond. & \SI{8}{\byte} \\ \hline
% 		\acrshort{RAI} & opt. & \SI{6}{\byte} \\ \hline
% 		Recovery & opt. & \SI{1}{\byte} \\ \hline
% 		Selection mode	& cond. & \SI{1}{\byte} \\ \hline
% 		\gls{TEID} Data I & mand. & \SI{4}{\byte} \\ \hline
% 		\gls{TEID} Control Plane & cond. & \SI{4}{\byte} \\ \hline
% 		\gls{NSAPI} & mand. & \SI{1}{\byte} \\ \hline
% 		Linked \gls{NSAPI} & cond. & \SI{1}{\byte} \\ \hline
% 		Charging Characteristics & cond. & \SI{2}{\byte} \\ \hline
% 		Trace Reference & opt. & \SI{2}{\byte} \\ \hline
% 		Trace Type & opt. & \SI{2}{\byte} \\ \hline
% 		End User Address & cond. & \SI{8}{\byte} \\ \hline
% 		\gls{APN} & cond. & max \SI{102}{\byte} \\ \hline % APN format defined in 23.003 section 9
% 		\acrshort{PCO} & opt. & max \SI{255}{\byte} \\ \hline % defined in 24.008 section 10.5.6.3
% 		\gls{SGSN} signaling address & mand.  & \SI{6}{\byte} \\ \hline % defined in 23.003 section 5 (without address type and length fields)
% 		\gls{SGSN} user traffic address & mand. & \SI{6}{\byte} \\ \hline % same as above
% 		\gls{MSISDN} & cond. & max \SI{17}{\byte} \\ \hline % ITU-T E.164 msisdn format recommendation of max 15 chars
% 		\gls{QoS} Profile & mand. & max \SI{257}{\byte} \\ \hline
% 		\gls{TFT} & cond. & max \SI{257}{\byte} \\ \hline % defined in 24.008 section 10.5.6.12
% 		 Trigger Id & opt. & var. \\ \hline % no definition found
% 		 \acrshort{OMC} Identity & opt. & var. \\ \hline % maybe in MAP 29.002, however not further definition found
% 		 Common Flags & opt. & \SI{3}{\byte} \\ \hline
% 		 \gls{APN} Restriction & opt. & \SI{3}{\byte} \\ \hline
% 		 \gls{RAT} & opt. & \SI{3}{\byte} \\ \hline
% 		 User Location Information & opt. & \SI{10}{\byte} \\ \hline
% 		 \gls{MS} Time Zone & opt. & \SI{4}{\byte} \\ \hline
% 		 \gls{IMEI}(\acrshort{SV}) & cond. & \SI{10}{\byte} \\ \hline
% 		 \gls{CAMEL} Charging Information Container & opt. & var. \\ \hline
% 		 Additional Trace Info & opt. & \SI{11}{\byte} \\ \hline
% 		 Correlation-ID & opt. & \SI{3}{\byte} \\ \hline
% 		 Evolved Allocation Retention Priority I & opt. & \SI{3}{\byte} \\ \hline
% 		 Extended Common Flags & opt. & \SI{3}{\byte} \\ \hline % might be more, but seems unused 
% 		 User \acrshort{CSG} Information & opt. & \SI{10}{\byte} \\ \hline
% 		 \gls{APN}-\acrshort{AMBR} & opt.  & \SI{11}{\byte} \\ \hline
% 		 Signaling Priority Indication & opt. & \SI{3}{\byte} \\ \hline % might be more, but seems unused
% 		 Private Extension & opt. & var. \\ \hline
% 	\end{longtabu}
% \unskip
% \unpenalty
% \unpenalty}

% \unvbox\ltmcbox

% \end{multicols}


%%%%%%%%%%%%%%%%%%%%%%%%%%%%%%%%%%%%%%%%%%%%%%%%%%%%%%%%%%%%%%%%%%%%%%%%%%%%%%%
%!TEX root = ../../dissertation.tex
%%%%%%%%%%%%%%%%%%%%%%%%%%%%%%%%%%%%%%%%%%%%%%%%%%%%%%%%%%%%%%%%%%%%%%%%%%%%%%%%
\section{Related Work}
\label{c4:sec:relwork}

The investigations conducted in both this and the subsequent chapter do not fall strictly into an existing research category but instead aim to provide diverse insights into the control plane from the perspective of the core network. Nonetheless, a selection of publications from the tackled fields is collected here and the interesting aspects for this work are noted. In the following sections the related work is divided into four distinct fields.

Work in the first and second sections evaluate properties of the mobile network and its traffic. They are distinguished in their approach to the investigation, as the first group uses active measurements from mobile devices or conclude from other sources of traffic whereas to the other one has access to passive measurements from inside a \gls{3G} mobile network. Publications from the third category can be generally subsumed under the term \textit{traffic modeling} and may not be specific to cellular networks. The final field concerns itself with investigative work conducted by the responsible standardization and organizational bodies themselves, i.e., the \gls{3GPP} and \gls{GSMA}.


%%
\subsection{Device Active Measurement Investigations}

The approach taken by active measurement studies is simple yet still very insightful. They are performed by writing custom application layer measurement programs for a mobile device. Specific traffic patterns are then generated, recorded, and evaluated. While this can provide very detailed information about the higher network layers, it is limited both in lower layer information as well as scale, due to being limited to a rather low number of devices.

Despite being more or less completely specified in the \gls{3GPP} documents, there is no open layer 1 and 2 (together also called ``baseband'') implementation for \gls{3G}.\footnote{Apart from OsmocomBB (\url{http://bb.osmocom.org/trac/}), but it only provides \gls{GSM} and partial \gls{GPRS} functionality.} Therefore, the baseband's behavior can not be directly instrumented from the application layer. Attempts to infer some properties are still worth conducting as the following selection of publication demonstrates.

In~\cite{Xu:2011:CDN:2007116.2007149} Xu et al.\ use data from a location service combined with active measurements to determine the possible geographic location of a \gls{GGSN} in order to improve the location of application content caches for the current network infrastructure. Similarly, in \cite{sigcomm11middleboxes} Wang et al.\ developed a program to probe mobile networks for middle boxes. That term includes any node that alters traffic and affects performance not intended by the actual end-to-end protocols. Examples are \gls{CGN}~\cite{rfc7021}, firewalls, or intercepting \gls{HTTP} proxies. A large number of such nodes were present in the investigated mobile networks and resulted in increased device power usage and download durations and even pose security issues themselves.

Concerning methods to infer specific baseband and \gls{RRC} state machine timer values with active measurements, a 2007 paper~\cite{4640935} presents a way to do this by transmitting packets with a varying inter-departure time and studying the resulting arrival pattern. Indeed, the dynamics of the radio interface's \gls{RRC} signaling and involved state machines are under investigation by several publications. However, almost all focus solely on the impact at the radio interface but pay little attention to potential implications in the \gls{CN}.

The aforementioned work is continued in \cite{5360763} and uses the presented tools to derive \gls{RRC} transitions and power usage from traffic patterns. They found, that operators have a rather larger freedom in configuring the mobile network control plane state machines and deviate from the standard and even omit some states completely.

A further example of cross-layer influences in mobile cellular networks is \cite{qian2011profiling}. It discusses the impact of application layer behavior on \gls{RRC} signaling and its consequences for device energy consumption and radio channel allocation efficiency. The authors argue that there is much room for improvement in this area, and propose some enhancements.

This is further elaborated on by research from Schwartz et al.\cite{schwartz2013angrybirds} using the same technique to analyze the radio signaling load and thus power efficiency from several mobile phone applications. The impact of custom set state machine timers interacting with application traffic is further investigated and the \gls{QoE} is investigated.


%%
\subsection{Research Based On Network Traces}

The second approach to mobile network investigations comes in the form of recording and evaluation traffic traces inside the network. This brings a much larger experiment scale with it, albeit usually at the cost of some finer grained details in the higher protocol layers because of aggregation to flow level. With core network measurements, the signaling traffic of the observed link can also be directly investigated, which is a huge benefit compared to the guesswork in active measurements.

The authors of \cite{4675847} investigate the influence of individual \gls{CN} nodes on the one-way delay distribution of user traffic packets. According to the work, the latency portion added by the \gls{SGSN} is larger but also fluctuating more, while the \gls{GGSN} added a small but steady amount of latency. This provides us with initial clues on the expected load impact of the \gls{CN} for the investigations in this work.

Following up on the topic of mobile network one-way delays is Laner~et~al.\ in \cite{laner2012delaycomparison}. The end-to-end latency of an early \gls{LTE}/\gls{EPC} network implementation is compared to that of a \gls{HSPA} network at several measurement points in the networks. The results show a lower median latency for \gls{LTE}, despite some scenarios still being in favor of \gls{3G} networks.

The authors of \cite{Shafiq:2012:FLC:2254756.2254767} limit their focus to a specific subset of connected devices, namely those of \gls{M2M} type. These are small automated devices that periodically send out data, e.g., sensor readings, or receive control commands. The paper attempts to characterize these on the basis of their generated mobile network traffic. The patterns are clearly distinguishable from traffic caused by other device types such as smartphones.

A 2012 publication~\cite{Zhang:2012:UCC:2377677.2377764} presents us with a more general look on the traffic composition of cellular access networks in comparison to wired access network. More and shorter flows are occurring in the case of cellular networks. It will be interesting to see if this shorter-but-more theme is also evident in signaling traffic. Additionally, even traffic pattern distinctions between types of applications are made showing a wide range of possible outcomes across the investigated applications.

Both the authors of \cite{shafiq2011characterizing} and \cite{paul2011understanding} take the approach of looking at high-level user traffic characteristics in a mobile network, focusing on temporal and spatial variations of user traffic volume and peeking at the influence of different devices on this metric. Additionally, \cite{baer2011two} delivers a theoretical introduction on how to conduct large scale network measurements and compares some data evaluation approaches. The 2008 paper of \cite{4570772} takes a look at times scales and time of day deviations observed in aggregated user traffic in a mobile network.

Up until now no trace-based investigation considered the control plane in their evaluation. The following publications include this at least to some degree.

In 2006, Svoboda~et~al.~\cite{svoboda2006composition} conducted a core network measurement study of various user traffic related patterns, and also provided an initial insight into \gls{PDP} context activity and durations. Another paper~\cite{lee2007detection} combines simulations based on WiFi and synthetic traces with prior knowledge of \gls{RRC} states and their effects to investigate detection methods for signaling \gls{DDoS} occurring on the radio interface. A possible magnitude of this type of attack is discussed. This also gives an indication of the correlation between user traffic patterns and radio signaling.

A 2010 publication\cite{Qian:2010:CRR:1879141.1879159} uses the indirect \gls{RRC} inferring method described earlier on a core network \gls{TCP} trace data set and finds that the involved \gls{RRC} state machine is largely inefficient in terms of signaling overhead and the device's energy consumption for the traffic patterns seen in the data. 

A more recent publication at \cite{he2012panoramic} performs a \gls{RRC} investigation at the path between \gls{RNC} and \gls{SGSN}. The authors classify their evaluations based on device model and vendor and on the application type, and find that different devices have strongly different \gls{RRC} characteristics, which could possibly also have an impact on \gls{gtp} signaling. Here, the \gls{RRC} evaluation was done in a direct manner using explicit logs from the \gls{RNC}. A final paper~\cite{Ricciato2010551} recaps some general attack scenarios on \gls{3G} networks that exploit the specific \gls{3GPP} system design. These are often closely related to the control plane.


%%
\subsection{Traffic Modeling}

Extracting viable models from mobile traffic measurements will also play a significant role onwards. The first related work is a survey of source modeling approaches for \gls{GPRS} user traffic from the year 2000 \cite{staehle2000source}. Models for \gls{HTTP} traffic and user behavior are compared and a combined model is recommended. One has to keep in mind, though, that due to the rapid developments in the Web in recent years those models might no longer be valid. 

Similarly, the authors of \cite{965876} derive a synthetic \gls{UMTS} traffic model from wired dial-up traces. By using a batch Markovian arrival process they characterize session traffic in most cases with a lognormal distribution.

Work conducted in \cite{Halepovic:2005:CMU:1089803.1089969} derives a model for the users' mobility in a mobile network. The mobility model is however more focused on the circuit-switched voice communication features of a phone. Likewise, the authors of \cite{Pesch2005385} introduce a traffic model for \gls{SIP} \gls{VoIP} communication in \gls{UMTS} networks. However, this model is specific to the \gls{IMS} domain of \gls{UMTS} and potentially not applicable to the more common over-the-top pure \gls{SIP} traffic. The model additionally investigates some initial \gls{UMTS} control plane timing values, such as the processing time of \gls{PDP} context activation messages.

A further publication in 2005 \cite{Landman200568} attempts to model the delay of \gls{IP} packets passing through an \gls{UMTS} network using a batch Markovian arrival process. However, the model specifically focuses solely on the delay originating from processing at the radio link and not at the core nodes.

Finally, a further paper by Laner~et~al.~\cite{6214330} investigates, amongst other things, a user's session duration and throughput in a \gls{HSDPA} network. The duration is modeled as an exponential distributions and the throughput using a lognormal distribution, albeit both exhibit additional heavy tail characteristics.


%%
\subsection{\texorpdfstring{\acrshort{3GPP}}{3GPP} and \texorpdfstring{\acrshort{GSMA}}{GSMA} Related Work}

The two associations related to the mobile network under scrutiny, the \gls{3GPP} as well as the \gls{GSMA} themselves have also released some studies and recommendations concerning potential effects of and issues with the control plane. 

In reaction to the mentioned \gls{RRC} signaling \gls{DDoS} the \gls{GSMA} released some best practices \cite{gsma2011fdbestpract} intended to reduce the number of signaling messages in these circumstances. The cause of the \gls{DDoS} were in most cases mobile devices that circumvented the \gls{RRC} state transition timers and explicitly switched the radio to the idle state after a transmission was finished, re-enabling it whenever needed. This can greatly reduce the power usage but increases the number of signaling messages to be sent, and thus the load in the radio network and possibly also inside the \gls{CN}. With the presented ``Fast Dormancy'' mechanism, mobile devices are supposed to reduce the radio signaling amount while still saving energy. The implications of this mechanism on the core are not investigated.

A \gls{3GPP}-released study \cite{3gpp.22.801} also describes the diverse traffic mix originating from modern smartphones and its associated signaling problems.

The aim of the study in \gls{TS}~23.843~\cite{3gpp.23.843} is to document some of the control plane bottlenecks and attack vectors on the \gls{CN}. This also includes the interesting case of \gls{GTP-C} overload and causes for this scenario. Some approaches to alleviate the effects are also presented, but mostly targeted at the \gls{EPC}. The final study is an extension to the last one \cite{3gpp.29.807} and focuses solely on \gls{GTP-C} overload control to be included in a future version of the \gls{3GPP} architecture. Therefore, the mostly unfinished document again targets just at a future version of \gls{LTE} and provides no investigation of the actual load situation in current \gls{3G} networks.

All of the presented publications relate only to some degree to the forthcoming investigations. The combination of the aspects of \gls{CN} signaling with a statistical evaluation and load modeling of \gls{PDP} contexts should be a genuine contribution of the thesis.



%24.826 \cite{3gpp.24.826} Study on impacts on signalling between User Equipment (UE) and core network from energy saving; deals mostly with switching off cells and moving over UEs, not actual core network efficiency


%%%%%%%%%%%%%%%%%%%%%%%%%%%%%%%%%%%%%%%%%%%%%%%%%%%%%%%%%%%%%%%%%%%%%%%%%%%%%%%
%!TEX root = ../../dissertation.tex
%%%%%%%%%%%%%%%%%%%%%%%%%%%%%%%%%%%%%%%%%%%%%%%%%%%%%%%%%%%%%%%%%%%%%%%%%%%%%%%%
\section{Mobile Core Network Load}
\label{c4:sec:loaddefinition}

Now that both the basic architecture and protocols have been introduced and related work presented, a specific perspective on the \gls{CN} control plane can be defined. Existing core network measurement studies look at the control plane mostly in a rather incoherent manner. Some aspects are singled out and presented without regarding the bigger picture. In this thesis, the driving question for this research aspect is core network load. This section attempts to define a load metric 
% oder "load metrics"?
in this context. Following afterwards is a discussion of potential factors that could influence load.


%%
\subsection{Load Definition}

A traditional 
% [Citation available]?
definition of link load $\rho_{l}$ 
% Auch: utilization
is the ratio of the used bandwidth $b_u$ versus the available bandwidth $b_a$ on a link,

\begin{equation}
	\phantom{.}\rho_{l} = \frac{b_{u}}{b_{a}}\text{.}
\end{equation}

The network load $\rho_{N}$ can then simply be defined as the average load of all involved links,

\begin{equation}
	\phantom{.}\rho_{N} = \frac{\sum_{i \in N} \rho_{l,i}}{|N|}\text{.}
% i in N --> N ist eine Menge --> |N| als deren Kardinalität
\end{equation}

% Sollte man hier erwähnen, dass das rein die User Plane Load ist? Später, in der Gleichung mit dem max, gehen ja dann auch CP-Lasten mit ein.

However, the link itself is not the only component that has a limited capacity and thus can experience load. A link's capacity is determined by the transmission speed of the interfaces of the two involved network nodes. 
% Wenn die Absolutgeschwindigkeit auch wichtig ist, warum halten wir uns dann mit der relativen Last auf? Ist rho_i=100% > rho_j=100%, wenn i eine vielfache Kapazität von j hat?
% Oder sollte da stehen, "Other load metrics might be defined using different limited / scarce resources as well: ...." (so wie im Endeffekt im Folgenden).
Any involved node has also other limiting factors, mostly related to the available memory, processing power, or some other resource with fixed capacity.

In Internet core routers those other factors are mostly well known and researched. 
% [Citation available]?
The main functionality of routers is to forward packets on the basis of a routing table. This table is generated by exterior and interior gateway protocols, usually \gls{BGP} in conjunction with \gls{RIP} or \gls{OSPF}, and may grow rather large. 
% [Citation available]? Oder eine Abschätzung wenigstens? \mathscr{O}(alle IPv4-Adressen + bislang zugeteilte v6) oder so, um's mal großzügig zu machen?
The generation and maintenance of a routing table including all lookups --- which might be expensive in a large table --- incurs load on the router's available memory and processing capacity. This kind of load can be attributed to the router's control plane and occurs in addition to the user plane packet forwarding load. Still, the Internet's control plane is rather lightweight and isolated. Only minimal state is kept where necessary.
% Ich würde sogar sagen, it was designed to be lightweight / use little distributed state / ...

The situation is different in a mobile network. Here, the control and user plane are tightly coupled as discussed in Section~\ref{c4:sec:3gpparchitecture}.  Therefore, load of individual resources cannot be looked at separately anymore, and a node will be limited by any one of these resources. The load of any particular mobile core network node $\rho_{n}$ 
% Würde ja persönlich "j" als Index für den Node verwenden, um nicht mit "N" von vorher ins Gehege zu kommen.
could then be defined as the maximum of the node's link load $\rho_{l,n}$, memory load $\rho_{m,n}$, and processing load $\rho_{p,n}$,

\begin{equation}
	\phantom{,}\rho_{n} = \max(\rho_{l,n}, \rho_{m,n}, \rho_{p,n})\text{.}
\end{equation}
% (Indizes repariert, sodass sie mit der Definition von \rho_N zusammenpassen)

The \gls{CN} load $\rho_{CN}$ is then the maximum load of any of the \gls{CN} nodes,

\begin{equation}
	\phantom{.}\rho_{CN} = \max_{n \in CN}(\rho_{n})\text{.}
\end{equation}

In this definition the performance of the \gls{CN} can limited by any one of the involved core nodes\footnote{It is worth reiterating that a \textit{node} in the \gls{3G} architecture can be implemented using separate physical machines, but is still regarded one monolithic unit for the purpose of the architecture. Refer to Section~\ref{c4:sec:3gpparchitecture} for details.}. With this model definition at hand, one can now attempt to apply it to an actual mobile network and determine the individual load values. 
% Obiger Satz ist fast identisch mit dem Beginn der nächsten subsection, Load Influencing Factors, aaaaaber: ....

One additional limitation that reflects the situation in real mobile networks has to be introduced first. 
% Hä? Oben noch "attempt to apply", hier "additional....introduced first"? Was denn jetzt?
Core network nodes should be considered as black boxes. They are custom pieces of hardware sold by vendors as-is, providing no opportunity to directly monitor the inner workings of the node, including memory and processor usage. Only the network traffic entering and leaving these nodes can be directly tapped and investigated as will be described in Section~\ref{c4:sec:methodology}. The amount of signaling traffic exchanged on these observable links gives ample opportunities to indirectly infer some of the nodes' inner state and load.

With the basics of the architecture in mind, the \gls{GGSN} can be considered a top candidate for being a load bottleneck. All traffic leaving or entering the \gls{PS} domain must pass through this element, and it is involved in all \gls{CN} \gls{gtp} signaling procedures as well. Being an endpoint for the \gls{gtp} tunnel makes it responsible to sort and encapsulate incoming traffic into the corresponding user tunnel. To accomplish this a lot of state has to be kept and processed when signaling occurs. Therefore, the \gls{GGSN} will be the node under scrutiny in the trace evaluation and performance model creation.

While looking at the \gls{GGSN} may be the most obvious choice, it is by far not the only one. In addition to \gls{gtp} tunnels the \gls{SGSN} acts as the interface to the radio network as well, which involves handling \glspl{RAB} and mobility management. However, it can be assumed that the number of \glspl{SGSN} employed in a mobile network is larger than that of \glspl{GGSN}, as they are typically kept closer to the regionally distributed radio networks. This means that a single node would have to handle less mobile devices and related signaling interactions. One has also to bear in mind that the \gls{SGSN} can be completely circumvented by setting up a direct tunnel between \gls{GGSN} and \gls{RNC}.

Apart from the two gateways directly inside the traffic path there are several other nodes essential to the control plane decision making, which may be very load-sensitive as well. The \gls{HLR} for example is a central database storing all user related information which need to be retrieved any time a user needs to undergo initial authentication and authorization. Typically, the procedures the elements are involved in are fewer. Hence, this investigation concentrates just on the case of the \gls{GGSN}.


%%
\subsection{Load Influencing Factors}

With the described understanding of core network load at hand, one can now speculate on factors that could influence the load in mobile networks. Such factors will also play an important role for the following evaluation.

The first 
%and arguably one of the most important factors 
% (Removed as biased / dubious)
factor comprises the mobile devices themselves. They are the source of any user traffic and the cause for most signaling procedures, for some procedures directly but for most indirectly. However it stands to reason that the device factor is actually an aggregate of several influence subfactors, and the specific selection of subfactors and their parametrization will be unique to each device.

Specifically, the device factor covers the type of device --- which in turn is a composite of the hardware, the baseband, and the \gls{os} --- and the running applications. The usage of applications decides when the device should establish a mobile data connection, how long the connection is held, or which mobile technology takes preference. Depending on the \gls{RAT} in use, subtle behavioral and signaling differences can be expected, e.g., in the timing of the radio transmission intervals, which could influence the investigation. Some specific 
% \gls{gtp}
tunnel duration properties could stem from the \gls{os}'s \gls{IP} and transport protocol implementation. For example, \gls{TCP} timeouts might be configured to different default values influencing the duration of transport layer connections and therefore also the underlying tunnels. Some further influence factors of the protocol stack are discussed in Section~\ref{c5:sec:stack-influences}.

The actual user-traffic patterns are also generated by the applications running on the device. The application traffic spectrum ranges from low volume but extremely long duration instant messaging applications over recurring ad-retrieval patterns up to short duration burst downloads. Since the mobile application ecosystem is very versatile every device will pose a unique combination of applications. The governing factor in everything device-related is the user and her behavioral patterns. This expresses itself both in the traffic dynamics and in the mobility pattern. But it is nigh impossible to single out individual behavior in a network's traffic mix or a large network trace.
% Warum "But"? Wo ist der Einwand?

Easier to observe are the temporal statistics of large user groups, not targeting individual users but the overall effects of device usage in a certain time span, e.g., based on the time of day or the day of the week. In network user traffic analyses diurnal effects are typically very distinct, with peak traffic some time during the day and the lowest traffic shortly after midnight. Studies investigating this usually look at user traffic only. It should prove interesting to find out if the \gls{CN} control plane shows similar patterns and can be correlated to user traffic.

The second large influence factor are the mobile network's control plane state machines and their related signaling procedures. If a network-side state machine inactivity timer decides to remove an existing tunnel, signaling will occur, which suggests there will be a noticeable number of tunnels with a duration in this range. While most \gls{3G} control plane timers have default values, they are often changed by the manufacturer or network operator, and will vary from one implementation to another. It is therefore quite difficult to give any hard numbers in advance, one has to correlate such aspects with certain events in the results.



%%%%%%%%%%%%%%%%%%%%%%%%%%%%%%%%%%%%%%%%%%%%%%%%%%%%%%%%%%%%%%%%%%%%%%%%%%%%%%%%
% \subsection{Factors Influencing Tunnel Durations}

% With such a dataset available and with the intent to evaluate core network signaling by looking at tunnel durations, let's first discuss some of the factors that influence this duration.

% One factor are the mobile devices themselves. The device decides when it should establish a mobile data connection, how long the connection is held, or which mobile technology takes preference. Devices can be further differentiated by their operating system and their firmware (sometimes called \textit{baseband}) which usually takes care of much of layers 1 and 2.

% Some specific tunnel durations could stem from the TCP/IP stack implementations in the operating systems of the devices. TCP timeouts might be configured to different default values in different releases of OSs. Also, mobile network firewalls have been found to interfere with transport and application-layer timeout and keep-alive or heartbeat mechanisms on mobile devices \cite{sigcomm11middleboxes}.

% Of course, the applications that run on top of the OS and generate the actual user-traffic patterns play a role as well. An example for how applications can influence network signaling is the casual game ``Angry Birds'' mentioned before. Since the application ecosystem for smartphones is extremely rich (and grows still), we cannot pinpoint individual ones from our aggregate dataset.

% An additional factor in the picture is the user and his or her behavioral patterns. They express themselves both in the traffic dynamics and in the mobility pattern, but they are rather difficult to distinguish in such a dataset given the large amount of data and the difficulty of correctly correlating tunnel management messages. We leave this as a potential future work.

% We also expect the mobile network and its protocol implementations to express themselves in the measurements. For example, the \gls{RRC} idle timer is typically in the range of 10 to 30 minutes, which could mean there will be a large number of tunnels with a duration in this range. Such choices are usually made either by the mobile network operator or the device manufacturer and can vary from one implementation to another. It is therefore quite difficult to give any hard numbers in advance, and one has to correlate such aspects with certain events in the results.

% Based on these factors, it was decided to make a first categorization according to the device type, be it either a smartphone, a regular or feature phone, or one of the many 3G dongles or mobile routers. Second, we also differentiate based on the device operating system, if known. Both differentiating aspects should prove valuable for example in deciding if currently some phone types put more signaling load on the network and to direct measures to improve this situation. Pitfalls in this differentiation are described in the next sections.



%work in 23.843 \cite{3gpp.23.843} Study on Core Network (CN) overload solutions
%GTP-C retransmission of unacknowledged requests"
%currently: semi-static DNS based load balancing (does this apply only to LTE/SGW?)


%%%%%%%%%%%%%%%%%%%%%%%%%%%%%%%%%%%%%%%%%%%%%%%%%%%%%%%%%%%%%%%%%%%%%%%%%%%%%%%
%!TEX root = ../../dissertation.tex
%%%%%%%%%%%%%%%%%%%%%%%%%%%%%%%%%%%%%%%%%%%%%%%%%%%%%%%%%%%%%%%%%%%%%%%%%%%%%%%%
\section{Evaluation Methodology}
\label{c4:sec:methodology}

With the mobile network load defined and possible influencing factors described, the findings can now be applied to an actual mobile network. For this data from passive network traces will be employed. Before that, this section describes the monitoring setup and the captured data. This also includes a description of some methods required to examine specific device types and other device-based factors from the dataset.

While this chapter only employs passive measurements, Chapter~\ref{chap:mobilestreaming-measurements} will additionally deal with approaches to conduct meaningful active device-based measurements and set up a mobile streaming simulation testbed based on some of the results.


%%%%%%%%%%%%%%%%%%%%%%%%%%%%%%%%%%%%%%%%%%%%%%%%%%%%%%%%%%%%%%%%%%%%%%%%%%%%%%%%
\subsection{Network and Monitoring Setup}

For the analysis the \acrshort{METAWIN} monitoring system, developed in a previous third-party research project and deployed in the network of an Austrian mobile operator, is used. Detail information on this setup can be found in \cite{ricciato_2011,ricciato2006traffic}.
% Der Quelle ricciato_2011 fehlen auch ein paar Eigenschaften

\begin{figure}[htb]
	\centering
	\includegraphics[width=0.7\textwidth]{images/umts-network.pdf}
	\caption{Location of the \acrshort{METAWIN} monitoring probe in the \acrshort{3G} core network.}
% Das Interface GGSN-to-RNC hat keinen Namen?
\label{c4:fig:umtsnetwork}
\end{figure}

The measurement taps are located at the Gn interface at one \gls{GGSN} within the core network as depicted in Figure~\ref{c4:fig:umtsnetwork}. It gives access to a wide spectrum of core \gls{gtp} signaling, including the mobility and tunnel management. The system does not offer a complete packet trace, but aggregates every signaling transaction and user traffic flow down to a number of select fields. This includes \gls{gtp} \glspl{IE} such as the \gls{RAT} as well as the terminal types of the mobile clients. The latter is determinable by the \acrshort{TAC}-part of the \gls{IMEI} and will be discussed later in detail.

In the network under study, a direct link between \glspl{GGSN} and the \glspl{RNC} and circumventing the \gls{SGSN} is present. It is only used for transporting user-plane traffic under specific circumstances, and signaling procedures are still carried out in the normal way between \glspl{SGSN} and \gls{GGSN}. Therefore, only the Gn interface at \gls{GGSN} is seeing the complete core network traffic, which explains the location of the tap. The network under study has more than one \gls{GGSN} at different physical locations. The tapped \gls{GGSN} manages about half of the operator's total traffic volume in this period. 

Recording data in a live network necessitates meeting strict privacy requirements regarding the handling of user-related data. \acrshort{METAWIN} complies with this by anonymizing all user-identifying markers. Application-level payload is removed and all remaining user-specific data (e.g. the \gls{IMSI}) are non-reversibly hashed before recording. \glspl{UE} in a dataset can still be differentiated by the hashes but not traced back to the actual user. The wiretaps deployed within the monitoring system are time-synchronized with \gls{GPS}. Accordingly, the packet timestamps have an accuracy of least \SI{100}{\nano\second}.


%%%%%%%%%%%%%%%%%%%%%%%%%%%%%%%%%%%%%%%%%%%%%%%%%%%%%%%%%%%%%%%%%%%%%%%%%%%%%%%%
\subsection{Dataset Description}

Using \acrshort{METAWIN} a week-long core trace was acquired. It was recorded in April 2011, specifically beginning at Monday, \yyyymmdddate\formatdate{10}{4}{2011}, \formattime{0}{0}{0} and ending Sunday, \formatdate{17}{4}{2011}, \formattime{23}{59}{59}.

The trace includes user plane as well as control plane traffic. User plane traffic is recorded at a traffic flow granularity level with the trace containing data on \num{2.2e9} aggregated flows. No exact flow start time is given, instead it is rounded down to a \SI{2}{\hour} window with the timestamp at the beginning. 
% Was bedeutet "with the timestamp at the beginning"?
A flow entry further consists of hashed identifiers for the \gls{IMSI} and the remote server on the Internet. Besides the usual protocol and port information, the transmitted data volume is given for both link directions in number of packets and as a byte count. Additional extended information is stored about \gls{HTTP} traffic: This portion of the trace includes precise timestamps as well as the \acrshort{MIME}-type, result code, and size of the requested objected.

The recorded control plane traffic consists of \num{4.1e8} \gls{gtp} tunnel management transactions, i.e., every Create, Update, and Delete request and response. Not all of the \glspl{IE}' data is included. Most importantly, it includes the \gls{TAC}, \gls{RAT} and hashed \gls{IMSI} for the purpose of device discrimination. Also present are several timestamps with \SI{64}{\bit} precision describing the time of the request, response and the tunnel's start time. Finally, the \gls{gtp} data contains the response codes for each request. With these codes, failed transactions can be distinguished from successful ones and examined separately. Since the hashed \gls{IMEI} is consistent across the user and control plane data, both can be cross-correlated.

All trace information was exported from \acrshort{METAWIN} as pure line-based text data. For this investigation all records were fed into a \acrshort{SQL} database. Evaluations were then conducted through scripted queries on the database using Python scripts and further statistically evaluated in R.

%%%%%%%%%%%%%%%%%%%%%%%%%%%%%%%%%%%%%%%%%%%%%%%%%%%%%%%%%%%%%%%%%%%%%%%%%%%%%%%
\subsection{Device Identification}

Individual device types in a mobile network can be identified in the data through the \gls{TAC} field on every entry. The \gls{TAC}, defined in \cite{3gpp.23.003}, represents the first eight decimal digits of the \gls{IMEI} and uniquely identifies each device type. The following six digits of the \gls{IMEI} constitute the serial number of a specific device, which is omitted in the data to protect the privacy of subscribers. Due to the short length of this serial number, popular devices will often be assigned more than one \gls{TAC}, somewhat complicating the identification of certain device models.

\glspl{TAC} are assigned to individual device models by the regional members, or \gls{RBI}, 
% Sollte das \glspl{RBI} sein?
of the \gls{GSMA}, distinguished by the first two digits of the \gls{TAC}. The full allocation information is not freely available, but only to members of the \gls{GSMA}, which is not a viable option for research institutions and other interested parties. Some independent efforts have been made to collect \glspl{TAC} from devices. Most of them allow just low-volume queries for specific \glspl{TAC} for non-commercial purposes. However, one \gls{TAC} dataset is publicly available and can be used freely.\footnote{Available at: \url{http://www.mulliner.org/tacdb/}.}

The evaluation presented here uses this dataset with some additional device identifiers and classification annotations collected during the course of the investigation. With this at hand, many 
% Oder gar "most"?
of the devices associated with the flows and \gls{gtp} messages from the trace were identified and categorized.
% Iterativ oder nicht find ich irrelevant.


%%%%%%%%%%%%%%%%%%%%%%%%%%%%%%%%%%%%%%%%%%%%%%%%%%%%%%%%%%%%%%%%%%%%%%%%%%%%%%%%
\subsection{\texorpdfstring{\acrshort{TAC}}{TAC} Evaluation Validity}

It is important to know whether the information available in the \gls{TAC} dataset covers enough of the devices seen in the traces to conduct sufficiently meaningful evaluations. After all, the \gls{TAC} data is large but might still be very incomplete due to the sheer number of devices in existence.

\begin{table}
\centering
\caption{Relative \acrshort{TAC} statistics.}
\label{c4:tbl:tacstats}
	\begin{tabu}{XX[r]}
		\toprule
		\textbf{Type} & \textbf{Relative number of devices with an entry in the \gls{TAC} dataset}\\ 
		\midrule
		Total number of flows & \SI{99.72}{\percent} \\
		Ratio of total traffic & \SI{99.97}{\percent} \\
		Total number of tunnels & \SI{87.57}{\percent} \\
		Total number of \gls{gtp} signaling messages & \SI{90.95}{\percent} \\
		Number of distinct \glspl{UE} & \SI{80.93}{\percent} \\ 
		\bottomrule
	\end{tabu}
\end{table}

Table~\ref{c4:tbl:tacstats} provides statistics on the devices that could be identified in the trace. About \SI{81}{\percent} of all unique \glspl{TAC} present in the trace could be mapped to a known device. More importantly, when looking at the total number of tunnels and \gls{gtp} messages during the week, even \SI{91}{\percent} of the responsible device can be determined. Finally, the flow data shows an even clearer picture, as almost all of the devices involved can be identified.

This is an interesting result in itself, as the \SI{19}{\percent} of devices present in the dataset that could not be identified through the \gls{TAC} are the cause for only about \SI{9}{\percent} of signaling and \SI{0.3}{\percent} of total traffic. This means there is a long tail of device types in this mobile network with very little impact on the load. 
% "impact on the load" ist mbMn nur "impact on load-influencing factors"
With these results, one can be rather confident that evaluations using device discrimination based on this \gls{TAC} mapping should give viable results.


%%%%%%%%%%%%%%%%%%%%%%%%%%%%%%%%%%%%%%%%%%%%%%%%%%%%%%%%%%%%%%%%%%%%%%%%%%%%%%%%
\subsection{Device Classification}

With these device-to-\glspl{TAC} mappings available, additional meta-information can now be added to it, intended to distinguish some of the described load influencing factors. Knowing the model gives also a good knowledge of the device's category and of the \gls{os} it is running by default.\footnote{The \gls{os} actually running on the device at the time of the measurement can not be inferred on this way. But the number of devices running a different \gls{os} than the one installed by default should be relatively low.
% Würde ich bei den iDevices bspw. nicht erwarten, außer Du meinst mit "different OS" explizit nicht die Version. Dass niemand iOS --> Android updatet, liegt auf der Hand; 2.3.7 --> 4.0.2 schon weniger.
}

The device's category represents a general classification of the device and should give some initial hints on the fields of use. The devices are partitioned into smartphones, feature phones, \gls{3G} USB dongles or \gls{3G}+WiFi routers, and all other devices. The term ``feature phone'' usually points to low-end mobile phones with at least some kind of data capability, often with a physical numerical keyboard. Phones that could subjectively fall into either the smartphone or the feature phone category were generally attributed as smartphone. Not covered here are any kind of \gls{M2M} devices, because the \gls{TAC} mappings are very inconclusive and incomplete in this area.
% Außerdem sollte deren Impact gemessen an der TAC-Tabelle oben vernachlässigbar sein.

The next classification variable is the \gls{os}. Most popular in the trace were the two dominant smartphone \glspl{os}, Android and iOS, but also Symbian\footnote{While not completely accurate phones running Series 40 were also attributed to this category because of their close relationship.}, often found on feature phones, was present. Other systems of note are Blackberry OS and Windows Phone or Windows Mobile, but they occur in such a low volume in the trace that it was decided to completely neglect them and count them towards the other and unknown devices. It should also be noted that USB dongles and routers cannot be linked to any specific \gls{os} solely by the knowledge of the \gls{TAC}. Also not distinguishable are the exact release versions of the \gls{os} on a specific device. This could diminish the evaluations, as the network behavior could change noticeably between two major versions.

With this knowledge, one can even conjecture about the applications running on the device. Combining the \gls{os} with lists of the most popular applications for this platform can already give some very helpful hints on what can be expected from the traffic mix these types of devices are generating. One final possible \gls{TAC} classification could be a categorization by the phone vendor. 
% Das versteh ich jetzt nicht, außer es soll implizieren, dass eben auf iSachen nur iOS rennt usw.
However, this was not conducted because it can be safely assumed that the impact is negligible in comparison to the device type and \gls{os}.


%%%%%%%%%%%%%%%%%%%%%%%%%%%%%%%%%%%%%%%%%%%%%%%%%%%%%%%%%%%%%%%%%%%%%%%%%%%%%%%%
\subsection{Preliminary Device Statistics}

After applying the categorization to the network dataset the device composition is evaluated to get a first grasp of the network's makeup and to help understand the later investigations.

Smartphones and \gls{3G} dongles form the two largest observed shares of devices, while classic feature phones do not seem to play a major role anymore. About twice as many Android as iOS devices are present, possibly attributable either to the contractual situation of the operator or the wider price range of Android devices.

Regarding traffic, feature phones generate negligible amounts of user traffic despite still making up one tenth of the device fraction. The difference between \gls{3G} dongles and smartphones is also noteworthy. While the former cause large amounts of user plane traffic (compared to the device numbers), they are responsible for but a few core network signaling events and tunnels. This picture is reversed for smartphones.

One observation across all device types is that about \SI{14}{\percent} of all mobile devices have activated their \gls{GPRS} data service and \gls{gtp} tunnel and cause signaling traffic, but do not initiate any user plane traffic at all. It is unclear if this is an intended behavior as this will lead to an increase of the devices' power usage and of radio spectrum resources with seemingly no benefit to the user.
%!TEX root = ../../dissertation.tex
%%%%%%%%%%%%%%%%%%%%%%%%%%%%%%%%%%%%%%%%%%%%%%%%%%%%%%%%%%%%%%%%%%%%%%%%%%%%%%%%
\subsection{Statistical Methods}

As a final preparation for the evaluation all the statistical tools that will be used in the evaluation, are briefly defined in this section with material based on \cite{field2012discovering} and \cite{Knuth:1997:ACP:270146}.


%%%%%%%%%%%%%%%%%%%%%%%%%%%%%%%%%%%%%%%%%%%%%%%%%%%%%%%%%%%%%%%%%%%%%%%%%%%%%%%%
\subsubsection{Distribution Functions and Fitting}

With a distribution function, also called \gls{CDF}, a monotonous mapping of continuous values to a probability can be well represented. It is defined as the probability that a random variable $X$ is less than or equal to a value $x$,

\begin{equation}
	\phantom{.} F(x) = P(X\leq x)\text{.}
\end{equation}

Sample of real data are generally finite and not continuous. Hence, the distribution can only be approximated by an \gls{ECDF} $F_n(x)$ for values $X_1, X_2, \ldots , X_n$ and

\begin{equation}
	\phantom{.}F_n(x) = \frac{\text{number of }X_1, X_2, \ldots , X_i \leq x}{n}\text{.}
% Warum X_i? Kannst meiner Meinung nach alle bis X_n nehmen, und/oder bei Knuth nachlesen, wie man Summen mit Bedingungen verwendet (\sum_{0<i<=n} X_i <= x).
\end{equation}

One of the goals of the analysis is to break down the actual measured system to a simplified probability model. This can be conducted by attempting to match the data's \gls{ECDF} to an existing basic probability distribution, e.g., exponential Gamma, log-normal, or Weibull. 
% Hmm... DF vs CDF vs ECDF?
In order to achieve this one of several readily available matching methods can be used which rely either on closed formulas or numerical optimization. Two simple methods are \textit{Matching Moments}~\cite[pp.~99-143]{vose2000risk} and \textit{Maximum Likelihood}.
% \cite?

The former estimates parameters for a preselected distribution function by optimizing the target distribution function so that its moments converge to those of the sample data. The latter approach finds a fitting target probability function by calculating the log-likelihood of the data for a preselected distribution and maximizing the likelihood.

In such cases where none of the basic probability distributions proved to be a good fit an attempt was made to converge rational functions to the sample \gls{ECDF} with an optimization tool specialized for this case, Eureqa~\cite{eureqa_software, eureqa_paper}. While not as good as a simple model with a probability distribution, having a rational function as a description for a dataset can still enable some further statistical and queuing theoretic evaluation.


%%%%%%%%%%%%%%%%%%%%%%%%%%%%%%%%%%%%%%%%%%%%%%%%%%%%%%%%%%%%%%%%%%%%%%%%%%%%%%%%
\subsubsection{Statistical Tests}

To check the statistical goodness of the generated fits, statistical tests can be used. Generally, tests compare the values observed in an experiment to expected values following a theoretical distribution. In this case, the tests are used to validate and estimate the quality of the discovered fits to the empirical data.

First, as a simple measure, the \textit{Pearson correlation coefficient} can be facilitated, comparing the covariance and standard deviation of the empirical and fitted variables. Another possible approach is \textit{Pearson's $\chi^2$ test for independence}~\cite{doi:10.1080/14786440009463897}, 
% Sollte der Titel der Quelle mit \chi anstatt X beginnen?
which is the oldest known test and defined as

\begin{equation}
	\phantom{.}V=\sum_{i=1}^{k} \frac{{(o_i - e_i)}^2}{e_i}\text{.}
\end{equation}

This simply calculates the sum of the squared difference between the observed $o_i$ an expected values $e_i$ and adjusts each for their weight. The result can then be compared to the $\chi^2$-distribution with the same degrees of freedom
%\footnote{The degree of freedom of count experiments is one less than the number of observable categories.}
as the test for a given significance level. In most practical cases this comparison is conducted against precomputed tables with set significance levels. The data collected in this thesis is typically continuous in nature 
% Wieso dieses?
on which this test cannot be used directly. However, data could still be split into a finite number of intervals, as is done when generating a histogram, and then using the intervals as categories for the $\chi^2$ test, albeit with a certain loss of precision.

Continuous data can be checked with the \textit{Kolmogorov-Smirnov test}. First suggested by Kolmogorov in 1933~\cite{kolmogorov1933sulla} and expanded on by Smirnov in 1939~\cite{smirnov1939estimation} it is defined as

\begin{align}
	K_n^+ &= \sqrt{n} \sup_{-\infty < x < + \infty} \left( F_n(x) - F(x) \right) \\
	\shortintertext{and}
	\phantom{,}K_n^- &= \sqrt{n} \sup_{-\infty < x < + \infty} \left( F(x) - F_n(x) \right)\text{,}
\end{align}
%
for the \gls{ECDF} $F_n(x)$ and \gls{CDF} $F(x)$. Once again the results are compared against a precomputed table of values from the Kolmogorov-Smirnov distribution to test the significance of the observed results' deviation from expected values. 

Finally, every fit should in addition always undergo a \textit{Visual Inspection}. 
% Öhm... ich mein ja, aber wer sagt das?
Diagrams of the empirical and fitted distribution --- especially histograms, density, and \gls{CDF} --- should be compared and checked for specific artifacts or outliers. 



%%%%%%%%%%%%%%%%%%%%%%%%%%%%%%%%%%%%%%%%%%%%%%%%%%%%%%%%%%%%%%%%%%%%%%%%%%%%%%%%
\subsubsection{Random Sampling}

Most of the evaluations in Section~\ref{c4:sec:evaluations} use random sampling to work on a subset of the original data.  Not only does this simplify the handling of a dataset this large sets --- working on a set with two billion entries can be quite problematic --- but can even improve statistical significance, as rare outliers tend to get removed by drawing samples. 
% Warum das für die Signifikanz gut ist, sei dahingestellt. Es hilft vielleicht, dass die Tests besser passen o.ä., aber "gut" im herkömmlichen Wortsinne würde ich das nicht nennen. Ich würde das eher als ein Artefakt der Methode betrachten.
By selecting entries using a uniform distribution it is ensured that no unintentional sampling bias 
% Sagst Du oben nicht gerade, dass Sampling eine Art "intentional bias" einführt? (Widerspricht dieser Aussage hier natürlich nicht.) 
occurs. The intended evaluation is now applied onto multiple and independently drawn sample groups. If the results of every sample agree then it is also highly likely that the assumption holds for the whole data set.
% Vergleich Jackknife / Bootstrapping




%http://cran.r-project.org/web/packages/fitdistrplus/fitdistrplus.pdf


 % is now subsection of methodology


%%%%%%%%%%%%%%%%%%%%%%%%%%%%%%%%%%%%%%%%%%%%%%%%%%%%%%%%%%%%%%%%%%%%%%%%%%%%%%%%
\section{Core Network Architecture Summary}
\label{c41:sec:conclusion}

The chapter served as an introduction to mobile architectures, giving a broad overview of the \gls{3G} mobile core network control plane, and the evaluation methodology, including a description of the dataset. Modeling mobile networks cannot be achieved without first understanding many of the aspects and protocols unique and intrinsic to mobile network which differ a lot from the conventional wisdom found in wired network architectures. The definition of a control plane load as laid out in this chapter serves as an essential distinction to foregoing mobile network investigations which usually consider just the user plane. The knowledge gained will be utilized as a basis for the evaluations in the following chapter.


